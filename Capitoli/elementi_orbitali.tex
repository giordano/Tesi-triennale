\chapter{Elementi orbitali}
\label{chap:elementi-orbitali}

Nel precedente capitolo abbiamo determinato l'equazione
dell'orbita~\eqref{eq:orbita} che lega la coordinata polare $r$ all'altra
coordinata $\theta$, che dipende dal tempo. La seconda legge di Keplero, però,
ci dice che $\theta$ non varia in maniera costante con il tempo, eccetto nel
caso particolare di orbita circolare. In questo capitolo ci occuperemo della
determinazione della posizione, in un dato istante di tempo, della particella
relativa che si muove su un'orbita circolare.

\section{Equazione di Keplero}
\label{sec:equazione-keplero}

La seconda legge di Keplero permette di trovare la posizione relativa di un
corpo di un sistema binario in un qualsiasi istante di tempo. Con riferimento
alla figura % TODO: mettere figura e riferimento
$Q$ indica la posizione del corpo all'istante di tempo $t$ e $P$ la posizione
del periapside, da cui il corpo passa all'istante $t_0$. Nell'intervallo di
tempo $t-t_0$ il raggio vettore che congiunge il fuoco $F$ alla posizione del
corpo sull'ellisse si sposta da $FP$ a $FQ$, spazzando dunque l'area
$FPQ$. Allora per la seconda legge di Keplero
\begin{equation}
  \text{area } FPQ : \text{area ellisse} = t-t_0 : T,
\end{equation}
da cui otteniamo
\begin{equation}
  \text{area } FPQ = \frac{\pi ab(t-t_0)}{T}.
\end{equation}
Definendo la \emph{velocità angolare media} $\omega = 2\pi/T$ questa equazione
diventa
\begin{equation}
  \text{area } FPQ = \frac{1}{2}\omega ab(t-t_0).
\end{equation}
Conoscendo l'istante di passaggio al periapside $t_0$, il periodo di rivoluzione
$T$, il semiasse maggiore $a$ e l'eccentricità $e$, legata al semiasse minore
dalla~\eqref{eq:semiasse-minore-ellisse}, è possibile calcolare in un qualsiasi
istante di tempo $t$ l'area $FPQ$ da cui si ricava la posizione $Q$. Sebbene
questa tecnica sia molto semplice teoricamente, è poco utile nella pratica. Si
usa invece il metodo che ci apprestiamo a illustrare.

Circoscriviamo all'ellisse una circonferenza ausiliaria di raggio $a$ e
concentrica con l'ellisse, come nella figura. %TODO: mettere riferimento figura
Sia $G = (x_Q,y_G)$ l'intersezione con la circonferenza della perpendicolare
all'asse $x$ innalzata dal punto $Q = (x_Q,y_Q)$. L'angolo $G\widehat{C}P$, dove
$C$ è il centro della circonferenza, è chiamato \emph{anomalia eccentrica} e lo
indicheremo con $\psi$. Al periapside si ha $\chi = \psi = 0$ e all'apoapside
$\chi = \psi = \pi$. L'equazione cartesiana di una circonferenza di raggio $a$ e
centro l'origine del sistema di riferimento è
\begin{equation}
  x_\textup{c}^2 + y_\textup{c}^2 = a^2.
\end{equation}
L'equazione di un'ellisse con centro nell'origine del sistema e semiassi
maggiore $a$, lungo l'asse $x_\textup{e}$, e minore $b$, lungo l'asse
$y_\textup{e}$, è
\begin{equation}
    \frac{x_\textup{e}^2}{a^2} + \frac{y_\textup{e}^2}{b^2} = 1.
\end{equation}
Se gli assi dei sistemi di riferimento delle due figure sono coincidenti,
cerchiamo una relazione fra le ordinate dei punti della circonferenza e
dell'ellisse che hanno la stessa ascissa:
\begin{equation}
  x_\textup{c}^2 = x_\textup{e}^2 \iff a^2 - y_\textup{c}^2 = a^2 -
  y_\textup{e}^2a^2/b^2 \iff \abs{y_\textup{c}} = \abs{y_\textup{e}}a/b.
\end{equation}
All'ora, poiché i punti $Q$ e $G$ hanno la stesa ascissa, le loro ordinate sono
legate dalla relazione $y_G = y_Qa/b$. D'altra parte, $y_Q = r\sin\chi$ e $y_G =
a\sin\psi$, dunque
\begin{equation}
  \label{eq:r-sin-chi}
  r\sin\chi = b\sin\psi.
\end{equation}
Inoltre $\overline{FR} =
r\cos\chi$,\footnote{Un
  valore negativo di $\overline{FR}$ indica che il punto $R$ si trova alla
  sinistra di $F$ sull'asse $x$.} ma è anche $\overline{FR} = \overline{CR} -
\overline{CF} = a\cos\psi - ae$, da cui
\begin{equation}
  \label{eq:r-cos-chi}
  r\cos\chi = a(\cos\psi - e).
\end{equation}
Sommando membro a membro i quadrati delle equazioni~\eqref{eq:r-sin-chi}
e~\eqref{eq:r-cos-chi} otteniamo
\begin{equation}
  \begin{split}
    r^2 &= r^2\cos^2\chi + r^2\sin^2\chi = a^2(\cos\psi - e)^2 + b^2\sin^2\psi\\
    &= a^2\cos^2\psi-2a^2e\cos\psi+a^2e^2+a^2\sin^2\psi-a^2e^2\sin^2\psi\\
    &= a^2(1+e^2-2e\cos\psi-e^2\sin^2\psi) = a^2(1-2e\cos\psi+e^2\cos^2\psi)\\
    &= a^2(1 - e\cos\psi)^2,
  \end{split}
\end{equation}
da cui ricaviamo la seguente equazione che lega la coordinata polare $r$ con
l'anomalia eccentrica $\psi$
\begin{equation}
  \label{eq:r-anomalia-eccentrica}
  r = a(1 - e\cos\psi).
\end{equation}
Dall'equazione~\eqref{eq:terza-legge-keplero} che esprime la terza legge di
Keplero abbiamo
\begin{equation}
  \omega^2 = \frac{4\pi^2}{T^2} = \frac{GM_\textup{T}}{a^3}.
\end{equation}
Ricordando la~\eqref{eq:reciproco-semilato}, la~\eqref{eq:semilato-ellisse} e la~\eqref{eq:velocita-tangenziale} abbiamo
\begin{equation}
  \begin{split}
    r^2\dot{\chi}^2 &= \frac{l_0^2}{\mu^2p^2}(1 + e\cos\chi)^2 =
    \frac{l_0^2}{\mu^2p} \frac{(1+e\cos\chi)^2}{a(1-e^2)} \\
    &= GM_\textup{T} \frac{(1+e\cos\chi)^2}{a(1-e^2)} = \omega^2a^3
    \frac{(1+e\cos\chi)^2}{a(1-e^2)} \\
    &= \omega^2a^4 \frac{(1+e\cos\chi)^2}{a^2(1-e^2)^2}(1-e^2) =
    \frac{\omega^2a^4(1-e^2)}{r^2}.
  \end{split}
\end{equation}
Quindi, dalla~\eqref{eq:velocita-ellisse}, il quadrato della derivata temporale
di $r$ è
\begin{equation}
  \begin{split}
    \dot{r}^2 &= v^2 - r^2\dot{\chi}^2 = GM_\textup{T}
    \left(
      \frac{2}{r} - \frac{1}{a}
    \right) - \frac{\omega^2a^4(1-e^2)}{r^2} \\
        &= \omega^2a^3
    \left(
      \frac{2}{r} - \frac{1}{a}
    \right) - \omega^2a^2\frac{(1+e\cos\chi)^2}{1-e^2} \\
    &= \frac{2\omega^2a^3r-r^2\omega^2a^2-\omega^2a^4-\omega^2a^4e^2}{r^2}\\
    &= \frac{\omega^2a^2}{r^2}(a^2e^2-(r-a)^2)
  \end{split}
\end{equation}
da cui
\begin{equation}
  \toder{r}{t} = \dot{r} = \frac{\omega a}{r}\sqrt{a^2e^2 - (r-a)^2}.
\end{equation}
Per risolvere questa equazione differenziale ordinaria cambiamo la funzione
incognita da $r$ in $\psi$ effettuando la
sostituzione~\eqref{eq:r-anomalia-eccentrica}. Deriviamo
la~\eqref{eq:r-anomalia-eccentrica} rispetto al tempo
\begin{equation}
  \dot{r} = ae\dot{\psi}\sin\psi.
\end{equation}
Sostituiamo la~\eqref{eq:r-anomalia-eccentrica} nell'equazione differenziale
\begin{equation}
  r = \frac{\omega a\sqrt{a^2e^2 - a^2e^2\cos^2\psi}}{a(1-e\cos\psi)} =
  \frac{\omega ae\sin\psi}{1-e\cos\psi}.
\end{equation}
Uguagliando le due espressioni di $\dot{r}$ così ottenute ricaviamo
\begin{equation}
  \toder{\psi}{t} = \dot{\psi} = \frac{\omega}{1-e\cos\psi}.
\end{equation}
Ricordando che al perielio abbiamo $t = t_0$ e $\psi = 0$, integriamo questa
equazione differenziale fino a un generico istante di tempo $t$
  \begin{gather}
    \int_{t_0}^t \omega\dd \tau = \int_0^\psi(1-e\cos\psi')\dd \psi' \iff\\
    \phi = \omega(t - t_0) = \psi - e\sin\psi, \label{eq:keplero}
  \end{gather}
dove $\phi = \omega(t-t_0)$ è chiamata \emph{anomalia media}. Questa equazione è
detta \emph{equazione di Keplero}. In ogni fissato istante di tempo $t$, la sua
soluzione $\psi$ permette di ricavare, tramite
la~\eqref{eq:r-anomalia-eccentrica}, la coordinata polare $r$.
% TODO: completare il paragrafo mostrando la relazione fra \psi e \chi

\section{Soluzioni dell'equazione di Keplero}
\label{sec:soluzioni}

L'equazione di Keplero~\eqref{eq:keplero} è trascendente nell'incognita $\psi$
per via del termine $\sin\psi$, e non ammette soluzione analitica, ma possono
essere utilizzati diversi metodi numerici per risolverla.

\subsection{Metodo di Newton~-~Raphson}
\label{sec:newton}

\subsection{Integrali ellittici}
\label{sec:integrali-ellittici}

\subsection{Funzioni di Bessel}
\label{sec:bessel}

%%% Local Variables: 
%%% mode: latex
%%% TeX-master: "../tesi"
%%% End: 
