\chapter{Programma per la soluzione dell'equazione di Keplero}
\label{cha:soluzione-keplero}

Di seguito è riportato il codice sorgente del programma usato per risolvere
l'equazione di Keplero con il metodo numerico di Newton~-~Raphson.
% TODO: ho commentato l'inserimento del codice per accelerare i tempi di
% compilazione del documento, ricordarsi di decommentare nella versione finale!
% \lstinputlisting[language=C,%
% numbers=left,numberstyle=\tiny]{keplero/keplero.c}

Qui riportiamo inoltre il codice dello script \verb|gnuplot| utilizzato per
ottenere le figure~\ref{fig:newton-anomalia_eccentrica},
\ref{fig:newton-raggio} e~\ref{fig:newton-anomalia_vera} a partire dal file di
output generato dall'esecuzione del programma.
\lstinputlisting[language=gnuplot,
numbers=left,numberstyle=\tiny]{keplero/keplero.gnuplot}

%%% Local Variables:
%%% mode: latex
%%% TeX-master: "../tesi"
%%% End:
