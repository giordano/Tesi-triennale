\chapter{Il problema dei due corpi}
\label{chap:due-corpi}

In questo capitolo affronteremo il \emph{problema dei due corpi}, cioè lo studio
del moto di due corpi, supposti puntiformi, soggetti solamente alla mutua
interazione dovuta a forze che soddisfano il principio di azione e reazione. In
particolare ci occuperemo del \emph{problema di Keplero}, ovvero del caso in cui
l'interazione fra i due corpi è quella gravitazionale che è una forza di tipo
centrale. Il problema sarà risolto usando il formalismo newtoniano e quello
lagrangiano, di quest'ultimo presenteremo una breve introduzione. Riusciremo a
determinare analiticamente le equazioni delle orbite dei due corpi e a dedurre
alcune proprietà.

\section{Formalismo newtoniano}
\label{sec:formalismo-newton}

Consideriamo un sistema costituito da due corpi, puntiformi, di masse $m_1$ e
$m_2$ e in un fissato sistema di riferimento inerziale indichiamo con $\bm{r}_1$
e $\bm{r}_2$ i rispettivi vettori posizione. Per la seconda legge di Newton la
forza $\bm{F}_{21}$ che il corpo di massa $m_1$ esercita sul corpo di massa
$m_2$ è data da
\begin{equation}
  \label{eq:f21}
  \bm{F}_{21} = m_2\toder[2]{\bm{r}_2}{t}
\end{equation}
e la forza $\bm{F}_{12}$ che il corpo di massa $m_2$ esercita sull'altro è
\begin{equation}
  \label{eq:f12}
  \bm{F}_{12} = m_1\toder[2]{\bm{r}_1}{t}.
\end{equation}
Assumiamo per ipotesi che queste due forze soddisfino il principio di azione e
reazione nella forma forte:
\begin{equation}
  \label{eq:az-reaz}
  \bm{F}_{12} = -\bm{F}_{21}.
\end{equation}
Definiamo il vettore \emph{posizione relativa}
\begin{equation}
  \label{eq:pos-rel}
  \bm{r}=\bm{r}_2-\bm{r}_1
\end{equation}
e il \emph{centro di massa}
\begin{equation}
  \label{eq:cdm}
  \bm{r}_\textup{g} = \frac{m_1\bm{r}_1+m_2\bm{r}_2}{m_1+m_2}.
\end{equation}
Dalle equazioni~\eqref{eq:pos-rel} e \eqref{eq:cdm} possiamo ricavare $\bm{r}_1$
e $\bm{r}_2$ in funzione di $\bm{r}$ e $\bm{r}_\textup{g}$:
\begin{subequations}
  \begin{align}
    \bm{r}_1 &= \bm{r}_\textup{g} - \frac{m_2}{m_1+m_2}\bm{r} =
    \bm{r}_\textup{g} - \frac{\mu}{m_1}\bm{r},\\
    \bm{r}_2 &= \bm{r}_\textup{g} + \frac{m_1}{m_1+m_2}\bm{r} =
    \bm{r}_\textup{g} + \frac{\mu}{m_2}\bm{r},
  \end{align}
\end{subequations}
in cui abbiamo introdotto la \emph{massa ridotta} $\mu$ definita dalla metà
della media armonica fra $m_1$ e $m_2$
\begin{equation}
  \frac{1}{\mu} = \frac{1}{m_1} + \frac{1}{m_2} \iff \mu=\frac{m_1m_2}{m_1+m_2}.
\end{equation}
Osserviamo che la massa ridotta è sempre minore delle due masse, infatti
\begin{equation}
  \mu =\frac{m_1}{m_1/m_2+1} = \frac{m_2}{1+m_2/m_1}.
\end{equation}
Definiamo inoltre la \emph{massa totale} $M_\textup{T}$ del sistema
\begin{equation}
  M_\textup{T}=m_1+m_2.
\end{equation}
Dalle definizioni di massa ridotta e totale deriva che
$M_\textup{T}\mu=m_1m_2$. L'equazione del moto del centro di massa può essere
trovata semplicemente calcolando l'accelerazione di $\bm{r}_\textup{g}$:
\begin{equation}
  \toder[2]{\bm{r}_\textup{g}}{t} = \frac{1}{M_\textup{T}}
  \left(
    m_1\toder[2]{\bm{r}_1}{t} + m_2\toder[2]{\bm{r}_2}{t}
  \right) = \frac{\bm{F}_{12}+\bm{F}_{21}}{M_\textup{T}} = \bm{0},
\end{equation}
quindi il centro di massa è in quiete o si muove di moto rettilineo uniforme. A
ogni modo il sistema in cui il centro di massa è a riposo è un sistema inerziale
allora lo scegliamo come nostro nuovo sistema di riferimento e in particolare
fissiamo come origine del sistema la posizione del centro di massa. In questo
nuovo sistema le posizioni delle due masse saranno dunque descritte dai vettori
\begin{subequations} %TODO: alla fine controllare aspetto di questa equazione
  \begin{align}
    \bm{r}_1 &= -\frac{\mu}{m_1}\bm{r},\\
    \bm{r}_2 &= \frac{\mu}{m_2}\bm{r}.
  \end{align}
\end{subequations}
Dividendo ambo i membri della~\eqref{eq:f21} per $m_2$ e ambo i membri
della~\eqref{eq:f12} per $m_1$ e sottraendo membro a membro le equazioni così
ottenuti ricaviamo
\begin{equation}
  \frac{\bm{F}_{21}}{m_2}-\frac{\bm{F}_{12}}{m_1} = \bm{F}_{21}
  \left(
    \frac{1}{m_2}+\frac{1}{m_1}
  \right) = \frac{\bm{F}_{21}}{\mu} = \toder[2]{(\bm{r}_2-\bm{r}_1)}{t} =
  \toder[2]{\bm{r}}{t}
\end{equation}
in cui abbiamo ricordato che vale il principio di azione e reazione espresso
dalla~\eqref{eq:az-reaz}. L'equazione precedente può essere riscritta nella forma
\begin{equation}
  \label{eq:f21-mu}
  \bm{F}_{21}=\mu\toder[2]{\bm{r}}{t}
\end{equation}
da cui possiamo dedurre che il problema dei due corpi interagenti si è ridotto
al problema del moto di un solo corpo puntiforme fittizio, di massa $\mu$ e
posizione $\bm{r}$ nel sistema del centro di massa, all'interno campo generato
da un altro corpo puntiforme fittizio di massa $M_\textup{T}$ fisso
nell'origine.

\subsection{Problema di Keplero}

Consideriamo il caso dell'interazione gravitazionale. Se indichiamo con $M$ la
massa del corpo che genera il campo a cui è soggetto il corpo di massa $m$ e con
$\bm{r}$ il vettore diretto dal corpo di massa $M$ verso il corpo di massa $m$
allora la forza $\bm{F}_\textup{G}$ è data da
\begin{equation}
  \label{eq:legge-attrazione}
  \bm{F}_\textup{G} = -\frac{GMm}{r^2}\versor{r},
\end{equation}
in cui $G=\SI{6.674
  28(67)e-11}{\metre\tothe{3}\per\kilogram\per\second\squared}$ è la costante di
gravitazione universale~\cite{codata-costanti}. Si tratta di una forza centrale,
cioè puramente radiale, e il segno meno indica che è sempre attrattiva verso il
centro del campo. Sostituendo la~\eqref{eq:legge-attrazione}
nella~\eqref{eq:f21-mu} abbiamo
\begin{equation}
  \label{eq:forza-mu}
  \bm{F}_{21} = -\frac{GM_\textup{T}\mu}{r^2}\versor{r} =
  -\frac{Gm_1m_2}{r^2}\versor{r} = \mu\toder[2]{\bm{r}}{t}.
\end{equation}

Per una qualsiasi forza di tipo centrale il momento angolare
$\bm{L}=\bm{r}\times\bm{p}$ rispetto al centro della forza è una costante del
moto. Infatti il momento torcente $\bm{N}$ di una forza $\bm{F}$ rispetto a un
polo $O$ è definito da
\begin{equation}
  \bm{N}= \toder{\bm{L}}{t} = \bm{r}\times\bm{F}
\end{equation}
dove qui con $\bm{r}$ indichiamo il vettore congiungente il polo con il punto di
applicazione della forza. Dunque il momento della forza rispetto alla sorgente
di un campo centrale è nullo essendo in questo caso $\bm{F}$ diretto lungo
$\bm{r}$ e il momento angolare calcolato rispetto allo stesso polo è
costante.

Ritorniamo al problema di Keplero. Abbiamo appena visto che il momento angolare
della particella di massa ridotta rispetto al centro di massa del sistema è un
integrale primo del moto. Notiamo ora che, per definizione di momento angolare,
i vettori $\bm{r}$ e $\bm{L}\equiv\bm{l}_0$ sono fra loro perpendicolari. La
costanza del vettore $\bm{l}_0$, e in particolare della sua direzione, allora
implica che, nel caso $\bm{l}_0\neq\bm{0}$, il vettore $\bm{r}$, che istante per
istante indica la posizione rispetto al centro di massa della particella
fittizia di massa $\mu$, giace sempre sullo stesso piano perpendicolare a
$\bm{l}_0$. Se invece risulta $\bm{l}_0=\bm{0}$, $\bm{r}$ è parallelo alla
quantità di moto $\bm{p}$ e il moto è unidimensionale. Da qui in seguito
supporremo che il momento angolare $\bm{l}_0$ della particella di massa ridotta
sia non nullo.

Dal momento che il moto della particella di massa ridotta si svolge in un piano
possiamo utilizzare le coordinate polari $r,\theta$ per rappresentare la
posizione del corpo nel sistema di riferimento del centro di massa. Dalla
meccanica si ricava che per un moto piano l'accelerazione $\ddot{\bm{r}}$ del
vettore posizione $\bm{r}$ è data da
\begin{equation} %TODO: alla fine controllare aspetto di questa equazione
  \ddot{\bm{r}} = (\ddot{r}-r\dot{\theta}^2)\versor{r} + (r\ddot{\theta} +
  2\dot{r}\dot{\theta}) \versor{\theta}
\end{equation}
Usando le coordinate polari l'equazione vettoriale~\eqref{eq:forza-mu} può
allora essere riscritta come due equazioni scalari:
\begin{subequations}
  \begin{gather}
    -\frac{GM_\textup{T}\mu}{r^2} = \mu(\ddot{r} - r\dot{\theta}^2) \iff
    -\frac{GM_\textup{T}}{r^2}=\ddot{r}-r\dot{\theta}^2, \label{eq:forza-mu-r}\\
    0 = \mu(r\ddot{\theta} + 2\dot{r}\dot{\theta}). \label{eq:forza-mu-az}
  \end{gather}
\end{subequations}
Dalla~\eqref{eq:forza-mu-az} si trova nuovamente la costanza del modulo del
momento angolare, infatti
\begin{equation}
  0 = r\mu(r\ddot{\theta} + 2\dot{r}\dot{\theta}) = \toder{\mu
    r^2\dot{\theta}}{t} = \toder{l_0}{t}.
\end{equation}
L'area differenziale $\dd A$ spazzata dalla particella di massa ridotta che si
sposta di un angolo $\dd\theta$ è espressa da
\begin{equation}
  \dd A = \frac{r\cdot r\dd\theta}{2} = \frac{r^2\dd\theta}{2},
\end{equation}
quindi la velocità areolare $\ltoder{A}{t}$ è
\begin{equation}
  \toder{A}{t} = \frac{1}{2}r^2\dot{\theta} = \frac{1}{2}\frac{l_0}{\mu} =
  \text{costante}.
\end{equation}
Abbiamo così ottenuto la seconda legge di Keplero: \emph{il vettore
  posizione della particella rispetto al centro di massa spazza aree uguali in
  intervalli di tempo uguali}. Osserviamo che per giungere a questo risultato è
stato sufficiente utilizzare la costanza del momento angolare che deriva a sua
volta dal carattere centrale della forza. Dunque questo risultato è valido per
tutte le forze di questo tipo.

\subsubsection{Equazione dell'orbita}
\label{sec:equazione-dellorbita}

Vogliamo ora cercare un'espressione esplicita della coordinata polare $r$ in
funzione dell'altra coordinata $\theta$, che a sua volta dipenderà dal
tempo. Per fare ciò è conveniente effettuare la sostituzione $u=1/r$

\section{Formalismo lagrangiano}
\label{sec:formalismo-lagrange}



%%% Local Variables: 
%%% mode: latex
%%% TeX-master: "../tesi"
%%% End: 
