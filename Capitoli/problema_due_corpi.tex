\chapter{Il problema dei due corpi}
\label{chap:due-corpi}

In questo capitolo affronteremo il \emph{problema dei due corpi}, cioè lo studio
del moto di due corpi, supposti puntiformi, soggetti solamente alla mutua
interazione dovuta a forze che soddisfano il principio di azione e reazione. In
particolare ci occuperemo del \emph{problema di Keplero}, ovvero del caso in cui
l'interazione fra i due corpi è quella gravitazionale che è una forza di tipo
centrale. Il problema sarà risolto usando il formalismo newtoniano e quello
lagrangiano, di quest'ultimo presenteremo una breve introduzione. Riusciremo a
determinare analiticamente le equazioni delle orbite dei due corpi e a dedurre
alcune proprietà del sistema. La risoluzione del problema di Keplero permette di
descrivere, fra le altre cose, il moto di due corpi celesti, per esempio un
pianeta e la sua stella, che sono soggetti solo alla mutua interazione
gravitazionale.

\section{Formalismo newtoniano}
\label{sec:formalismo-newton}

Consideriamo un sistema costituito da due corpi, puntiformi, di masse $m_1$ e
$m_2$ e in un fissato sistema di riferimento inerziale indichiamo con $\bm{r}_1$
e $\bm{r}_2$ i rispettivi vettori posizione. Per la seconda legge di Newton la
forza $\bm{F}_{21}$ che il corpo di massa $m_1$ esercita sul corpo di massa
$m_2$ è data da
\begin{equation}
  \label{eq:f21}
  \bm{F}_{21} = m_2\toder[2]{\bm{r}_2}{t}
\end{equation}
e la forza $\bm{F}_{12}$ che il corpo di massa $m_2$ esercita sull'altro è
\begin{equation}
  \label{eq:f12}
  \bm{F}_{12} = m_1\toder[2]{\bm{r}_1}{t}.
\end{equation}
Assumiamo per ipotesi che queste due forze soddisfino il principio di azione e
reazione nella forma forte:
\begin{equation}
  \label{eq:az-reaz}
  \bm{F}_{12} = -\bm{F}_{21}.
\end{equation}
Definiamo il vettore \emph{posizione relativa}
\begin{equation}
  \label{eq:pos-rel}
  \bm{r}=\bm{r}_2-\bm{r}_1
\end{equation}
e il \emph{centro di massa}
\begin{equation}
  \label{eq:cdm}
  \bm{r}_\textup{g} = \frac{m_1\bm{r}_1+m_2\bm{r}_2}{m_1+m_2}.
\end{equation}
Dalle equazioni~\eqref{eq:pos-rel} e \eqref{eq:cdm} possiamo ricavare $\bm{r}_1$
e $\bm{r}_2$ in funzione di $\bm{r}$ e $\bm{r}_\textup{g}$:
\begin{subequations}
  \begin{align}
    \bm{r}_1 &= \bm{r}_\textup{g} - \frac{m_2}{m_1+m_2}\bm{r} =
    \bm{r}_\textup{g} - \frac{\mu}{m_1}\bm{r},\\
    \bm{r}_2 &= \bm{r}_\textup{g} + \frac{m_1}{m_1+m_2}\bm{r} =
    \bm{r}_\textup{g} + \frac{\mu}{m_2}\bm{r},
  \end{align}
\end{subequations}
in cui abbiamo introdotto la \emph{massa ridotta} $\mu$ definita dalla metà
della media armonica fra $m_1$ e $m_2$
\begin{equation}
  \frac{1}{\mu} = \frac{1}{m_1} + \frac{1}{m_2} \iff \mu=\frac{m_1m_2}{m_1+m_2}.
\end{equation}
Osserviamo che la massa ridotta è sempre minore delle due masse, infatti
\begin{equation}
  \mu =\frac{m_1}{m_1/m_2+1} = \frac{m_2}{1+m_2/m_1}.
\end{equation}
Definiamo inoltre la \emph{massa totale} $M_\textup{T}$ del sistema
\begin{equation}
  M_\textup{T}=m_1+m_2.
\end{equation}
Dalle definizioni di massa ridotta e totale deriva che
$M_\textup{T}\mu=m_1m_2$. Se, per esempio, $m_2\gg m_1$ allora $\mu\simeq m_1$ e
$M_\textup{T}\simeq m_2$. L'equazione del moto del centro di massa può essere
trovata semplicemente calcolando l'accelerazione di $\bm{r}_\textup{g}$:
\begin{equation}
  \toder[2]{\bm{r}_\textup{g}}{t} = \frac{1}{M_\textup{T}}
  \left(
    m_1\toder[2]{\bm{r}_1}{t} + m_2\toder[2]{\bm{r}_2}{t}
  \right) = \frac{\bm{F}_{12}+\bm{F}_{21}}{M_\textup{T}} = \bm{0},
\end{equation}
quindi il centro di massa è in quiete o si muove di moto rettilineo uniforme. A
ogni modo il sistema in cui il centro di massa è a riposo è un sistema inerziale
allora lo scegliamo come nostro nuovo sistema di riferimento e in particolare
fissiamo come origine del sistema la posizione del centro di massa. In questo
nuovo sistema le posizioni delle due masse saranno dunque descritte dai vettori
\begin{subequations} %TODO: alla fine controllare aspetto di questa equazione
  \begin{align}
    \bm{r}_1 &= -\frac{\mu}{m_1}\bm{r},\\
    \bm{r}_2 &= \frac{\mu}{m_2}\bm{r}.
  \end{align}
\end{subequations}
Osserviamo che se, per esempio, $m_2\gg m_1$ allora $\bm{r}_2\simeq\bm{0}$, cioè
il corpo più massivo si trova praticamente nel centro di massa, come è logico
aspettarsi dalla definizione di centro di massa. Dividendo ambo i membri
della~\eqref{eq:f21} per $m_2$ e ambo i membri della~\eqref{eq:f12} per $m_1$ e
sottraendo membro a membro le equazioni così ottenuti ricaviamo
\begin{equation}
  \frac{\bm{F}_{21}}{m_2}-\frac{\bm{F}_{12}}{m_1} = \bm{F}_{21}
  \left(
    \frac{1}{m_2}+\frac{1}{m_1}
  \right) = \frac{\bm{F}_{21}}{\mu} = \toder[2]{(\bm{r}_2-\bm{r}_1)}{t} =
  \toder[2]{\bm{r}}{t}
\end{equation}
in cui abbiamo ricordato che vale il principio di azione e reazione espresso
dalla~\eqref{eq:az-reaz}. L'equazione precedente può essere riscritta nella forma
\begin{equation}
  \label{eq:f21-mu}
  \bm{F}_{21}=\mu\toder[2]{\bm{r}}{t}
\end{equation}
da cui possiamo dedurre che il problema dei due corpi interagenti si è ridotto
al problema del moto di un solo corpo puntiforme fittizio, di massa $\mu$ e
posizione $\bm{r}$ nel sistema del centro di massa, all'interno campo generato
da un altro corpo puntiforme fittizio di massa $M_\textup{T}$ fisso
nell'origine.

\subsection{Problema di Keplero}

Consideriamo il caso dell'interazione gravitazionale. Se indichiamo con $M$ la
massa del corpo che genera il campo a cui è soggetto il corpo di massa $m$ e con
$\bm{r}$ il vettore diretto dal corpo di massa $M$ verso il corpo di massa $m$
allora la forza $\bm{F}_\textup{G}$ è data da
\begin{equation}
  \label{eq:legge-attrazione}
  \bm{F}_\textup{G} = -\frac{GMm}{r^2}\versor{r},
\end{equation}
in cui $G=\SI{6.674
  28(67)e-11}{\metre\tothe{3}\per\kilogram\per\second\squared}$ è la costante di
gravitazione universale~\cite{codata-costanti}. Si tratta di una forza centrale,
cioè puramente radiale, e il segno meno indica che è sempre attrattiva verso il
centro del campo. Sostituendo la~\eqref{eq:legge-attrazione}
nella~\eqref{eq:f21-mu} abbiamo
\begin{equation}
  \label{eq:forza-mu}
  \bm{F}_{21} = -\frac{GM_\textup{T}\mu}{r^2}\versor{r} =
  -\frac{Gm_1m_2}{r^2}\versor{r} = \mu\toder[2]{\bm{r}}{t}.
\end{equation}

Per una qualsiasi forza di tipo centrale il momento angolare
$\bm{L}=\bm{r}\times\bm{p}$ rispetto al centro della forza è una costante del
moto. Infatti il momento torcente $\bm{N}$ di una forza $\bm{F}$ rispetto a un
polo $O$ è definito da
\begin{equation}
  \bm{N}= \toder{\bm{L}}{t} = \bm{r}\times\bm{F}
\end{equation}
dove qui con $\bm{r}$ indichiamo il vettore congiungente il polo con il punto di
applicazione della forza. Dunque il momento della forza rispetto alla sorgente
di un campo centrale è nullo essendo in questo caso $\bm{F}$ diretto lungo
$\bm{r}$ e il momento angolare calcolato rispetto allo stesso polo è
costante.

Ritorniamo al problema di Keplero. Abbiamo appena visto che il momento angolare
della particella di massa ridotta rispetto al centro di massa del sistema è un
integrale primo del moto. Notiamo ora che, per definizione di momento angolare,
i vettori $\bm{r}$ e $\bm{L}\equiv\bm{l}_0$ sono fra loro perpendicolari. La
costanza del vettore $\bm{l}_0$, e in particolare della sua direzione, allora
implica che, nel caso $\bm{l}_0\neq\bm{0}$, il vettore $\bm{r}$, che istante per
istante indica la posizione rispetto al centro di massa della particella
fittizia di massa $\mu$, giace sempre sullo stesso piano perpendicolare a
$\bm{l}_0$. Se invece risulta $\bm{l}_0=\bm{0}$, $\bm{r}$ è parallelo alla
quantità di moto $\bm{p}$ e il moto è unidimensionale. Da qui in seguito
supporremo che il momento angolare $\bm{l}_0$ della particella di massa ridotta
sia non nullo.

Dal momento che il moto della particella di massa ridotta si svolge in un piano
possiamo utilizzare le coordinate polari $r,\theta$ per rappresentare la
posizione del corpo nel sistema di riferimento del centro di massa. Dalla
meccanica si ricava che per un moto piano la velocità $\dot{\bm{r}}$ e
l'accelerazione $\ddot{\bm{r}}$ del vettore posizione $\bm{r}$ sono date da
\begin{subequations}
  \begin{align} %TODO: alla fine controllare aspetto di questa equazione
    \dot{\bm{r}}  &= \dot{r}\versor{r} +
    r\dot{\theta}\versor{\theta}, \label{eq:velocita-polare}\\
    \ddot{\bm{r}} &= (\ddot{r}-r\dot{\theta}^2)\versor{r} + (r\ddot{\theta} +
    2\dot{r}\dot{\theta}) \versor{\theta}.
  \end{align}
\end{subequations}
Usando le coordinate polari l'equazione vettoriale~\eqref{eq:forza-mu} può
allora essere riscritta come due equazioni scalari:
\begin{subequations}
  \begin{gather}
    -\frac{GM_\textup{T}\mu}{r^2} = \mu(\ddot{r} - r\dot{\theta}^2) \iff
    -\frac{GM_\textup{T}}{r^2}=\ddot{r}-r\dot{\theta}^2, \label{eq:forza-mu-r}\\
    0 = \mu(r\ddot{\theta} + 2\dot{r}\dot{\theta}). \label{eq:forza-mu-az}
  \end{gather}
\end{subequations}
Dalla~\eqref{eq:forza-mu-az} si trova nuovamente la costanza del modulo del
momento angolare $l_0 = \mu r^2\dot{\theta}$, infatti
\begin{equation}
  0 = r\mu(r\ddot{\theta} + 2\dot{r}\dot{\theta}) = \toder{\mu
    r^2\dot{\theta}}{t} = \toder{l_0}{t}.
\end{equation}
L'area differenziale $\dd A$ spazzata dalla particella di massa ridotta che si
sposta di un angolo infinitesimo $\dd\theta$ è espressa da
\begin{equation}
  \dd A = \frac{r\cdot r\dd\theta}{2} = \frac{r^2\dd\theta}{2},
\end{equation}
quindi la velocità areolare $\ltoder{A}{t}$ è
\begin{equation}
  \toder{A}{t} = \frac{1}{2}r^2\dot{\theta} = \frac{1}{2}\frac{l_0}{\mu} =
  \text{costante}.
\end{equation}
Abbiamo così ottenuto la seconda legge di Keplero: \emph{il vettore posizione
  della particella rispetto al centro di massa spazza aree uguali in intervalli
  di tempo uguali}. Osserviamo che per giungere a questo risultato è stato
sufficiente utilizzare la costanza del momento angolare che deriva a sua volta
dal carattere centrale della forza, pertanto questo risultato è valido per tutte
le forze di questo tipo.

\subsubsection{Equazione dell'orbita}
\label{sec:equazione-dellorbita}

Vogliamo ora cercare un'espressione esplicita della coordinata polare $r$ in
funzione dell'altra coordinata $\theta$, che a sua volta dipenderà dal
tempo. Per fare ciò è conveniente effettuare la sostituzione $u=1/r$. Espandiamo
le derivate temporali di $r$:
\begin{subequations}
  \begin{align}
    \dot{r} &= \toder{r}{t} = \toder{r}{\theta}\toder{\theta}{t} =
    \toder{(1/u)}{\theta}\dot{\theta} =
    -\frac{\dot{\theta}}{u^2}\toder{u}{\theta}
    = -\frac{l_0}{\mu}\toder{u}{\theta}, \label{eq:derivata1-r}\\
    \begin{split}
      \ddot{r} &= \toder[2]{r}{t} = \toder{}{\theta} \left( \toder{r}{t}
      \right)\toder{\theta}{t} = \toder{}{\theta} \left(
        -\frac{l_0}{\mu}\toder{u}{\theta} \right)\dot{\theta} =
      -\frac{l_0\dot{\theta}}{\mu}\toder[2]{u}{\theta} \\
      &= - \left(
        \frac{l_0}{\mu} \right)^2u^2\toder[2]{u}{\theta}. \label{eq:derivata2-r}
    \end{split}
  \end{align}
\end{subequations}
Sostituendo la~\eqref{eq:derivata2-r} nella~\eqref{eq:forza-mu-r} abbiamo
\begin{equation}
  -GM_\textup{T}u^2 = -\left(
        \frac{l_0}{\mu} \right)^2u^2\toder[2]{u}{\theta} - \frac{l_0^2u^3}{\mu^2}.
\end{equation}
$r$ è la distanza della particella di massa ridotta dal centro di massa, quindi
assume valori finiti e il suo reciproco $u$ non è mai nullo, allora
nell'equazione precedente possiamo dividere ambo i membri per $u^2$ e otteniamo
\begin{equation}
  \toder[2]{u}{\theta} + u = \frac{GM_\textup{T}\mu^2}{l_0^2}.
\end{equation}
Questa è una equazione differenziale ordinaria lineare a coefficienti costanti
del secondo ordine, chiamata \emph{equazione di Binet}. Per semplificare i
calcoli poniamo
\begin{equation}
  \label{eq:rec-semilato}
  \frac{GM_\textup{T}\mu^2}{l_0^2} = \frac{1}{p}
\end{equation}
e così l'integrale generale dell'equazione di Binet è
\begin{equation}
  \label{eq:sol-binet}
  u(\theta) = c_0\cos\theta + c_1\sin\theta + \frac{1}{p}
\end{equation}
con $c_0$ e $c_1$ costanti determinate dalle condizioni iniziali. La soluzione
può essere riscritta in maniera diversa se imponiamo
\begin{subequations}
  \label{eq:cost-binet}
  \begin{align}
    c_0 &= \frac{e}{p}\cos\tilde\omega, \\
    c_1 &= \frac{e}{p}\sin\tilde\omega,
  \end{align}
\end{subequations}
dove $e$ (un'ampiezza) e $\tilde\omega$ (una fase) svolgono il ruolo di nuove
costanti di integrazione. Le~\eqref{eq:cost-binet} possono essere invertite:
\begin{subequations}
  \begin{align}
    \tilde\omega &= \arctan \frac{c_1}{c_0}, \\
    e &= \frac{c_0p}{\cos\omega}.
  \end{align}
\end{subequations}
Dunque fissate $c_0$ e $c_1$, $\tilde\omega$ e $e$ sono univocamente
determinate. In questo modo la~\eqref{eq:sol-binet} diventa
\begin{equation}
  u(\theta) = \frac{1}{p}(1+e\cos(\theta-\tilde\omega))
\end{equation}
e ricordando che $u=1/r$ giungiamo all'equazione polare dell'orbita della
particella di massa ridotta nel sistema di riferimento del centro di massa
\begin{equation}
  \label{eq:orbita}
  r(\theta) = \frac{p}{1+e\cos(\theta-\tilde\omega)}.
\end{equation}

\subsubsection{Classificazione geometrica delle orbite}
\label{sec:class-geom-orbite}

La~\eqref{eq:orbita} è l'equazione polare di una conica. La costante $e$ è
chiamata \emph{eccentricità} della conica, la costante $p$ invece è detta
\emph{semilato retto}. Entrambe possono assumere valori non negativi. A seconda
dei diversi valori dell'eccentricità le orbite possono essere classificate nel
seguente modo:
\begin{itemize}
\item $e<1$: \emph{ellisse}. Nel caso particolare $e=0$ si ha una
  \emph{circonferenza};
\item $e=1$: \emph{parabola};
\item $e>1$: \emph{iperbole}.
\end{itemize}
\begin{figure}
  \centering
  \input{Immagini/gnuplot/ellisse}
  \caption{Ellisse in coordinate polari. L'origine del sistema di riferimento è
    il centro di massa (CDM) che coincide con uno dei fuochi dell'ellisse, $a$ è
    il semiasse maggiore, $b$ è il semiasse minore, $p$ è il semilato retto, $c$
    è la distanza fra il centro dell'ellisse e i fuochi.}
  \label{fig:ellisse}
\end{figure}
Occupiamoci in dettaglio delle orbite chiuse. L'unica conica chiusa è l'ellisse
e nel caso dell'ellisse la~\eqref{eq:orbita} è l'equazione con centro in uno dei
fuochi, quindi se uno dei due corpi è molto più massivo dell'altro (per esempio
il Sole e un pianeta che gli orbita intorno) abbiamo ricavato la prima legge di
Keplero: \emph{I pianeti, considerati puntiformi, descrivono orbite ellittiche
  di cui il Sole occupa uno dei fuochi}. Dalla geometria analitica è noto che
per l'ellisse il semilato retto è uguale a
\begin{equation}
  \label{eq:semilato-ellisse}
  p = a(1-e^2),
\end{equation}
dove $a$ è il semiasse maggiore dell'ellisse (vedi la
figura~\ref{fig:ellisse}). Osserviamo inoltre che $r(\pi/2)=p$. Inserendo
la~\eqref{eq:rec-semilato} nella~\eqref{eq:semilato-ellisse} possiamo ricavare
l'eccentricità dell'ellisse
\begin{equation}
  \label{eq:eccentricita-ellisse}
  e = \sqrt{1-\frac{1}{p}} = \sqrt{1-\frac{l_0^2}{GM_\textup{T}\mu^2a}}.
\end{equation}
Si noti che, poiché $l_0\neq 0$ per ipotesi, si ha $0\leq e<1$, coerentemente
con la classificazione illustrata prima.
%TODO: illustrare i legami fra p, a, e, b. Definire periapside e apoapside (per
%ora l'ho fatto più avanti, ricordati di sistemare!)

\subsubsection{Velocità}
\label{sec:velocita}
\begin{figure}
  \centering
  \input{Immagini/gnuplot/velocita}
  \caption{Andamento del modulo quadro della velocità del corpo di massa ridotta
  in funziona dell'angolo $\theta$. Abbiamo posto $\tilde\omega=0$.}
  \label{fig:velocita}
\end{figure}
Calcoliamo il modulo quadro della velocità del corpo fittizio di massa ridotta
durante la sua orbita, data dalla~\eqref{eq:velocita-polare}: $v(\theta)^2 =
\norm{\dot{\bm{r}}(\theta)}^2 = \dot{r}^2(\theta) +
r^2(\theta)\dot{\theta}^2$. Ricordando le formule~\eqref{eq:derivata1-r}
e~\eqref{eq:sol-binet} e che $l_0=\mu r^2\dot{\theta}$ abbiamo:
\begin{subequations}
  \begin{align}
    \dot{r}(\theta) &= -\frac{l_0}{\mu}\toder{u}{\theta} = -\frac{l_0}{p}
    \left(
      -\frac{1}{p}e\sin(\theta-\tilde\omega)
    \right) = \frac{l_0}{\mu p}e\sin(\theta-\tilde\omega), \\
    r(\theta)\dot{\theta} &= \frac{l_0}{\mu}u = \frac{l_0}{\mu
      p}(1+e\cos(\theta-\tilde\omega)).
  \end{align}
\end{subequations}
Quindi, mettendo insieme le relazioni appena trovate, la~\eqref{eq:rec-semilato}
e la~\eqref{eq:orbita} risulta
\begin{equation}
  \begin{split}
    v^2(\theta) &= \dot{r}^2(\theta) + r^2(\theta)\dot{\theta}^2 \\
    &= \left(
      \frac{l_0}{\mu p}
    \right)^2(e^2\sin^2(\theta-\tilde\omega) + 1 + 2e\cos(\theta-\tilde\omega) +
    e^2\cos^2(\theta-\tilde\omega)) \\
    &= \left(
      \frac{l_0}{\mu p}
    \right)^2(1+2e\cos(\theta-\tilde\omega)+e^2) \\
    &= \left(
      \frac{l_0}{\mu p}
    \right)^2(2(1+e\cos(\theta-\tilde\omega)) + (e^2-1)) \\
    &= \frac{l_0^2}{\mu^2p}
    \left(
      2\frac{1+e\cos(\theta-\tilde\omega)}{p} - \frac{1-e^2}{p}
    \right) = \frac{l_0^2}{\mu^2 p}
    \left(
      \frac{2}{r(\theta)} - \frac{1}{a}
    \right) \\
    &= \frac{l_0^2}{\mu^2}\frac{GM_\textup{T}\mu^2}{l_0^2}
    \left(
      \frac{2}{r(\theta)} - \frac{1}{a}
    \right) = GM_\textup{T}
    \left(
      \frac{2}{r(\theta)} - \frac{1}{a}
    \right).
  \end{split}
\end{equation}
Questa relazione lega il modulo quadro della velocità alla distanza della
particella fittizia dal centro di massa. Si può osservare in particolare che
$v^2(\theta)$ è proporzionale a $1/r(\theta)$, quindi il modulo della velocità è
massimo quando il corpo di massa ridotta si trova nel punto più vicino al centro
di massa, chiamato \emph{periapside}, e vale %TODO: definire per bene il
                                %periapside e spiegare che r(w)=a*(1-e).
\begin{equation}
  v^2(\tilde\omega) = \frac{GM_\textup{T}}{a}\frac{1+e}{1-e}.
\end{equation}
Inoltre il modulo quadro della velocità sarà minimo quando la distanza dal
centro di massa è massima, cioè quando il corpo fittizio è in \emph{apoapside}
\begin{equation}
  v^2(\tilde\omega+\pi) = \frac{GM_\textup{T}}{a}\frac{1-e}{1+e}.
\end{equation}
Questi risultati sono in accordo con la seconda legge di Keplero. L'andamento di
$v^2$ in funzione della coordinata $\theta$ è riportato nella
figura~\ref{fig:velocita}.

\subsubsection{Energia e classificazione energetica delle orbite}
\label{sec:energ-class}

Possiamo ora calcolare anche l'energia meccanica del sistema costituito dai due
corpi
\begin{equation}
  \label{eq:energia}
  \begin{split}
    E &= T + U = \frac{1}{2}\mu v^2(\theta) - \frac{GM_\textup{T}\mu}{r(\theta)}
    = \frac{1}{2}\mu(\dot{r^2(\theta)} + r^2(\theta)\dot{\theta}^2) -
    \frac{GM_\textup{T}\mu}{r(\theta)} \\
    &= \frac{1}{2}GM_\textup{T}\mu
    \left(
      \frac{2}{r(\theta)} - \frac{1}{a}
    \right) - \frac{GM_\textup{T}\mu}{r(\theta)} = -\frac{GM_\textup{T}\mu}{2a}.
  \end{split}
\end{equation}
Dunque l'energia è una costante del moto e dipende solo dalle masse $m_1$ e
$m_2$ dei due corpi e dal semiasse maggiore dell'ellisse descritta dal loro moto
relativo, ma non dall'eccentricità di questa. Naturalmente il fatto che
l'energia si conservi non deve sorprendere perché abbiamo supposto per ipotesi
che il sistema fosse isolato e che l'unica forza agente fosse quella
gravitazionale. %TODO: svolgere esercizio su dE in funzione di da.
Dalla~\eqref{eq:energia} possiamo osservare che il moto relativo è equivalente a
un moto unidimensionale e con un potenziale efficace
\begin{equation}
  U_\textup{eff}(r(\theta)) = \frac{1}{2}\mu r^2(\theta)\dot{\theta}^2 -
  \frac{GM_\textup{T}\mu}{r(\theta)} = \frac{1}{2}\frac{l_0^2}{\mu r^2(\theta)} -
  \frac{GM_\textup{T}\mu}{r(\theta)}.
\end{equation}
\begin{figure} %TODO: scegliere valori migliori per il grafico
  \centering
  % coefficienti che compaiono nel potenziale efficace
\pgfmathsetmacro{\a}{25}
\pgfmathsetmacro{\b}{50}
% funzione potenziale efficace
\pgfmathdeclarefunction{ueff}{1}{\pgfmathparse{\a/#1^2-\b/#1}}
%% valori dell'energia
% energia E_0 = minimo di U_eff, il minimo si ha nel punto r = 2a/b e
% U_eff(2a/b) = -b^2/(4a)
\pgfmathsetmacro{\ezero}{-\b^2/(4*\a)}
\pgfmathsetmacro{\euno}{-20} % energia E_1
\pgfmathsetmacro{\edue}{15} % energia E_2

\draw[->] (0,-30) -- (0,30) node[above] {$U_{\textup{eff}}$}; % asse y
\draw[->] (0,0) node[left] {$O$} -- (8,0) node[right] {$r(\theta)$}; %asse x
% grafico della funzione potenziale efficace
\draw[name path=U] [domain=0.4:8,samples=100] plot (\x,{ueff(\x)});
\draw[dashed,name path={E0}] (0,\ezero) node[left] {$E_0$} -- +(8,0);
\draw[dashed,name path={E1}] (0,\euno) node[left] {$E_1$} -- +(8,0);
\draw[dashed] (0,\edue) node[left] {$E_2$} -- +(8,0);
% individuo il punto di intersezione fra il potenziale e l'energia E_0
\path[name intersections={of=U and {E0},by={A}}];
\draw[loosely dashed] (A) -- +(0,-\ezero) node[above] {$r_0$};
% individuo i punti di intersezione fra il potenziale e l'energia E_1
\path[name intersections={of=U and {E1},by={B,C}}];
\draw[loosely dashed] (B) -- +(0,-\euno) node[above] {$r_{\textup{min}}$}
                      (C) -- +(0,-\euno) node[above] {$r_{\textup{max}}$};

%%% Local Variables:
%%% mode: latex
%%% TeX-master: "../../tesi"
%%% End:

  \caption{Andamento del potenziale efficace nel problema di Keplero in funzione
    della distanza dal centro di massa.}
  \label{fig:potenziale-efficace}
\end{figure}
Il termine $l_0/(2\mu r^2(\theta))$ è chiamato \emph{potenziale
  centrifugo}. Nella figura~\ref{fig:potenziale-efficace} è riportato il grafico
della curva del potenziale efficace. Per $r=r_0=l_0/(GM_\textup{T}\mu^2$ si ha
un punto di minimo per $U_\textup{eff}$ e risulta $U_\textup{eff}(r_0) =
-G^2M_\textup{T}^2\mu^3/(2l_0^2)$. La differenza fra l'energia e la curva di
$U_\textup{eff}$ dà $\mu\dot{r}^2/2$, quindi il moto è ammesso solo in quelle
regioni in cui $E-U_\textup{eff}\geq 0$, cioè:
\begin{enumerate}
\item se $E = E_0 = G^2M_\textup{T}^2\mu^3/(2l_0^2)$ risulta $\dot{r}=0$, cioè
  $r(t) = r_0 = \text{costante}$. L'orbita è dunque circolare e il moto
  circolare uniforme con frequenza $\omega_0 = l_0/(\mu r_0^2)$;
\item se $E = E_1 \in \mathopen{]}E_0,0\mathclose{[}$, il moto è limitato fra le
  distanze apsidali $r_\textup{min}$ e $r_\textup{max}$ e l'orbita è un'ellisse;
\item se $E = E_2 \geq 0$ il moto è limitato inferiormente ma non
  superiormente. Si può far vedere che se $E_2 = 0$ l'orbita descritta è una
  parabola, se $E_2>0$ si ha un'iperbole.
\end{enumerate}
Confrontando la~\eqref{eq:eccentricita-ellisse} e la~\eqref{eq:energia} possiamo
ricavare una relazione fra l'energia, nel caso $E<0$, e l'eccentricità
dell'ellisse
\begin{equation}
  e = \sqrt{1+\frac{2El_0^2}{G^2M_\textup{T}^2\mu^3}}.
\end{equation}
Notiamo che più è negativa l'energia e minore è l'eccentricità dell'ellisse,
cioè maggiormente legato è il sistema. %TODO

\section{Formalismo lagrangiano}
\label{sec:formalismo-lagrange}



%%% Local Variables: 
%%% mode: latex
%%% TeX-master: "../tesi"
%%% End: 
