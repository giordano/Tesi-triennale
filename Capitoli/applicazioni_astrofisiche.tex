\chapter{Applicazioni astrofisiche}
\label{chap:applicazioni}

\section{Il sistema binario Sgr~A*~-~S2}
\label{sec:sgra}

La terza legge di Keplero permette di determinare la massa di un corpo celeste
se sono noti i parametri orbitali e la massa di un altro corpo con cui
costituisce un sistema binario.

\begin{figure}
  \centering
  \includegraphics[width=7cm]{Immagini/orbite_sgra}
  \caption[Orbite di alcune delle stelle che intorno al buco nero
  Sgr~A*]{Rappresentazione delle orbite di alcune delle stelle che orbitano
    intorno al buco nero. La figura, tratta da \textcite{2009ApJ...692.1075G}, è
    centrata in Sgr~A*}
  \label{fig:orbite-sgra}
\end{figure}
Si suppone che la regione Sagittarius~A* (Sgr~A*), nel centro della nostra
galassia, sia sede di un buco nero supermassivo, cioè con una massa oltre $10^6$
volte più grande di quella del Sole. Intorno a questo buco nero orbitano
numerose stelle e le orbite di alcune di esse possono essere osservate nella
Figura~\ref{fig:orbite-sgra}. La stella più importante per i nostri scopi è S2
(chiamata a volte S0-2 e di massa circa \SI{15}{\solarmass}), poiché fra le
stelle che orbitano intorno al buco nero è quella che ha il più breve periodo di
rivoluzione, pari a circa $15$ anni, e fra le stelle di questa regione a breve
periodo è la più luminosa, quindi più facile da individuare. Le osservazioni
astronomiche di S2 sono cominciate nel 1992 e da pochi anni ha completato
un'intera rivoluzione a partire da quella data, quindi è una ricca fonte di
informazioni per lo studio del buco nero. L'orbita di S2 intorno al buco nero
può essere considerata con buona approssimazione kepleriana quindi possiamo
utilizzare la~\eqref{eq:terza-legge-keplero} per stimare la massa
$M_\textup{BH}$ del buco nero. \textcite{2008ApJ...689.1044G} hanno studiato il
moto di S2 ricavando i dati riportati nella
Tabella~\ref{tab:parametri-orbitali-S2}.
\begin{table}
  \centering
  \caption[Parametri orbitali della stella S2]{Parametri orbitali della stella
    S2. $R_0$ è la distanza dalla Terra, $P$ è il periodo di rivoluzione, $a$
    è il semiasse maggiore dell'orbita ed $e$ la sua eccentricità,
    $R_\textup{min}$ è la distanza di periapside e $M_{\textup{S}2}$ è la
    massa della stella}
  \label{tab:parametri-orbitali-S2}
  \begin{tabular}{lc}
    \toprule
    Grandezza & Valore \\
    \midrule
    $R_0$ & \SI{7.96}{\kilo\parsec} \\
    $P$ & \SI{15.86}{\year} \\
    $a$ & \SI{126.5}{\milli\arcsecond} \\
    $e$ & $0.8970$ \\
    $R_\textup{min}$ & \SI{0.535}{\milli\parsec} \\
    $M_{\textup{S}2}$ & circa \SI{15}{\solarmass} \\
    \bottomrule
  \end{tabular}
\end{table}
Il semiasse maggiore dell'orbita è espresso in millesimi di arcosecondo, questo
valore può essere convertito in parsec, conoscendo la distanza $R_0$ della Terra
dal corpo, con la seguente relazione
\begin{equation}
  a [\si{\parsec}] = \frac{a [\si{\arcsecond}] \cdot R_0 [\si{\parsec}] \cdot
    \pi}{3600 \cdot 180} = \SI{4.88e-3}{\parsec}.
\end{equation}
Poiché $M_\textup{BH}$ è sicuramente molto più grande di $M_{\textup{S}2}$
possiamo porre $M_\textup{T} \approx M_\textup{BH}$
nella~\ref{eq:terza-legge-keplero} e otteniamo
\begin{equation}
  M_\textup{BH} \approx M_\textup{T} = \frac{4\pi^2a^3}{GP^2} =
  \SI{4.06e6}{\solarmass}.
\end{equation}
In realtà quella calcolata non è esattamente la massa del buco nero ma tutta la
massa che, nel piano dell'orbita di S2, è contenuta nella circonferenza di
raggio $R_\textup{min}$ e centro nel fuoco. Metodi più elaborati per la stima
della massa del buco nero possono essere trovati in
\textcite{2008ApJ...689.1044G} e \textcite{2009ApJ...692.1075G}.

Gli astronomi continuano a studiare il moto di S2 poiché sperano di osservare
delle deviazioni dall'orbita puramente kepleriana previste dalla teoria della
relatività le quali permetterebbero di effettuare una diversa stima della massa
del buco nero.

\section{I pianeti extrasolari}
\label{sec:extrasolari}

\section{La funzione di massa. Sistemi binari X}
\label{sec:funzione-massa}

Vediamo ora un nuovo modo per stimare la massa di un corpo celeste sfruttando la
terza legge di Keplero. Sappiamo che
\begin{equation}
  \label{eq:terza-legge-keplero2}
  G(m_1 + m_2)  P^2 = 4\pi^2a^3,
\end{equation}
dove $a$ è il semiasse maggiore dell'ellisse descritta dalla particella
relativa. Per ragioni di similitudine, il rapporto $a_1/a$, con $a_1$ semiasse
maggiore dell'orbita descritta dal corpo di massa $m_1$, è uguale al rapporto
$\norm{\bm{r}_1/\bm{r}}$, cioè dalla~\eqref{eq:r1-nel-cdm}
\begin{equation}
  \label{eq:semiasse-m1}
  a_1 = \frac{\mu}{m_1}a = \frac{m_2}{m_1 + m_2}a.
\end{equation}
Quindi, sostituendo la~\eqref{eq:semiasse-m1}
nella~\eqref{eq:terza-legge-keplero2} abbiamo
\begin{equation}
  GP^2\frac{m_2^3}{(m_1 + m_2)^3} = 4\pi^2a_1^3.
\end{equation}
Nelle osservazioni spettroscopiche non possono essere misurati separatamente il
semiasse maggiore $a_1$ e l'angolo di inclinazione $i$, ma la proiezione di
$a_1$ nel piano del cielo data da $a_1\sin i$. Moltiplicando ambo i membri per
$\sin^3 i$ e portando al secondo membro tutte le quantità misurabili risulta
\begin{equation}
  \label{eq:valore-funzione-massa}
  \frac{(m_2\sin i)^3}{(m_1 + m_2)^2} = \frac{4\pi^2}{GP^2}(a_1\sin i)^3.
\end{equation}
Il primo membro dell'equazione prende il nome di \emph{funzione di massa} per il
corpo $1$
\begin{equation}
  f_1(m_1,m_2,i) = \frac{(m_2\sin i)^3}{(m_1 + m_2)^2}.
\end{equation}
Il valore della funzione di massa è noto quando si misurano il periodo orbitale
$P$ e il semiasse maggiore proiettato $a_1\sin i$. È possibile fare questo, in
particolare misurare $a_1\sin i$, solo nei sistemi visuali, quelli cioè in cui
entrambi i corpi che costituiscono sono visibili. Vedremo più avanti come si può
calcolare la funzione di massa negli altri casi. Ragionando in maniera analoga
per il corpo di massa $m_2$ possiamo definire la funzione di massa $f_2$
\begin{equation}
  f_2(m_1,m_2,i) \equiv \frac{(m_1\sin i)^3}{(m_1 + m_2)^2} =
  \frac{4\pi^2}{GP^2}(a_2\sin i)^3,
\end{equation}
con $a_2$ semiasse maggiore dell'ellisse descritta dal corpo di massa
$m_2$. Conoscendo solo le funzioni di massa non è possibile determinare
univocamente le due masse se l'angolo di inclinazione non è noto. Sarà quindi
necessaria un'altra equazione per poter fissare i valori di tutte e tre queste
grandezze. Tuttavia, dato un qualsiasi valore di $m_1$ e $i$, la funzione di
massa $f_1$ fornisce il minimo valore della massa $m_2$. Allo stesso modo, il
valore di $f_2$ è un limite inferiore per la massa del corpo $1$.

Utilizziamo la funzione di massa per stimare la massa di un corpo non visibile
che costituisce insieme alla stella variabile supergigante HD 226868 il sistema
binario
Cygnus-X1.\footnote{I dati riportati di seguito sono presi
  da~\textcite{melia:astrophysics}.} Poiché il corpo non emette nella banda
ottica sicuramente non è una stella. È noto che l'eccentricità del sistema è
molto piccola ($e \lesssim 0.02$), quindi possiamo considerare l'orbita
circolare. Misure basate sull'effetto Doppler forniscono la velocità orbitale
proiettata $v_1$ della stella e risulta $v_1 =
\SI{75}{\kilo\metre\per\second}$. Inoltre le variazioni periodiche del flusso
misurato della stella fornisce il periodo di rotazione $P =
\SI{5.6}{\day}$. Possiamo scrivere la velocità orbitale proiettata come
\begin{equation}
  \label{eq:velocità-proiettata}
  v_1 = \frac{2\pi}{P}a_1\sin i,
\end{equation}
con $a_1$ semiasse maggiore dell'orbita della stella. Inserendo
la~\eqref{eq:velocità-proiettata} nella~\eqref{eq:valore-funzione-massa} abbiamo
\begin{equation}
  f_1(m_1,m_2,i) \equiv \frac{(m_2\sin i)^3}{(m_1 + m_2)^2} = \frac{v_1^3P}{2\pi
    G}.
\end{equation}
Poiché per questo sistema si hanno delle eclissi l'angolo di inclinazione vale
$i \simeq \pi/2$, quindi $\sin \simeq 1$. Da osservazioni nella banda ottica si
è trovato
\begin{equation}
  f_1 = \SI{0.252(10)}{\solarmass}.
\end{equation}
Si stima che la stella abbia una massa $m_1 \gtrsim \SI{8.5}{\solarmass}$,
quindi la sua compagna ha una massa
\begin{equation}
  m_2 \gtrsim \SI{4}{\solarmass}.
\end{equation}
Poiché questo valore è maggiore del limite di Chandrasekhar % TODO: mettere un
                                % riferimento bibliografico.
$M_\textup{Ch} \simeq \SI{3}{\solarmass}$ per una stella di neutroni in
rotazione, dobbiamo concludere che il corpo non visibile del sistema Cygnus-X1 è
un buco nero.

%%% Local Variables:
%%% mode: latex
%%% TeX-master: "../tesi"
%%% End:
