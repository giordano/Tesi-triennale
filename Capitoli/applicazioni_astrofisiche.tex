\chapter{Applicazioni astrofisiche}
\label{chap:applicazioni}

\section{Il sistema binario Sgr~A*~-~S2}
\label{sec:sgra}

La terza legge di Keplero permette di determinare la massa di un corpo celeste
se sono noti i parametri orbitali e la massa di un altro corpo con cui
costituisce un sistema binario.

\begin{figure}
  \centering
  \includegraphics[width=7cm]{Immagini/orbite_sgra}
  \caption[Orbite di alcune delle stelle che orbitano intorno al buco nero
  Sgr~A*]{Rappresentazione delle orbite di alcune delle stelle che orbitano
    intorno al buco nero. La figura, tratta da \textcite{2009ApJ...692.1075G}, è
    centrata in Sgr~A*}
  \label{fig:orbite-sgra}
\end{figure}
Si suppone che la regione Sagittarius~A* (Sgr~A*), nel centro della nostra
galassia, sia sede di un buco nero supermassivo, cioè con una massa oltre $10^6$
volte più grande di quella del Sole. Intorno a questo buco nero orbitano
numerose stelle e le orbite di alcune di esse possono essere osservate nella
Figura~\ref{fig:orbite-sgra}. La stella più importante per i nostri scopi è S2
(chiamata a volte S0-2 e di massa circa \SI{15}{\solarmass}), poiché fra le
stelle che orbitano intorno al buco nero è quella che ha il più breve periodo di
rivoluzione, pari a circa $15$ anni, e fra le stelle di questa regione a breve
periodo è la più luminosa, quindi più facile da individuare. Le osservazioni
astronomiche di S2 sono cominciate nel 1992 e da pochi anni ha completato
un'intera rivoluzione a partire da quella data, quindi è una ricca fonte di
informazioni per lo studio del buco nero. L'orbita di S2 intorno al buco nero
può essere considerata con buona approssimazione kepleriana quindi possiamo
utilizzare la~\eqref{eq:terza-legge-keplero} per stimare la massa
$M_\textup{BH}$ del buco nero. \textcite{2008ApJ...689.1044G} hanno studiato il
moto di S2 ricavando i dati riportati nella
Tabella~\ref{tab:parametri-orbitali-S2}.
\begin{table}
  \centering
  \caption[Parametri orbitali della stella S2]{Parametri orbitali della stella
    S2. $R_0$ è la distanza dalla Terra, $P$ è il periodo di rivoluzione, $a$
    è il semiasse maggiore dell'orbita ed $e$ la sua eccentricità,
    $R_\textup{min}$ è la distanza di periapside e $M_{\textup{S}2}$ è la
    massa della stella}
  \label{tab:parametri-orbitali-S2}
  \begin{tabular}{lc}
    \toprule
    Grandezza         & Valore                       \\
    \midrule
    $R_0$             & \SI{7.96}{\kilo\parsec}      \\
    $P$               & \SI{15.86}{\year}            \\
    $a$               & \SI{126.5}{\milli\arcsecond} \\
    $e$               & $0.8970$                     \\
    $R_\textup{min}$  & \SI{0.535}{\milli\parsec}    \\
    $M_{\textup{S}2}$ & circa \SI{15}{\solarmass}    \\
    \bottomrule
  \end{tabular}
\end{table}
Il semiasse maggiore dell'orbita è espresso in millesimi di arcosecondo, questo
valore può essere convertito in parsec, conoscendo la distanza $R_0$ della Terra
dal corpo, con la seguente relazione
\begin{equation}
  a [\si{\parsec}] = \frac{a [\si{\arcsecond}] \cdot R_0 [\si{\parsec}] \cdot
    \pi}{3600 \cdot 180} = \SI{4.88e-3}{\parsec}.
\end{equation}
Poiché $M_\textup{BH}$ è sicuramente molto più grande di $M_{\textup{S}2}$
possiamo porre $M_\textup{T} \approx M_\textup{BH}$
nella~\eqref{eq:terza-legge-keplero} e otteniamo
\begin{equation}
  M_\textup{BH} \approx M_\textup{T} = \frac{4\pi^2a^3}{GP^2} =
  \SI{4.06e6}{\solarmass}.
\end{equation}
In realtà quella calcolata non è esattamente la massa del buco nero ma tutta la
massa che, nel piano dell'orbita di S2, è contenuta nella circonferenza di
raggio $R_\textup{min}$ e centro nel fuoco. Metodi più elaborati per la stima
della massa del buco nero possono essere trovati in
\textcite{2008ApJ...689.1044G} e \textcite{2009ApJ...692.1075G}.

Gli astronomi continuano a studiare il moto di S2 poiché sperano di osservare
delle deviazioni dall'orbita puramente kepleriana previste dalla teoria della
relatività generale le quali permetterebbero di effettuare una stima
indipendente della massa del buco nero.

\section{La funzione di massa. Sistemi binari X}
\label{sec:funzione-massa}

Vediamo ora un nuovo modo per stimare la massa di un corpo celeste sfruttando la
terza legge di Keplero. Sappiamo che
\begin{equation}
  \label{eq:terza-legge-keplero2}
  G(m_1 + m_2)  P^2 = 4\pi^2a^3,
\end{equation}
dove $a$ è il semiasse maggiore dell'ellisse descritta dalla particella
relativa. Per ragioni di similitudine, il rapporto $a_1/a$, con $a_1$ semiasse
maggiore dell'orbita descritta dal corpo di massa $m_1$, è uguale al rapporto
$\norm{\bm{r}_1/\bm{r}}$, cioè dalla~\eqref{eq:r1-nel-cdm}
\begin{equation}
  \label{eq:semiasse-m1}
  a_1 = \frac{\mu}{m_1}a = \frac{m_2}{m_1 + m_2}a.
\end{equation}
Quindi, sostituendo la~\eqref{eq:semiasse-m1}
nella~\eqref{eq:terza-legge-keplero2} abbiamo
\begin{equation}
  GP^2\frac{m_2^3}{(m_1 + m_2)^3} = 4\pi^2a_1^3.
\end{equation}
Nelle osservazioni spettroscopiche non possono essere misurati separatamente il
semiasse maggiore $a_1$ e l'angolo di inclinazione $i$, ma la proiezione di
$a_1$ nel piano del cielo data da $a_1\sin i$. Moltiplicando ambo i membri per
$\sin^3 i$ e portando al secondo membro tutte le quantità misurabili risulta
\begin{equation}
  \label{eq:valore-funzione-massa}
  \frac{(m_2\sin i)^3}{(m_1 + m_2)^2} = \frac{4\pi^2}{GP^2}(a_1\sin i)^3.
\end{equation}
Il primo membro dell'equazione prende il nome di \emph{funzione di massa} per il
corpo $1$
\begin{equation}
  f_1(m_1,m_2,i) = \frac{(m_2\sin i)^3}{(m_1 + m_2)^2}.
\end{equation}
Il valore della funzione di massa è noto quando si misurano il periodo orbitale
$P$ e il semiasse maggiore proiettato $a_1\sin i$. È possibile fare questo, in
particolare misurare $a_1\sin i$, solo nei sistemi binari visuali, quelli cioè
in cui entrambi i corpi sono visibili. Vedremo più avanti come si può calcolare
la funzione di massa negli altri casi. Ragionando in maniera analoga per il
corpo di massa $m_2$ possiamo definire la funzione di massa $f_2$
\begin{equation}
  f_2(m_1,m_2,i) \equiv \frac{(m_1\sin i)^3}{(m_1 + m_2)^2} =
  \frac{4\pi^2}{GP^2}(a_2\sin i)^3,
\end{equation}
con $a_2 = am_1/(m_1+m_2)$ semiasse maggiore dell'ellisse descritta dal corpo di
massa $m_2$. Conoscendo solo le funzioni di massa non è possibile determinare
univocamente le due masse se l'angolo di inclinazione non è noto. Sarà quindi
necessaria un'altra equazione per poter fissare i valori di tutte e tre queste
grandezze. Tuttavia, dato un qualsiasi valore di $m_1$ e $i$, la funzione di
massa $f_1$ fornisce il minimo valore della massa $m_2$. Allo stesso modo, il
valore di $f_2$ è un limite inferiore per la massa del corpo $1$.

Utilizziamo la funzione di massa per stimare la massa di un corpo non visibile
che costituisce insieme alla stella variabile supergigante HD 226868 il sistema
binario
Cygnus-X1.\footnote{I dati riportati di seguito sono presi
  da~\textcite[212]{melia:astrophysics}.} Poiché il corpo non emette nella banda
ottica sicuramente non è una stella. È noto che l'eccentricità del sistema è
molto piccola ($e \lesssim 0.02$), quindi possiamo considerare l'orbita
circolare. Misure basate sull'effetto Doppler forniscono la velocità orbitale
proiettata $v_1$ della stella e risulta $v_1 =
\SI{75}{\kilo\metre\per\second}$. Inoltre le variazioni periodiche del flusso
misurato della stella fornisce il periodo di rotazione $P = \SI{5.6}{\day}$.

Possiamo scrivere la velocità orbitale proiettata come
\begin{equation}
  \label{eq:velocità-proiettata}
  v_1 = \frac{2\pi}{P}a_1\sin i,
\end{equation}
con $a_1$ semiasse maggiore dell'orbita della stella. Inserendo
la~\eqref{eq:velocità-proiettata} nella~\eqref{eq:valore-funzione-massa} abbiamo
\begin{equation}
  f_1(m_1,m_2,i) \equiv \frac{(m_2\sin i)^3}{(m_1 + m_2)^2} = \frac{v_1^3P}{2\pi
    G}.
\end{equation}
Per questo sistema si hanno delle eclissi e, come vedremo nel
paragrafo~\ref{sec:extrasolari}, in questo caso l'angolo di inclinazione vale
$i \simeq \pi/2$, da cui $\sin \simeq 1$. Da osservazioni nella banda ottica si
è trovato
\begin{equation}
  f_1 = \SI{0.252(10)}{\solarmass}.
\end{equation}
Si stima che la stella abbia una massa $m_1 \gtrsim \SI{8.5}{\solarmass}$,
quindi la sua compagna ha una massa
\begin{equation}
  m_2 \gtrsim \SI{4}{\solarmass}.
\end{equation}
Poiché questo valore è maggiore di $\SI{3.2}{\solarmass}$, valore massimo della
massa di una stella di neutroni in rotazione (si
veda~\textcite{1974PhRvL..32..324R}), dobbiamo concludere che il corpo non
visibile del sistema Cygnus-X1 è un buco nero.

\section{I pianeti extrasolari}
\label{sec:extrasolari}

Molti pianeti extrasolari costituiscono, almeno in prima approssimazione, un
sistema binario insieme alla stella intorno alla quale orbitano e dallo studio
delle eclissi della stella dietro al pianeta è possibile ricavare delle
caratteristiche dei due corpi. Qui ci limiteremo a studiare alcune proprietà del
fenomeno dell'eclissi.

\begin{figure}
  \centering
  \begin{tikzpicture}
    % coordinata x del centro del pianeta (uguale al semiasse maggiore perché ho
    % posto il centro della stella nell'origine)
    \pgfmathsetmacro{\a}{8}
    \pgfmathsetmacro{\runo}{2} % raggio stella = 2
    \pgfmathsetmacro{\rdue}{1} % raggio pianeta = 1
    \pgfmathsetmacro{\i}{acos(sqrt(1-\rdue*\rdue/(\a*\a)))} % angolo i

    \coordinate (O1) at (0,0); % centro stella
    \coordinate (O2) at (\a,0); % centro pianeta
    \draw[thick] (O1) circle (\runo); % disco stella
    \node at ($(O1) + (2.3,-1.2)$) {$m_1$};
    \draw[thick] (O2) circle (\rdue); % disco pianeta
    \node at ($(O2) + (-1.1,-1)$) {$m_2$};
    \draw[->] (O1) -- +(0,3) node[above] {$z$}; % asse z
    \draw[->] (O1) -- +(10,0) node[right] {$x$}; % asse x
    % asse x''
    \draw[->] (O1) -- ($10*({cos(\i)},{sin(\i)})$) node[right] {$x''$};
    % angolo di inclinazione
    \draw[dashed] ($(O1) + (0,2.5)$) to[out=0,in=90+\i] node[right=5]
                  {$i_{\textup{min}}$} ($2.5*({cos(\i)},{sin(\i)})$);
    % distanza fra i centri di pianeta e stella
    \draw[<->] ($(O1) + (0,-0.2)$) -- node[fill=white] {$a$} ($(O2)+(0,-0.2)$);
    \draw (O1) -- node[left] {$r_\star$} +($-\runo*({cos(45)},{sin(45)})$);
    \draw (O2) -- node[right] {$r$} +($\rdue*({cos(90+\i)},{sin(90+\i)})$);
  \end{tikzpicture}
  \caption[Angolo di inclinazione minimo sotto il quale un osservatore può
  vedere
  un'eclissi]{Angolo di inclinazione minimo sotto il quale un osservatore può
    vedere un'eclissi. La direzione della linea di vista è l'asse $x''$. Il moto
    della stella e del pianeta si svolge nel piano ortogonale al piano della
    figura}
  \label{fig:minimo-angolo-eclissi}
\end{figure}
Vediamo sotto quali condizioni è possibile per un osservatore esterno vedere
un'eclissi della stella dietro al pianeta studiato. Per semplicità assumiamo che
le orbite dei due corpi siano circolari, quindi $e \approx 0$, che questi siano
perfettamente sferici e che possano essere trascurati gli effetti dovuti
all'atmosfera. Con riferimento alla Figura~\ref{fig:minimo-angolo-eclissi}, il
corpo di massa $m_1$ è la stella, di raggio $r_\star$, e quello di massa $m_2$ è
il pianeta, il cui raggio è $r$. La distanza fra i centri dei due corpi vale
$a$. Se indichiamo con $i \in \mathopen{[}0, \pi/2\mathclose{]}$ l'angolo di
inclinazione sotto il quale l'osservatore vede il sistema, l'angolo minimo che
permette all'osservatore di assistere a un'eclissi è quello rappresentato nella
figura, cioè con l'asse $x''$ passante per il centro della stella e tangente
alla superficie del pianeta. Dunque
\begin{equation}
  \cos(\pi/2 - i) = \frac{\sqrt{a^2 - r^2}}{a} = \sqrt{1 - \frac{r^2}{a^2}}.
\end{equation}
Generalmente si ha $a \gg r_\star \gg r$, quindi $\cos(\pi/2 - i) \approx 1$,
cioè $i \approx \pi/2$.

\begin{figure}
  \centering
  \begin{tikzpicture}[scale=0.6,font=\footnotesize]
    \pgfmathsetmacro{\rdue}{1} % raggio pianeta = 1
    \pgfmathsetmacro{\runo}{4*\rdue} % raggio stella = 4

    \coordinate (O) at (0,0); % centro stella
    \draw (O) circle (\runo); % disco stella
    \draw (O) -- node[right] {$r_\star$} +(0,-\runo);
    \draw (-\runo-\rdue,0) -- node[above] {$r$}
          +($-\rdue*({cos(45)},{sin(45)})$);
    \foreach \x in {-5,-3,-1,1,3,5} % varie posizioni del pianeta
      \draw (\x,0) circle (\rdue);
    \draw (-8,-\runo+1) node[left] {luminosità} -- (-\runo-\rdue,-\runo+1) --
          (-\runo+\rdue,-\runo-1) -- (\runo-1,-\runo-1) --
          (\runo+\rdue,-\runo+1) -- (8,-\runo+1); % curva di luce
    \draw[dashed] (-\runo-\rdue,0) -- (-\runo-\rdue,-\runo-2) node[below]
                  {$t_{\textup{ii}}$};
    \draw[dashed] (-\runo,-\runo) -- (-\runo,-\runo-2) node[below]
                  {$t_{\textup{mi}}$};
    \draw[dashed] (-\runo+\rdue,0) -- (-\runo+\rdue,-\runo-2) node[below]
                  {$t_{\textup{if}}$};
    \draw[dashed] (\runo-\rdue,0) -- (\runo-\rdue,-\runo-2) node[below]
                  {$t_{\textup{ei}}$};
    \draw[dashed] (\runo,-\runo) -- (\runo,-\runo-2) node[below]
                  {$t_{\textup{mf}}$};
    \draw[dashed] (\runo+\rdue,0) -- (\runo+\rdue,-\runo-2) node[below]
                  {$t_{\textup{ef}}$};
    \draw[->] (-8,-\runo-2) -- (8,-\runo-2) node[right] {$t$}; % asse del tempo
  \end{tikzpicture}
  \caption[Transito di un pianeta davanti alla stella compagna nel caso
  $i =
  \pi/2$]{Transito di un pianeta davanti alla stella compagna nel caso $i =
    \pi/2$. Davanti alla stella di raggio $r_\star$, fissa, sono riportate le
    posizioni del pianeta, di raggio $r$, in diversi istanti. La curva sotto lo
    schema mostra l'andamento della luminosità del sistema in funzione del tempo
    $t$}
  \label{fig:schema-transito}
\end{figure}
Nella Figura~\ref{fig:schema-transito} è rappresentata una schematizzazione di
un'eclissi della stella, nel caso in cui $i = \pi/2$. L'osservatore è fisso
davanti alla stella e vede il pianeta che le passa davanti, in vari
istanti. Quando il pianeta non copre la stella la luminosità del sistema è
massima, quando inizia a nasconderla parzialmente (istante di
\emph{ingresso iniziale}, $t_{\textup{ii}}$) la luminosità decresce fino a
raggiungere un minimo nell'istante in cui si trova completamente davanti alla
stella (\emph{ingresso finale}, $t_{\textup{if}}$). La luminosità rimane minima
per tutto il tempo in cui il pianeta si trova davanti alla stella,
successivamente aumenta quando il pianeta esce parzialmente dalla stella
(\emph{egresso iniziale}, $t_{\textup{ei}}$) e ritorna massima non appena il
pianeta non copre più la stella (\emph{egresso finale}, $t_{\textup{ef}}$). Gli
istanti fin qui definiti sono detti \emph{punti di contatto}. Esistono poi i
\emph{punti di mezzo ingresso} che sono quelli in cui la curva di luminosità
raggiunge il valore intermedio fra il massimo e il minimo. In particolare
abbiamo il \emph{mezzo ingresso iniziale} ($t_{\textup{mi}}$) quando la
luminosità sta diminuendo e il \emph{mezzo ingresso finale} ($t_{\textup{mf}}$)
quando la luminosità ricomincia ad aumentare. Definiamo la
\emph{durata dell'eclissi} $\Delta t$ come il tempo che intercorre fra
l'ingresso iniziale e l'egresso finale:
$\Delta t = t_{\textup{ef}} - t_{\textup{ii}}$. La velocità di rivoluzione del
pianeta intorno alla stella è $P/(2\pi a)$, avendo indicato con $P$ il periodo
di rivoluzione del pianeta intorno alla stella, e, sempre nell'approssimazione
$a \gg r_\star \gg r$, lo spazio percorso dal pianeta fra gli istanti
$t_{\textup{ii}}$ e $t_{\textup{ef}}$ è $2(r_\star + r)$. Allora la durata
dell'eclissi è
\begin{equation}
  \Delta t \approx \frac{P}{2\pi a}2(r_\star + r) = \frac{P(r_\star + r)}{\pi
    a}.
\end{equation}

\begin{figure}
  \centering
  \subfloat[][Posizione del pianeta, davanti alla stella, in corrispondenza dei
  punti di contatto\label{fig:transito-i-non-90.a}]
  {\begin{tikzpicture}[scale=0.7,font=\footnotesize]
      \pgfmathsetmacro{\rdue}{1} % raggio pianeta = 1
      \pgfmathsetmacro{\runo}{4*\rdue} % raggio stella = 4
      \pgfmathsetmacro{\acosi}{2.5} % a*cos(i) = 2.5
      % modulo della coordinata x dell'ingresso finale (e dell'egresso finale)
      \pgfmathsetmacro{\xingresso}{sqrt((\runo+\rdue)^2-\acosi^2)}
      % modulo della coordinata x dell'egresso iniziale (e dell'ingrsso finale)
      \pgfmathsetmacro{\xegresso}{sqrt((\runo-\rdue)^2-\acosi^2)}

      \draw (\runo,0) arc (0:180:\runo); % semi-disco stella
      \draw (-6,\acosi) -- (6,\acosi); % asse passante lungo equatori dei pianeti
      \draw (-\runo,0) -- (\runo,0); % diametro stella
      \draw[<->] (0,0) -- node[sloped,above,fill=white] {$a\cos i$} (0,\acosi);
      \draw[->] (-6,-1) -- (6,-1) node[right] {$t$}; % asse del tempo
      \draw (-\xingresso,\acosi) circle (\rdue) % posizione di ingresso iniziale
            (\xingresso,\acosi)  circle (\rdue) % posizione di egresso finale
            (0,0) -- node[sloped,below,fill=white] {$r_\star + r$}
            (-\xingresso,\acosi);
      \draw[dashed] ($-\xingresso*(1,0) + (0,\acosi)$) -- +(0,{-\acosi-1})
                    node[below] {$t_{\textup{ii}}$}
                    ($\xingresso*(1,0) + (0,\acosi)$) -- +(0,{-\acosi-1})
                    node[below] {$t_{\textup{ef}}$};
      \draw (-\xegresso,\acosi) circle (\rdue) % posizione di ingresso finale
            (\xegresso,\acosi)  circle (\rdue) % posizione di egresso iniziale
            (0,0) -- node[sloped,below,fill=white] {$r_\star-r$}
            (\xegresso,\acosi);
      \draw[dashed] ($-\xegresso*(1,0) + (0,\acosi)$) -- +(0,{-\acosi-1})
                    node[below] {$t_{\textup{if}}$}
                    ($\xegresso*(1,0) + (0,\acosi)$) -- +(0,{-\acosi-1})
                    node[below] {$t_{\textup{ei}}$};
    \end{tikzpicture}} \qquad
  \subfloat[][Geometria del sistema\label{fig:transito-i-non-90.b}]
  {\begin{tikzpicture}[scale=0.7,font=\footnotesize]
      \coordinate (O1) at (0,0); % centro stella
      \coordinate (O2) at (8,0); % centro pianeta
      \draw[thick] (O1) circle (2); % disco stella (r=2)
      \node at ($(O1) + (2.3,-1.2)$) {$m_1$};
      \draw[thick] (O2) circle (1); % disco pianeta (r=1)
      \node at ($(O2) + (-1.1,-1)$) {$m_2$};
      \draw[->] (O1) -- (0,3) node[above] {$z$}; % asse z
      \draw[->] (O1) -- (10,0) node[right] {$x$}; % asse x
      \draw[->] (O1) -- (10,1) node[right] {$x''$}; % asse x''
      % retta parallela all'asse x'' e passante per il centro del pianeta
      \draw (0,-0.8) -- (10,0.2);
      \draw[<->] let \n1 = {0.8/10.1},
                     \n2 = {-8/10.1}
                 in
                 (O1) -- node (P) {} (\n1,\n2); % distanza fra le due rette
      \draw[->] (-0.8,0.5) node[above] {$a\cos i$} to[out=-90,in=180] (P);
      \draw[dashed] let \n1 = {atan(0.1)} in
                        ($(O1) + (0,2.5)$) to[out=0,in=90+\n1] node[right=5]
                        {$i$} ($cos(\n1)*(2.5,0) + sin(\n1)*(0,2.5)$);
      \draw (O1) -- node[left] {$r_\star$} +($-2*({cos(45)},{sin(45)})$);
      \draw (O2) -- node[left] {$r$} +($({cos(45)},{-sin(45)})$);
    \end{tikzpicture}}
  \caption{Transito di un pianeta davanti alla stella compagna nel caso
    $i \neq \pi/2$}
  \label{fig:transito-i-non-90}
\end{figure}
Come abbiamo visto, per poter osservare un'eclissi deve aversi $i \approx
\pi/2$. Se l'angolo di inclinazione $i$ non vale esattamente $\pi/2$,
l'osservatore non vede il centro del pianeta passare davanti all'equatore della
stella ma leggermente più spostato, come mostrato nella
Figura~\ref{fig:transito-i-non-90.a}. In questa figura, della stella è
rappresentata solo metà del disco visibile. Per calcolare lo spostamento
apparente del pianeta rispetto all'equatore si può vedere la
Figura~\ref{fig:transito-i-non-90.b}. Da semplici calcoli trigonometrici si
ricava che lo spostamento apparente vale $a \cos i$. Le definizioni dei punti di
contatto e di mezzo ingresso valgono anche per questo caso. Osservando la
Figura~\ref{fig:transito-i-non-90.a} si ricava, sempre usando la trigonometria,
che lo spazio percorso dal pianeta fra gli istanti $t_{\textup{ii}}$ e
$t_{\textup{ef}}$ è circa $2\sqrt{(r_\star + r)^2 - a^2\cos^2 i}$, quindi la
durata dell'eclissi in questo caso è
\begin{equation}
  \Delta t \approx \frac{P}{2\pi a} 2\sqrt{(r_\star + r)^2 - a^2\cos^2 i} =
  \frac{P}{\pi} \sqrt{\left(\frac{r_\star + r}{a}\right)^2 - \cos^2 i}.
\end{equation}
Per $i = \pi/2$ si riottiene il risultato precedente.

\begin{figure}
  \centering
  \subfloat[][$\sqrt{r_\star^2 - r^2} \leq d \leq r_\star +
  r$\label{fig:area-coperta.a}]
  {\begin{tikzpicture}[scale=1.1,font=\footnotesize]
      \coordinate (O1) at (0,0); % centro stella
      \coordinate (O2) at (2.5,0); % centro pianeta
      \draw[name path=S,thick] (O1) circle (2); % disco stella (r=2)
      \draw[name path=P,thick] (O2) circle (1); % disco pianeta (r=1)
      % individuo punti di intersezione fra dischi di pianeta e stella
      \path[name intersections={of=S and P,by={A,B}}];
      \draw[dashed] (O1) -- (A) -- (O2) -- node[below,sloped] {$r$} (B) --
                    node[above,sloped] {$r_\star$}  (O1);
      \draw (A) -- (B);
      \draw[->] ($(B) - (0,0.5)$) node[left,fill=white] {$S_1$} to
                [out=45,in=-80] (1.9,-0.1);
      \draw[->] ($(A) + (0,0.5)$) node[right] {$S_2$}
                to[out=180] (1.7,0.1);
      \draw let \n1 = {acos((1-4+2.5*2.5)/(2*1*2.5))} in
                ($(O2) - cos(\n1)*(0.2,0) + sin(\n1)*(0,0.2)$) arc
                (180-\n1:180+\n1:0.2);
      \draw[->] ($(O2) + (-0.2,0.5)$) node[right] {$\theta_2$}
                to[out=180,in=180] (2.3,0);
      \draw let \n2 = {acos((4-1+2.5*2.5)/(2*2*2.5))} in
                ($(O1) + cos(\n2)*(0.3,0) - sin(\n2)*(0,0.3)$) arc
                (-\n2:\n2:0.3);
      \draw[->] ($(O1) + (0.5,-0.5)$) node[below] {$\theta_1$} to[in=0] (0.3,0);
    \end{tikzpicture}} \qquad
  \subfloat[][$r_\star - r \leq d <
  \sqrt{r_\star^2 - r^2}$\label{fig:area-coperta.b}]
  {\begin{tikzpicture}[scale=1.1,font=\footnotesize]
      \coordinate (O1) at (0,0); % centro stella
      \coordinate (O2) at (1.5,0); % centro pianeta
      \draw[name path=S,thick] (O1) circle (2); % disco stella (r=2)
      \draw[name path=P,thick] (O2) circle (1); % disco pianeta (r=1)
      % individuo punti di intersezione fra dischi di pianeta e stella
      \path[name intersections={of=S and P,by={A,B}}];
      \draw[dashed] (O1) -- (A) -- (O2) -- node[below,sloped] {$r$} (B) --
                    node[above,sloped] {$r_\star$} (O1);
      \draw (A) -- (B);
      \draw[->] ($(B) - (0,0.5)$) node[left,fill=white] {$S_1$} to
                [out=0,in=-60] (1.9,-0.1);
      \draw[->] ($(A) + (0,0.5)$) node[right] {$S_2$} to[out=180] (1.2,0.3);
      \draw let \n1 = {acos((1-4+1.5*1.5)/(2*1*1.5))} in
                ($(O2) - cos(\n1)*(0.2,0) + sin(\n1)*(0,0.2)$) arc
                (180-\n1:180+\n1:0.2);
      \draw[->] ($(O2) + (1,0.6)$) node[right] {$\theta_2$}
                to[out=200,in=0] (1.3,0);
      \draw let \n2 = {acos((4-1+1.5*1.5)/(2*2*1.5))} in
            ($(O1) + cos(\n2)*(0.3,0) - sin(\n2)*(0,0.3)$) arc (-\n2:\n2:0.3);
      \draw[->] ($(O1) + (0.4,-0.4)$) node[below] {$\theta_1$} to[in=0] (0.3,0);
    \end{tikzpicture}}
  \caption{Schema della sovrapposizione fra i dischi di una stella e di un
    pianeta compagno durante un transito}
  \label{fig:area-coperta}
\end{figure}
Calcoliamo l'area del disco della stella coperta dal pianeta durante
l'eclissi. La distanza che prenderemo in considerazione non sarà la distanza
reale $a$ fra i due corpi ma quella proiettata $d$ nel piano del cielo
dell'osservatore e introdotta nel paragrafo~\ref{sec:geometria-sistema}. Se in
un certo istante di tempo si ha $d > r_\star + r$, i due corpi non sono
sovrapposti nel cielo dell'osservatore quindi non si sta verificando
l'eclissi. Se invece $d \leq r_\star + r$ il pianeta copre, almeno parzialmente,
la stella. Finché $\sqrt{r_\star^2 - r^2} \leq d \leq r_\star + r$, l'area di
sovrapposizione è la somma delle due aree $S_1$ e $S_2$ della
Figura~\ref{fig:area-coperta.a}. Per il teorema del coseno, l'angolo $\theta_1$
è dato da
\begin{equation}
  \theta_1 = 2 \arccos \frac{r_\star^2 - r^2 + d^2}{2r_\star d}
\end{equation}
e analogamente per l'angolo $\theta_2$ abbiamo
\begin{equation}
  \label{eq:theta2}
  \theta_2 = 2 \arccos \frac{r^2 - r_\star^2 + d^2}{2rd}.
\end{equation}
Le aree $S_1$ e $S_2$ valgono rispettivamente
\begin{subequations}
  \begin{align}
    S_1 &= \frac{\theta_1}{2\pi}\pi r_\star^2 - \frac{r_\star^2}{2}\sin\theta_1
    = \frac{r_\star^2}{2}(\theta_1 - \sin\theta_1), \\
    S_2 &= \frac{\theta_2}{2\pi}\pi r^2 - \frac{r^2}{2}\sin\theta_2 =
    \frac{r^2}{2}(\theta_2 - \sin\theta_2).
  \end{align}
\end{subequations}
Dunque in questo caso l'area di sovrapposizione $\delta A$ vale
\begin{equation}
  \delta A = S_1 + S_2 = \frac{r_\star^2}{2}(\theta_1 - \sin\theta_1) +
  \frac{r^2}{2}(\theta_2 - \sin\theta_2).
\end{equation}
Se $r_\star - r \leq d < \sqrt{r_\star^2 - r^2}$, l'area di sovrapposizione
$\delta A$ è sempre data da $S_1 + S_2$ (si veda la
Figura~\ref{fig:area-coperta.b}) e risulta $\theta_2 > \pi$ (si osservi
l'equazione~\eqref{eq:theta2}). In questo caso, quindi, $S_2$ è la somma del
settore circolare di angolo $\theta_2$ e del triangolo isoscele con vertice nel
centro del disco del pianeta e angolo al vertice di $2\pi - \theta_2$. Così
l'area di sovrapposizione diventa
\begin{equation}
  \begin{split}
    \delta A &= S_1 + S_2 = \frac{r_\star^2}{2}(\theta_1 - \sin\theta_1) +
    \frac{r^2}{2}(\theta_2 + \sin(2\pi -\theta_2)) \\
    &= \frac{r_\star^2}{2}(\theta_1 - \sin\theta_1) + \frac{r^2}{2}(\theta_2 -
    \sin\theta_2).
  \end{split}
\end{equation}
Infine se $d < r_\star - r$ significa che il pianeta si trova completamente
davanti alla stella rispetto all'osservatore quindi l'area di sovrapposizione è
uguale all'area del disco del pianeta, cioè
\begin{equation}
  \delta A = \pi r^2.
\end{equation}
Riepilogando
\begin{equation}
  \delta A =
  \begin{dcases}
    0 & \text{se $d > r_\star + r$}, \\
    \frac{r_\star^2}{2}(\theta_1 - \sin\theta_1) + \frac{r^2}{2}(\theta_2 -
    \sin\theta_2) & \text{se $r_\star - r \leq d \leq r_\star + r$}, \\
    \pi r^2 & \text{se $d < r_\star - r$}.
  \end{dcases}
\end{equation}

Vediamo ora in dettaglio come varia la luminosità della sistema che giunge
all'osservatore nelle varie fasi dell'eclissi. Poiché nel sistema in esame solo
la stella emette luce propria, la luminosità che prendiamo in considerazione è
esattamente quella della stella. La luminosità propria $L_\star$ di una stella,
cioè l'energia emessa sotto forma di luce nell'unità di tempo, ha le dimensioni
di una potenza, quindi nel sistema internazionale si misura in \si{\watt},
mentre nel sistema CGS si userà \si{erg\per \second}. Questa grandezza è spesso
misurata anche in unità di luminosità solari ($L_\odot$). Definiamo il
\emph{flusso superficiale} $F_\star$ come
\begin{equation}
    F_\star = \frac{L_\star}{4\pi r_\star^2},
\end{equation}
avendo assunto che il disco stellare appaia uniformemente illuminato
all'osservatore. Abbiamo anche $L_\star = 4\pi r_\star^2 F_\star$. Definiamo
inoltre il \emph{flusso a Terra} $F_{\textup{T}}$ come
\begin{equation}
  F_{\textup{T}} = \frac{L_\star}{4\pi R_0^2} = F_\star \frac{r_\star^2}{R_0^2},
\end{equation}
in cui $R_0$ è la distanza fra l'osservatore e la stella. Le definizioni fin qui
date sono valide se la stella è direttamente visibile dall'osservatore senza
alcun ostacolo. Durante l'eclissi, però, il disco della stella sarà parzialmente
coperta dal pianeta, quindi bisognerà moltiplicare le quantità sopra definite
per la frazione di disco della stella visibile all'osservatore, vale a dire
$(A - \delta A)/A$, con $A= \pi r_\star^2$. Dunque
\begin{align}
  F_\star &= \frac{L_\star}{4\pi r_\star^2} \frac{A - \delta A}{A} =
  \frac{L_\star}{4\pi r_\star^2} \left( 1 - \frac{\delta A}{A} \right), \\
  F_{\textup{T}} &= \frac{L_\star}{4\pi R_0^2} \frac{A - \delta A}{A} =
  \frac{L_\star}{4\pi R_0^2} \left( 1 - \frac{\delta A}{A} \right).
\end{align}
Al di fuori della fase di eclissi $\delta A = 0$, quindi riotteniamo le
definizioni precedenti. Più in generale possiamo definire una funzione
\emph{flusso} $F$ come
\begin{equation}
  F = \frac{f}{4} L_\star \left( 1 - \frac{\delta A}{A} \right),
\end{equation}
con $f$ fattore geometrico che cambia a seconda della quantità che si vuole
misurare. Nel caso del flusso superficiale avremo
\begin{equation}
  f \equiv f_\star = \frac{1}{\pi r_\star^2},
\end{equation}
mentre per il flusso a Terra
\begin{equation}
  f \equiv f_{\textup{T}} = \frac{1}{\pi R_0^2}.
\end{equation}

Ho scritto un programma in linguaggio C che simula un'eclissi. Il programma
genera un file con i valori, per ogni istante di tempo in cui viene svolta al
simulazione, delle coordinate della particella relativa nel sistema di
riferimento iniziale e del piano del cielo dell'osservatore, delle coordinate
della stella e del pianeta nel piano del cielo, della distanza fra i due corpi
proiettata nel piano del cielo e del flusso luminoso della stella. Il codice
sorgente del programma è riportato
nell'appendice~\ref{cha:simulazione-eclissi}. Nelle
Figure~\ref{fig:sim-ecl-piano-cielo}, \ref{fig:sim-ecl-distanza-proiettata} e
\ref{fig:sim-ecl-flusso} sono rappresentati i risultati della simulazione. Ho
inserito i seguenti valori: $r_\star = \SI{e12}{\centi\metre}$,
$r = \SI{1e11}{\centi\metre}$, $\text{massa stella} = m_1 = \SI{1}{\solarmass}$,
$\text{massa pianeta} = m_2 = \SI{0.02}{\solarmass}$,
$a = \SI{e13}{\centi\metre}$, $e = 0.8$, $\phi = \SI{30}{\degree}$,
$i = \SI{85}{\degree}$. Ho posto uguale a $1$ la luminosità intrinseca della
stella in modo che il grafico del flusso avesse come massimo proprio $1$. Il
rapporto $r_\star / r$ è stato scelto volutamente molto più vicino all'unità
rispetto agli usuali rapporti fra i raggi di stelle e pianeti in maniera da
rendere più accentuata la curva del flusso. La fase indica la quantità $(t -
t_0)/P$, con $t_0$ istante iniziale scelto per la simulazione.

\begin{figure}
  \centering
  \input{programmi/piano_cielo}
  \caption[Orbite della stella e del pianeta nel piano del cielo]{Orbite della
    stella e del pianeta nel piano del cielo ottenute dalla simulazione discussa
    in questo paragrafo}
  \label{fig:sim-ecl-piano-cielo}
\end{figure}
\begin{figure}
  \input{programmi/distanza_proiettata}
  \caption[Distanza proiettata in funzione della fase]{Andamento della distanza
    proiettata, normalizzata alla somma dei raggi della stella e del pianeta, in
    funzione della fase ottenuto dalla simulazione discussa in questo paragrafo}
  \label{fig:sim-ecl-distanza-proiettata}
\end{figure}
\begin{figure}
  \input{programmi/flusso}
  \caption[Flusso luminoso, normalizzato a $1$, di un sistema binario a
  eclissi]{Andamento del flusso luminoso, normalizzato a $1$, ottenuto dalla
    simulazione discussa in questo paragrafo}
  \label{fig:sim-ecl-flusso}
\end{figure}

\section{Possibili sviluppi futuri}
\label{sec:sviluppi-futuri}

Nel paragrafo precedente abbiamo studiato i transiti di pianeti extrasolari
davanti alle stelle compagne ma adottando alcune approssimazioni. Questo lavoro
può essere migliorato prendendo in considerazione i fatti che i due corpi non
sono perfettamente sferici ma possono essere deformati, per esempio a causa del
potenziale gravitazionale, la luminosità della stella non è uniforme su tutta la
superficie ma generalmente si ha un \emph{limb darkening}, cioè un oscuramento
al bordo, inoltre gli effetti dovuti all'atmosfera stellare non sono
trascurabili. Tutto ciò comporta che la curva di luce di un'eclissi non ha la
forma squadrata della Figura~\ref{fig:sim-ecl-flusso} ma risulta più smussata e
irregolare.

%%% Local Variables:
%%% mode: latex
%%% TeX-master: "../tesi"
%%% End:
