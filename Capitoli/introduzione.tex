\chapter{Introduzione}
\label{cha:introduzione}
\markboth{Introduzione}{Introduzione}

Sin dall'antichità l'uomo ha avuto la curiosità di osservare il moto degli astri
per riuscire a prevederne l'evoluzione. Il più semplice sistema di corpi celesti
è quello costituito da due soli astri che si attraggono per effetto della forza
di gravità. Lontano dall'essere un semplice esercizio accademico, lo studio dei
sistemi binari permette di ricavare informazioni importanti usate poi in molte
aree della ricerca astrofisica quali l'evoluzione stellare e la cosmologia.

Nel capitolo~\ref{chap:due-corpi} di questo lavoro di tesi studieremo il moto di
due corpi soggetti alla mutua interazione gravitazionale e grazie alla meccanica
newtoniana (paragrafo~\ref{sec:formalismo-newton}) e a quelle lagrangiana e
hamiltoniana (paragrafo~\ref{sec:formalismo-lagrange}) riusciremo a determinarne
le equazioni delle orbite (paragrafo~\ref{sec:soluzione-problema}). Nel
capitolo~\ref{chap:equazione-keplero}, invece, studieremo l'evoluzione temporale
delle orbite, affrontando questo problema con due differenti metodi numerici (di
Newton~-~Raphson nel paragrafo~\ref{sec:newton} e dei coefficienti di Bessel nel
paragrafo~\ref{sec:bessel}). Infine nel capitolo~\ref{chap:applicazioni}
applicheremo gli strumenti acquisiti all'astrofisica. In particolare vedremo due
modi per stimare la massa dei corpi celesti (paragrafi~\ref{sec:sgra} e
\ref{sec:funzione-massa}) e come ottenere alcune informazioni sui pianeti
extrasolari dallo studio dei transiti di questi pianeti davanti alle stelle
compagne (paragrafo~\ref{sec:extrasolari}).

%%% Local Variables:
%%% mode: latex
%%% TeX-master: "../tesi"
%%% End:
