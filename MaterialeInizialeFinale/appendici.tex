\cleardoublepage{}
\chapter{Programma per la soluzione dell'equazione di Keplero}
\label{cha:soluzione-keplero}
\markboth%
{Appendice~\ref{cha:soluzione-keplero}. Programma per la soluzione
  dell'equazione di Keplero}%
{Appendice~\ref{cha:soluzione-keplero}. Programma per la soluzione
  dell'equazione di Keplero}

Di seguito è riportato il codice sorgente del programma usato per risolvere
l'equazione di Keplero con i metodi numerici di Newton~-~Raphson e dei
coefficienti di Bessel, illustrati nel
paragrafo~\ref{sec:soluzione-keplero}. Per calcolare i coefficienti di Bessel ho
utilizzato la GNU Scientific Library (GSL).\footnote{La GSL è software libero e
  il codice sorgente può essere scaricato e consultato da Internet all'URL
  \url{http://www.gnu.org/software/gsl/}.} Anche la libreria per il C che ho
usato, Embedded GLIBC (EGLIBC),\footnote{La EGLIBC è software libero il codice
  sorgente può essere scaricato e consultato da Internet all'URL
  \url{http://www.eglibc.org/}.} definisce una funzione per calcolare i
coefficienti di Bessel, ma la GSL è più efficiente. Il programma è suddiviso in
tre file: \verb|keplero.c| che contiene il \verb|main|, l'header
\verb|libreria.h| e la libreria \verb|libreria.c|. Per compilare il programma in
ambiente GNU/Linux ho usato i seguenti comandi da terminale
\begin{verbatim}
$ gcc -c -Wall -pedantic -lm -lgsl -lgslcblas libreria.c
$ gcc -Wall -pedantic -lm -lgsl -lgslcblas -o keplero \
  keplero.c libreria.o
\end{verbatim}
L'esecuzione del programma genera due file di testo, \verb|bessel.dat| e
\verb|newton.dat|, contenenti i risultati delle simulazioni effettuate con i due
diversi metodi.
\lstinputlisting[language=C,caption={File \texttt{keplero.c}}]
{programmi/keplero.c}
\lstinputlisting[language=C,caption={File \texttt{libreria.h}},
label={lst:lib.h}]{programmi/libreria.h}
\lstinputlisting[language=C,caption={File \texttt{libreria.c}},
label={lst:lib.c}]{programmi/libreria.c}

Qui riportiamo inoltre il codice dello script \verb|gnuplot| utilizzato per
ottenere le figure del paragrafo~\ref{sec:soluzione-keplero} a partire dai file
di output generati dall'esecuzione del programma. Per generare i grafici ho
utilizzato il seguente comando da terminale
\begin{verbatim}
$ gnuplot programmi/keplero.gnuplot
\end{verbatim}
che ho eseguito nella cartella superiore rispetto a quella in cui si trova lo
script \verb|keplero.gnuplot|.
\lstinputlisting[language=gnuplot,caption={File \texttt{keplero.gnuplot}}]
{programmi/keplero.gnuplot}

\cleardoublepage{}
\chapter{Programma per la simulazione di un'eclissi}
\label{cha:simulazione-eclissi}
\markboth%
{Appendice~\ref{cha:simulazione-eclissi}. Programma per la simulazione di
  un'eclissi}%
{Appendice~\ref{cha:simulazione-eclissi}. Programma per la simulazione di
  un'eclissi}

Riporto il codice sorgente del programma che ho scritto per effettuare la
simulazione di un'eclissi di una stella dietro a un pianeta, usando le
informazioni descritte nel paragrafo~\ref{sec:extrasolari}. Per calcolare le
posizioni della particella relativa e dei due corpi ho utilizzato le funzioni
presenti nella libreria riportata nei codici~\ref{lst:lib.h} e
\ref{lst:lib.c}. Il \verb|main| del programma è contenuto nel file
\verb|eclissi.c|. Per compilare il programma in ambiente GNU/Linux ho usato i
seguenti comandi da terminale
\begin{verbatim}
$ gcc -c -Wall -pedantic -lm -lgsl -lgslcblas libreria.c
$ gcc -Wall -pedantic -lm -lgsl -lgslcblas -o eclissi \
  eclissi.c libreria.o
\end{verbatim}
Eseguendo il programma si ottiene un file di testo, \verb|eclissi.dat|, con i
risultati della simulazione.
\lstinputlisting[language=C,caption={File \texttt{eclissi.c}}]
{programmi/eclissi.c}

Di seguito è riportato lo script \verb|gnuplot| utilizzato per produrre, con il
file generato dall'esecuzione del programma, le
figure~\ref{fig:sim-ecl-piano-cielo}, \ref{fig:sim-ecl-distanza-proiettata} e
\ref{fig:sim-ecl-flusso}. Per realizzare i grafici ho utilizzato il seguente
comando da terminale
\begin{verbatim}
$ gnuplot programmi/eclissi.gnuplot
\end{verbatim}
che ho eseguito nella cartella superiore rispetto a quella in cui si trova lo
script \verb|eclissi.gnuplot|.
\lstinputlisting[language=gnuplot,caption={File \texttt{eclissi.gnuplot}}]
{programmi/eclissi.gnuplot}

%%% Local Variables:
%%% mode: latex
%%% TeX-master: "../tesi"
%%% End:
