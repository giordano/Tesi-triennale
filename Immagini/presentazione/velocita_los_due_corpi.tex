\begin{axis}[xlabel=$\theta$,
  ylabel={$v_{\textup{los}}(\theta)$ (\si{\kilo\metre\per\second})},
  legend pos=north west]
  % l'opzione `raw gnuplot' serve per permettere di scrivere uno script
  % gnuplot come argomento di `\addplot'. È una funzione molto comoda ma
  % mal documentata in pgfplots.
  \addplot [raw gnuplot,mark=none,red] gnuplot {
    r(a,e,x)=a*(1-e**2)/(1+e*cos(x));
    GM=5e11;
    v(a,e,x)=sqrt(GM*(2./r(a,e,x)-1/a));
    alpha(x,e)= e==0 ? pi/2 : ((x<=pi) ? atan(1/tan(x)+1/(e*sin(x))) :
                                         atan(1/tan(x)+1/(e*sin(x)))-pi);
    beta(x,e)= e==0 ? x+pi/2 : x+alpha(x,e);
    vlos(a,e,x)=v(a,e,x)*(cos(beta(x,e)));
    a=5e7;
    e=0.5;
    m1=5.0;
    m2=1.0;
    mu=m1*m2/(m1+m2);
    plot [0.001:2*pi] -mu/m1*vlos(a,e,x)-50
  };
  \addplot [raw gnuplot,mark=none,blue] gnuplot {
    r(a,e,x)=a*(1-e**2)/(1+e*cos(x));
    GM=5e11;
    v(a,e,x)=sqrt(GM*(2./r(a,e,x)-1/a));
    alpha(x,e)= e==0 ? pi/2 : ((x<=pi) ? atan(1/tan(x)+1/(e*sin(x))) :
                                         atan(1/tan(x)+1/(e*sin(x)))-pi);
    beta(x,e)= e==0 ? x+pi/2 : x+alpha(x,e);
    vlos(a,e,x)=v(a,e,x)*(cos(beta(x,e)));
    a=5e7;
    e=0.5;
    m1=5.0;
    m2=1.0;
    mu=m1*m2/(m1+m2);
    plot [0.001:2*pi] mu/m2*vlos(a,e,x)-50
  };
  \legend{$m_{1}$,$m_{2}$}
\end{axis}

%%% Local Variables: 
%%% mode: latex
%%% TeX-master: "../../presentazione"
%%% End: 
