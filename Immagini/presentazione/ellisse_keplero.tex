\pgfmathsetmacro{\a}{1} % a=1
\pgfmathsetmacro{\e}{0.8} % e=0.8
\pgfmathsetmacro{\b}{\a*sqrt(1-\e*\e)} % b=a*sqrt(1-e^2)
\pgfmathsetmacro{\c}{\a*\e} % c=a*e
\pgfmathsetmacro{\p}{\a*(1-\e*\e)} % p=a*(1-e^2)
\pgfmathsetmacro{\anomalia}{120} % theta=120
% r(theta)=p/(1+e*cos(theta))
\pgfmathsetmacro{\r}{\p/(1+\e*cos(\anomalia))}
\pgfmathsetmacro{\xq}{\c+\r*cos(\anomalia)} % x_Q=x_G=c+r(theta)*cos(theta)
\pgfmathsetmacro{\yq}{\r*sin(\anomalia)} % y_Q=r(theta)*sin(theta)
\pgfmathsetmacro{\yg}{\yq*\a/\b} % y_G=y_Q*a/b
% calcolo anomalia eccentrica (da relazione fra anomalia vera ed eccentrica)
\pgfmathsetmacro{\angolopsi}{2*atan(sqrt((1-\e)/(1+\e))*tan(\anomalia/2))}

\draw [->] (-1.2,0) -- (1.2,0) node[right] {$x$}; %asse x
\draw [->] (0,-1.2) -- (0,1.2) node[above] {$y$}; %asse y
\draw[thick] (0,0) circle (\a); % circonferenza ausiliaria
\draw[thick] (0,0) ellipse ({\a} and \b); %ellisse
\draw (\c,0) -- (\xq,\yq) node[above right] {$Q$}; % segmento FQ
\draw [dashed] (\xq,\yq) -- (\xq,\yg); % segmento QG
\draw (0,0) -- (\xq,\yg) node[above right] {$G$}; % segmento CG
\draw (0,0) node[below left] {$C$}; % etichetta del punto C
\draw (\c,0) node[below] {$F$}; % etichetta del punto F
\draw (\a,0) node[above right] {$B$}; % etichetta del punto B
\draw [dashdotted] (\xq,\yq) -- (\xq,0) node[below] {$R$}; % segmento RQ
% angolo theta (anomalia vera(
\draw [->] (\c+0.1,0) to [out=90,in=\anomalia-90] node[above=-2] {$\theta(t)$}
           ($(\c,0)+cos(\anomalia)*(0.1,0) + sin(\anomalia)*(0,0.1)$);
% angolo psi (anomalia eccentrica)
\draw [->] (0.1,0) to [out=90,in=\angolopsi-90] node[above right=-4] {$\psi(t)$}
           ($(0,0)+cos(\angolopsi)*(0.1,0)+sin(\angolopsi)*(0,0.1)$);

%%% Local Variables:
%%% mode: latex
%%% TeX-master: "../../tesi"
%%% End:
