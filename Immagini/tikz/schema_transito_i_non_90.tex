\pgfmathsetmacro{\rdue}{1} % raggio pianeta = 1
\pgfmathsetmacro{\runo}{4*\rdue} % raggio stella = 4
\pgfmathsetmacro{\acosi}{2.5} % a*cos(i) = 2.5
% modulo della coordinata x dell'ingresso finale (e dell'egresso finale)
\pgfmathsetmacro{\xingresso}{sqrt((\runo+\rdue)^2-\acosi^2)}
% modulo della coordinata x dell'egresso iniziale (e dell'ingrsso finale)
\pgfmathsetmacro{\xegresso}{sqrt((\runo-\rdue)^2-\acosi^2)}

\draw (\runo,0) arc (0:180:\runo); % semi-disco stella
\draw (-6,\acosi) -- (6,\acosi); % asse passante lungo equatori dei pianeti
\draw (-\runo,0) -- (\runo,0); % diametro stella
\draw[<->] (0,0) -- node[sloped,above,fill=white] {$a\cos i$} (0,\acosi);
\draw[->] (-6,-1) -- (6,-1) node[right] {$t$}; % asse del tempo
\draw (-\xingresso,\acosi) circle (\rdue) % posizione di ingresso iniziale
      (\xingresso,\acosi)  circle (\rdue) % posizione di egresso finale
      (0,0) -- node[sloped,below,fill=white] {$r_1 + r_2$} (-\xingresso,\acosi);
\draw[dashed] ($-\xingresso*(1,0) + (0,\acosi)$) -- +(0,{-\acosi-1})
              node[below] {$t_{\textup{ii}}$} ($\xingresso*(1,0) + (0,\acosi)$)
              -- +(0,{-\acosi-1}) node[below] {$t_{\textup{ef}}$};
\draw (-\xegresso,\acosi) circle (\rdue) % posizione di ingresso finale
      (\xegresso,\acosi)  circle (\rdue) % posizione di egresso iniziale
      (0,0) -- node[sloped,below,fill=white] {$r_1-r_2$} (\xegresso,\acosi);
\draw[dashed] ($-\xegresso*(1,0) + (0,\acosi)$) -- +(0,{-\acosi-1})
              node[below] {$t_{\textup{if}}$} ($\xegresso*(1,0) + (0,\acosi)$)
              -- +(0,{-\acosi-1}) node[below] {$t_{\textup{ei}}$};

%%% Local Variables:
%%% mode: latex
%%% TeX-master: "../../tesi"
%%% End:
