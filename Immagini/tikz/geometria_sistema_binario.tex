%% colori
% colore per gli assi del sistema di riferimento Ox'y'z'
\colorlet{intrinseco}{black}
% colore per gli assi del sistema di riferimento Oxyz
\colorlet{ausiliario}{red}
% colore per gli assi del sistema di riferimento Ox'y'z'
\colorlet{pdc}{blue}

\pgfmathsetmacro{\angolophi}{20} % valore dell'angolo `phi_0'
\pgfmathsetmacro{\angoloi}{40} % valore dell'angolo `i'
% origine dei sistemi di riferimento
\node [shape=coordinate](O) at (0,0,0) [label=above right:$O$] {};
% asse x'
\draw[->,intrinseco] (O) -- (1,0,0) node[anchor=north east]{$x'$};
% asse y'
\draw[->,intrinseco] (O) -- (0,1,0) node[anchor=north west]{$y'$};
% asse z' (coincidente con l'asse z)
\draw[->,intrinseco!50!ausiliario] (O) -- (0,0,1) node[anchor=south]{$z'
  \equiv z$};
\draw[thick] (-0.4,0,0) ellipse (0.5 and 0.3); % ellisse

% ruoto il sistema di riferimento principale x'y'z' intorno all'asse z'
% dell'angolo `phi_0'
\tdplotsetrotatedcoords{\angolophi}{0}{0}
% asse x
\draw[tdplot_rotated_coords,->,ausiliario] (O) -- (1,0,0)
                                           node[anchor=north]{$x$};
% asse y (coincidente con l'asse y'')
\draw[tdplot_rotated_coords,->,ausiliario!50!pdc] (O) -- (0,1,0)
                                           node[anchor=west]{$y \equiv y''$};
% disegno archi dell'angolo `phi_0'
\tdplotdrawarc[->,tdplot_rotated_coords,ausiliario]{(O)}{0.8}{-\angolophi}%
{0}{anchor=north east}{$\phi_0$}
\tdplotdrawarc[->,tdplot_rotated_coords,ausiliario]{(O)}{0.8}%
{90-\angolophi}{90}{anchor=north west}{$\phi_0$}

% ruoto il sistema di riferimento xyz (che deriva a sua volta dalla rotazione di
% x'y'z' attorno a z' dell'angolo `phi_0') intorno all'asse y dell'angolo `i'
\tdplotsetrotatedcoords{\angolophi}{-\angoloi}{0}
% asse x''
\draw[tdplot_rotated_coords,->,pdc] (O) -- (1,0,0)
                                    node[anchor=east]{$x''$};
% asse z''
\draw[tdplot_rotated_coords,->,pdc] (O) -- (0,0,1)
                                    node[anchor=south]{$z''$};
% ruoto il piano theta (di `phi_0') per disegnare l'arco di `i'
\tdplotsetthetaplanecoords{\angolophi}
% disegno archi dell'angolo `i' e `pi/2-i'
\tdplotdrawarc[->,tdplot_rotated_coords,pdc]{(O)}{0.8}{90-\angoloi}%
{0}{anchor=south east}{$i$}
\tdplotdrawarc[->,tdplot_rotated_coords,pdc]{(O)}{0.8}{90}%
{90-\angoloi}{anchor=south east}{$\pi/2-i$}
\tdplotdrawarc[->,tdplot_rotated_coords,pdc]{(O)}{0.8}{0}%
{-\angoloi}{anchor=south}{$\pi/2-i$}

%%% Local Variables:
%%% mode: latex
%%% TeX-master: "../../tesi"
%%% End:
