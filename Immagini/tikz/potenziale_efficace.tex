% coefficienti che compaiono nel potenziale efficace
\pgfmathsetmacro{\a}{25}
\pgfmathsetmacro{\b}{50}
% funzione potenziale efficace
\pgfmathdeclarefunction{ueff}{1}{\pgfmathparse{\a/#1^2-\b/#1}}
%% valori dell'energia
% energia E_0 = minimo di U_eff, il minimo si ha nel punto r = 2a/b e
% U_eff(2a/b) = -b^2/(4a)
\pgfmathsetmacro{\ezero}{-\b^2/(4*\a)}
\pgfmathsetmacro{\euno}{-20} % energia E_1
\pgfmathsetmacro{\edue}{15} % energia E_2

\draw[->] (0,-30) -- (0,30) node[above] {$U_{\textup{eff}}$}; % asse y
\draw[->] (0,0) node[left] {$O$} -- (8,0) node[right] {$r(\theta)$}; %asse x
% grafico della funzione potenziale efficace
\draw[name path=U] [domain=0.4:8,samples=100] plot (\x,{ueff(\x)});
\draw[dashed,name path={E0}] (0,\ezero) node[left] {$E_0$} -- +(8,0);
\draw[dashed,name path={E1}] (0,\euno) node[left] {$E_1$} -- +(8,0);
\draw[dashed] (0,\edue) node[left] {$E_2$} -- +(8,0);
% individuo il punto di intersezione fra il potenziale e l'energia E_0
\path[name intersections={of=U and {E0},by={A}}];
\draw[loosely dashed] (A) -- +(0,-\ezero) node[above] {$r_0$};
% individuo i punti di intersezione fra il potenziale e l'energia E_1
\path[name intersections={of=U and {E1},by={B,C}}];
\draw[loosely dashed] (B) -- +(0,-\euno) node[above] {$r_{\textup{min}}$}
                      (C) -- +(0,-\euno) node[above] {$r_{\textup{max}}$};

%%% Local Variables:
%%% mode: latex
%%% TeX-master: "../../tesi"
%%% End:
