\coordinate (O1) at (0,0); % centro stella
\coordinate (O2) at (1.5,0); % centro pianeta
\draw[name path=S,thick] (O1) circle (2); % disco stella (r=2)
\draw[name path=P,thick] (O2) circle (1); % disco pianeta (r=1)
% individuo punti di intersezione fra dischi di pianeta e stella
\path[name intersections={of=S and P,by={A,B}}];
\draw[dashed] (O1) -- (A) -- (O2) -- node[below,sloped] {$r_2$} (B) --
              node[above,sloped] {$r_1$} (O1);
\draw (A) -- (B);
\draw[->] ($(B) - (0,0.5)$) node[left,fill=white] {$S_1$} to
          [out=0,in=-60] (1.9,-0.1);
\draw[->] ($(A) + (0,0.5)$) node[right] {$S_2$} to[out=180] (1.2,0.3);
\draw let \n1 = {acos((1-4+1.5*1.5)/(2*1*1.5))} in
          ($(O2) - cos(\n1)*(0.2,0) + sin(\n1)*(0,0.2)$) arc
          (180-\n1:180+\n1:0.2);
\draw[->] ($(O2) + (1,0.6)$) node[right] {$\theta_2$} to[out=200,in=0] (1.3,0);
\draw let \n2 = {acos((4-1+1.5*1.5)/(2*2*1.5))} in
      ($(O1) + cos(\n2)*(0.3,0) - sin(\n2)*(0,0.3)$) arc (-\n2:\n2:0.3);
\draw[->] ($(O1) + (0.4,-0.4)$) node[below] {$\theta_1$} to[in=0] (0.3,0);

%%% Local Variables:
%%% mode: latex
%%% TeX-master: "../../tesi"
%%% End:
