\pgfmathsetmacro{\a}{1.0} % Semiasse delle tre ellissi: a=1
% parametri della prima ellisse
\pgfmathsetmacro{\euno}{0.2} % e1 = 0.2
\pgfmathsetmacro{\buno}{\a*sqrt(1-\euno*\euno)} % b_1=a*sqrt(1-e_1^2)
\pgfmathsetmacro{\cuno}{\a*\euno} % c_1=a*e_1
% parametri della seconda ellisse
\pgfmathsetmacro{\edue}{0.5} % e2 = 0.5
\pgfmathsetmacro{\bdue}{\a*sqrt(1-\edue*\edue)} % b_2=a*sqrt(1-e_2^2)
\pgfmathsetmacro{\cdue}{\a*\edue} % c_2=a*e_2
% parametri della terza ellisse
\pgfmathsetmacro{\etre}{0.9} % e3 = 0.9
\pgfmathsetmacro{\btre}{\a*sqrt(1-\etre*\etre)} % b_3=a*sqrt(1-e_3^2)
\pgfmathsetmacro{\ctre}{\a*\etre} % c_3=a*e_3

% Fisso il fuoco comune nell'origine, quindi i centri si trovano in (-c_i,0)
\draw [->] (-2,0) -- (1.1,0) node[right] {$x$}; %asse x
\draw [->] (0,-1.2) -- (0,1.2) node[above] {$y$}; %asse y
\draw (-\cuno,0) ellipse ({\a} and \buno); %ellisse 1
\draw (0.8,0.8) node {$e=\euno$};
\draw[dashed] (-\cdue,0) ellipse ({\a} and \bdue); %ellisse 2
\draw (-1.2,0.7) node[left] {$e=\edue$};
\draw[dashdotted] (-\ctre,0) ellipse ({\a} and \btre); %ellisse 3
\draw (-1.9,0.2) node[left] {$e=\etre$};

%%% Local Variables:
%%% mode: latex
%%% TeX-master: "../../tesi"
%%% End:
