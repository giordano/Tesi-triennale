\pgfmathsetmacro{\rdue}{1} % raggio pianeta = 1
\pgfmathsetmacro{\runo}{4*\rdue} % raggio stella = 4

\coordinate (O) at (0,0); % centro stella
\draw (O) circle (\runo); % disco stella
\draw (O) -- node[right] {$r_1$} +(0,-\runo);
\draw (-\runo-\rdue,0) -- node[above,sloped] {$r_2$}
      +($-\rdue*({cos(45)},{sin(45)})$);
\foreach \x in {-5,-3,-1,1,3,5} % varie posizioni del pianeta
  \draw (\x,0) circle (\rdue);
\draw (-8,-\runo+1) node[left] {luminosità} -- (-\runo-\rdue,-\runo+1) --
      (-\runo+\rdue,-\runo-1) -- (\runo-1,-\runo-1) --
      (\runo+\rdue,-\runo+1) -- (8,-\runo+1); % curva di luce
\draw[dashed] (-\runo-\rdue,0) -- (-\runo-\rdue,-\runo-2) node[below]
              {$t_{\textup{ii}}$};
\draw[dashed] (-\runo,-\runo) -- (-\runo,-\runo-2) node[below]
              {$t_{\textup{mi}}$};
\draw[dashed] (-\runo+\rdue,0) -- (-\runo+\rdue,-\runo-2) node[below]
              {$t_{\textup{if}}$};
\draw[dashed] (\runo-\rdue,0) -- (\runo-\rdue,-\runo-2) node[below]
              {$t_{\textup{ei}}$};
\draw[dashed] (\runo,-\runo) -- (\runo,-\runo-2) node[below]
              {$t_{\textup{mf}}$};
\draw[dashed] (\runo+\rdue,0) -- (\runo+\rdue,-\runo-2) node[below]
              {$t_{\textup{ef}}$};
\draw[->] (-8,-\runo-2) -- (8,-\runo-2) node[right] {$t$}; % asse del tempo

%%% Local Variables:
%%% mode: latex
%%% TeX-master: "../../tesi"
%%% End:
