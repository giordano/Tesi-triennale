% coordinata x del centro del pianeta (uguale al semiasse maggiore perché ho
% posto il centro della stella nell'origine)
\pgfmathsetmacro{\a}{8}
\pgfmathsetmacro{\runo}{2} % raggio stella = 2
\pgfmathsetmacro{\rdue}{1} % raggio pianeta = 1
\pgfmathsetmacro{\i}{acos(sqrt(1-\rdue*\rdue/(\a*\a)))} % angolo i

\coordinate (O1) at (0,0); % centro stella
\coordinate (O2) at (\a,0); % centro pianeta
\draw[thick] (O1) circle (\runo); % disco stella
\node at ($(O1) + (2.3,-1.2)$) {$m_1$};
\draw[thick] (O2) circle (\rdue); % disco pianeta
\node at ($(O2) + (-1.1,-1)$) {$m_2$};
\draw[->] (O1) -- +(0,3) node[above] {$z$}; % asse z
\draw[->] (O1) -- +(10,0) node[right] {$x$}; % asse x
\draw[->] (O1) -- ($10*({cos(\i)},{sin(\i)})$) node[right] {$x''$}; % asse x''
\draw[dashed] ($(O1) + (0,2.5)$) to[out=0,in=90+\i] % angolo di inclinazione
              node[right=5] {$i_{\textup{min}}$} ($2.5*({cos(\i)},{sin(\i)})$);
% distanza fra i centri di pianeta e stella
\draw[<->] ($(O1) + (0,-0.2)$) -- node[fill=white] {$a$} ($(O2)+(0,-0.2)$);
\draw (O1) -- node[left] {$r_1$} +($-\runo*({cos(45)},{sin(45)})$);
\draw (O2) -- node[right] {$r_2$} +($\rdue*({cos(90+\i)},{sin(90+\i)})$);

%%% Local Variables:
%%% mode: latex
%%% TeX-master: "../../tesi"
%%% End:
