\pgfmathsetmacro{\a}{1} % a=1
\pgfmathsetmacro{\e}{0.8} % e=0.8
\pgfmathsetmacro{\b}{\a*sqrt(1-\e*\e)} % b=a*sqrt(1-e^2)
\pgfmathsetmacro{\c}{\a*\e} % c=a*e
\pgfmathsetmacro{\p}{\a*(1-\e*\e)} % p=a*(1-e^2)
\pgfmathsetmacro{\m}{0.3} % massa
\pgfmathsetmacro{\scala}{4} % fattore di scala per il momento angolare

% funzione che restituisce la quantità di moto nel punto di anomalia vera #1
% (valgono sempre le stesse considerazioni: la funzione restituisce valori
% corretti se #1 != 0, 180, 360 e numeri a loro congrui in modulo 360)
\pgfmathdeclarefunction{mv}{1}{\pgfmathparse{\m*sqrt(2/dist(#1)-1/\a)}}
% funzione che restituisce il modulo del momento angolare nel punto di
% anomalia vera #1. (valgono sempre le stesse considerazioni: la funzione
% restituisce valori corretti se #1 != 0, 180, 360 e numeri a loro congrui
% in modulo 360)
\pgfmathdeclarefunction{momang}{1}%
                      {\pgfmathparse{dist(#1)*mv(#1)*sin(alpha(#1))*\scala}}

%% parametri del primo punto
% anomalia vera del primo punto
\pgfmathsetmacro{\anomaliauno}{0}
% coordinata x del primo punto in cui disegno il vettore posizione
\pgfmathsetmacro{\xuno}{dist(\anomaliauno)*cos(\anomaliauno)}
% coordinata y del primo punto in cui disegno il vettore posizione
\pgfmathsetmacro{\yuno}{dist(\anomaliauno)*sin(\anomaliauno)}
% componente x del vettore quantità di moto. poiché anomalia = 0 pongo
% manualmente p_x = 0, p_y = |p| (negli altri casi posso usare le funzioni
% sopra definite)
\pgfmathsetmacro{\pxuno}{0}
% componente y del vettore quantità di moto
\pgfmathsetmacro{\pyuno}{mv(\anomaliauno)}
%% secondo punto
% anomalia vera del secondo punto
\pgfmathsetmacro{\anomaliadue}{130}
% coordinata x del secondo punto in cui disegno il vettore posizione
\pgfmathsetmacro{\xdue}{dist(\anomaliadue)*cos(\anomaliadue)}
% coordinata y del secondo punto in cui disegno il vettore posizione
\pgfmathsetmacro{\ydue}{dist(\anomaliadue)*sin(\anomaliadue)}
% componente x del vettore quantità di moto
\pgfmathsetmacro{\pxdue}{mv(\anomaliadue)*cos(beta(\anomaliadue))}
% componente y del vettore quantità di moto
\pgfmathsetmacro{\pydue}{mv(\anomaliadue)*sin(beta(\anomaliadue))}

% origine dei sistemi di riferimento
\node [shape=coordinate](O) at (0,0,0) [label=above right:$O$] {};
\draw[->] (O) -- (1,0,0) node[anchor=north east]{$x$}; % asse x
\draw[->] (O) -- (0,1,0) node[anchor=north west]{$y$}; % asse y
\draw[->] (O) -- (0,0,1) node[anchor=south]{$z$}; % asse z
\draw (-\c,0,0) ellipse ({\a} and \b); % ellisse

%% primo punto
\draw[->,thick] (O) -- node[above] {$\bm{r}$} (\xuno,\yuno); % posizione
% vettore quantità di moto
\draw[->,thick] (\xuno,\yuno) -- node[below] {$\bm{p}$} +(\pxuno,\pyuno);
% vettore momento angolare
\draw[->,thick] (\xuno,\yuno,0) -- node[left] {$\bm{l}_0$}
                +(0,0,{dist(\anomaliauno)*mv(\anomaliauno)*\scala});

% %% secondo punto
\draw[->,thick] (O) -- node[above] {$\bm{r}$} (\xdue,\ydue); % posizione
% vettore quantità di moto
\draw[->,thick] (\xdue,\ydue) -- node[below] {$\bm{p}$} +(\pxdue,\pydue);
% vettore momento angolare
\draw[->,thick] (\xdue,\ydue,0) -- node[right] {$\bm{l}_0$}
                +(0,0,{momang(\anomaliadue)});

%%% Local Variables:
%%% mode: latex
%%% TeX-master: "../../tesi"
%%% End:
