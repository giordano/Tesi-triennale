\documentclass[a4paper,oneside,fleqn]{book}
\usepackage[T1]{fontenc}
\usepackage[utf8]{inputenc}
\usepackage[italian]{babel}

% TODO: scegliere il titolo definitivo e creare il frontespizio con il pacchetto
% `frontespizio'
\title{Equazione di Keplero}

\begin{document}

\maketitle{} % TODO: il frontespizio andrà fatto con `frontespizio'

\chapter{Il problema dei due corpi}
\label{chap:due-corpi}

\section{Formalismo newtoniano}
\label{sec:formalismo-newton}

\section{Formalismo lagrangiano}
\label{sec:formalismo-lagrange}

\chapter{Elementi orbitali}
\label{chap:elementi-orbitali}

\section{Soluzioni dell'equazione di Keplero}
\label{sec:soluzioni}

\subsection{Metodo di Newton~-~Raphson}
\label{sec:newton}

\subsection{Integrali ellittici}
\label{sec:integrali-ellittici}

\subsection{Funzioni di Bessel}
\label{sec:bessel}

\chapter{Applicazioni astrofisiche}
\label{chap:applicazioni}

\section{La funzione di massa. Sistemi binari X}
\label{sec:funzione-massa}

\section{Il sistema binario Sgr A*~-~S2}
\label{sec:sgra}

\section{I pianeti extrasolari}
\label{sec:extrasolari}



\end{document}
