\documentclass[a4paper,oneside,fleqn]{book}
\usepackage[T1]{fontenc}
\usepackage[utf8]{inputenc}
\usepackage[italian]{babel}

\usepackage{siunitx}
\usepackage[autostyle=true]{csquotes}
\usepackage[style=numeric,backref,hyperref,abbreviate=false,backend=biber]{biblatex}
\usepackage[colorlinks=true,urlcolor=black,linkcolor=black,citecolor=black]{hyperref}

\sisetup{per-mode=symbol,
  inter-unit-separator={}\cdot{},
  exponent-product=\cdot,
  output-product=\cdot
}

\bibliography{bibliografia} % nome del file contenente la bibliografia
\defbibheading{subbibliography}{\section*{#1}\markboth{#1}{#1}}

% TODO: scegliere il titolo definitivo e creare il frontespizio con il pacchetto
% `frontespizio'
\title{Equazione di Keplero}

\begin{document}

\maketitle{} % TODO: il frontespizio andrà fatto con `frontespizio'

\chapter{Il problema dei due corpi}
\label{chap:due-corpi}

\section{Formalismo newtoniano}
\label{sec:formalismo-newton}

\section{Formalismo lagrangiano}
\label{sec:formalismo-lagrange}

\chapter{Elementi orbitali}
\label{chap:elementi-orbitali}

\section{Soluzioni dell'equazione di Keplero}
\label{sec:soluzioni}

\subsection{Metodo di Newton~-~Raphson}
\label{sec:newton}

\subsection{Integrali ellittici}
\label{sec:integrali-ellittici}

\subsection{Funzioni di Bessel}
\label{sec:bessel}

\chapter{Applicazioni astrofisiche}
\label{chap:applicazioni}

\section{La funzione di massa. Sistemi binari X}
\label{sec:funzione-massa}

\section{Il sistema binario Sgr A*~-~S2}
\label{sec:sgra}

\section{I pianeti extrasolari}
\label{sec:extrasolari}

\clearpage{}
\phantomsection{}
\nocite{*} % TODO: citare i libri
\addcontentsline{toc}{chapter}{Riferimenti bibliografici}
\printbibliography[title=Riferimenti bibliografici]

\end{document}

%%% Local Variables:
%%% mode: latex
%%% TeX-master:
%%% fill-column: 80
%%% End:
