\documentclass[a4paper,oneside,fleqn]{book}
\usepackage[T1]{fontenc}
\usepackage[utf8]{inputenc}
\usepackage[italian]{babel}

%%%%% Pacchetti caricati
% `mathtools' serve per definire la norma e il valore assoluto. Caricarlo prima
% di `amsthm' altrimenti i simboli QED potrebbero essere allineati a sinistra
% usando l'opzione di classe `fleqn'.
\usepackage{mathtools}
\usepackage{amsmath,amsfonts,amssymb,amsthm}
\usepackage{siunitx}
\usepackage[autostyle=true]{csquotes}
\usepackage[style=numeric,backref,hyperref,abbreviate=false,backend=biber]{biblatex}
\usepackage{hyperref}
% `bm' serve per scrivere i vettori in corsivo con il comando
% `\bm{vettore}'. Deve sostituire il comando `\mathbf{vettore}' perché questo
% restituisce erroneamente lettere in tondo, non in corsivo e non funziona con
% le lettere greche.
\usepackage{bm}
%%%%% Fine pacchetti

%%%%% Impostazioni
% impostazioni per il pacchetto `hyperref'. Per l'elenco di tutte le opzioni del
% documento consulta il manuale: `texdoc hyperref'
\hypersetup{colorlinks,% link colorati, non riquadrati
  breaklinks=true,% permette di spezzare i link su più righi
  bookmarksnumbered,% inserisce i numeri delle sezioni nei segnalibri
  urlcolor=black,linkcolor=black,citecolor=black
}

\sisetup{per-mode=symbol,
  inter-unit-separator={}\cdot{},
  exponent-product=\cdot,
  output-product=\cdot
}

\bibliography{bibliografia} % nome del file contenente la bibliografia
\defbibheading{subbibliography}{\section*{#1}\markboth{#1}{#1}}

% TODO: scegliere il titolo definitivo e creare il frontespizio con il pacchetto
% `frontespizio'
\title{Equazione di Keplero}
%%%%% Fine impostazioni

%%%%% Comandi personalizzati
% ridefinisco i comandi per alcune lettere greche in modo che si usino le
% varianti
\renewcommand{\phi}{\varphi}
\renewcommand{\epsilon}{\varepsilon}

% Operatori
\newcommand*{\dd}{\mathop{}\!\mathrm{d}} % Operatore differenziale \dd
\DeclareMathOperator{\uimm}{\mathrm{i}} % unità immaginaria
% Operatore valore assoluto \abs{x}. Usa \abs*{} per le frazioni
\DeclarePairedDelimiter{\abs}{\lvert}{\rvert}
% Operatore norma \norm{x}. Usa \norm*{} per le frazioni
\DeclarePairedDelimiter{\norm}{\lVert}{\rVert}

%% Derivate
% Derivata totale: \toder[ordine]{funzione}{variabile}
\newcommand*{\toder}[3][]{\frac{{\dd^{#1}}#2}{\dd {#3}^{#1}}}
% Derivata parziale \parder[ordine]{funzione}{variabile}
% Per la definizione del comando `parder' (per inserire le derivate parziali)
% vedi http://www.guit.sssup.it/phpbb/viewtopic.php?p=42188#42188
\makeatletter
\newcommand{\parder}[2]{\begingroup
  \@tempswafalse\toks@={}\count@=\z@
  \@for\next:=#2\do
    {\expandafter\check@var\next\@nil
     \advance\count@\parder@exp
     \if@tempswa
       \toks@=\expandafter{\the\toks@\,}%
     \else
       \@tempswatrue
     \fi
     \toks@=\expandafter{\the\expandafter\toks@\expandafter\partial\parder@var}}%
  \frac{\partial\ifnum\count@=\@ne\else^{\number\count@}\fi#1}{\the\toks@}%
  \endgroup}
\def\check@var{\@ifstar{\mult@var}{\one@var}}
\def\mult@var#1#2\@nil{\def\parder@var{#2^{#1}}\def\parder@exp{#1}}
\def\one@var#1\@nil{\def\parder@var{#1}\chardef\parder@exp\@ne}
\makeatother

% Derivate per le formule in linea (usare \frac in linea è eccessivo). La `l'
% iniziale nel nome distingue questi comandi da quelli per le formule fuori
% corpo. Non uso `\dd' ma `\mathrm{d}' perché nelle formule in linea `\dd'
% aggiunge una spaziatura non adatta. Non sono dei comandi bellissimi, ma
% permettono di passare facilmente da formula in linea a fuori corpo e viceversa
% cambiando una lettera.
% Derivata totale: \ltoder[ordine]{funzione}{variabile}
\newcommand*{\ltoder}[3][]{\mathrm{d}^{#1}#2 / \mathrm{d} {#3}^{#1}}
% Derivata parziale: \lparder[ordine]{funzione}{variabile}
\newcommand*{\lparder}[3][]{\partial^{#1} #2 / \partial {#3}^{#1}} 
% NOTA: `\parder' e `\lparder' non sono completamente interscambiabili, il primo
% comando è molto più complesso e permette di inserire le derivate miste, a
% differenza del secondo.

% Costante di Eulero
\DeclareMathOperator{\e}{\mathrm{e}}
% Versore. Esempi: versore x: `\versore{x}', versore i: \versor{\imath}, versore
% j: \versor{\jmath} (solo `i' e `j' richiedono `\imath' e `\jmath', altrimenti
% il puntino litiga con `\hat')
\newcommand*{\versor}[1]{\hat{\bm{#1}}}

% Ambiente per scrivere sistemi di equazioni.
% Vedi `L'arte di scrivere con LaTeX' di Pantieri.
% Esempio d'uso (in ambiente matematico):
%	\begin{sistema}
%         x+y+z=0 \\
%         2x-y=1 \\
%         y-4z=-3
%       \end{sistema}
\newenvironment{sistema}%
{\left\lbrace\begin{array}{@{}l@{}}}%
    {\end{array}\right.}
%%%%% Fine comandi personalizzati

\begin{document}
\frontmatter{}
\maketitle{} % TODO: il frontespizio andrà fatto con `frontespizio'

\clearpage{}
\tableofcontents{} % TODO: capire perché questa pagina è numerata `1'

\mainmatter{}
\chapter{Il problema dei due corpi}
\label{chap:due-corpi}

In questo capitolo affronteremo il \emph{problema dei due corpi}, cioè lo studio
del moto di due corpi, supposti puntiformi, soggetti solamente alla mutua
interazione dovuta a forze che soddisfano il principio di azione e reazione. In
particolare ci occuperemo del \emph{problema di Keplero}, ovvero del caso in cui
l'interazione fra i due corpi è quella gravitazionale che è una forza di tipo
centrale. Il problema sarà risolto usando il formalismo newtoniano e quello
lagrangiano, di quest'ultimo presenteremo una breve introduzione. Riusciremo a
determinare analiticamente le equazioni delle orbite dei due corpi e a dedurre
alcune proprietà del sistema. La risoluzione del problema di Keplero permette di
descrivere, fra le altre cose, il moto di due corpi celesti, per esempio un
pianeta e la sua stella, che sono soggetti solo alla mutua interazione
gravitazionale.

\section{Formalismo newtoniano}
\label{sec:formalismo-newton}

Consideriamo un sistema costituito da due corpi, puntiformi, di masse $m_1$ e
$m_2$ e in un fissato sistema di riferimento inerziale indichiamo con $\bm{r}_1$
e $\bm{r}_2$ i rispettivi vettori posizione. %TODO: inserire figura, magari tikz
Per la seconda legge di Newton la forza $\bm{F}_{21}$ che il corpo di massa
$m_1$ esercita sul corpo di massa $m_2$ è data da
\begin{equation}
  \label{eq:f21}
  \bm{F}_{21} = m_2\toder[2]{\bm{r}_2}{t}
\end{equation}
e la forza $\bm{F}_{12}$ che il corpo di massa $m_2$ esercita sull'altro è
\begin{equation}
  \label{eq:f12}
  \bm{F}_{12} = m_1\toder[2]{\bm{r}_1}{t}.
\end{equation}
Assumiamo per ipotesi che queste due forze soddisfino il principio di azione e
reazione nella forma forte:
\begin{equation}
  \label{eq:az-reaz}
  \bm{F}_{12} = -\bm{F}_{21}.
\end{equation}
Definiamo il vettore \emph{posizione relativa}
\begin{equation}
  \label{eq:pos-rel}
  \bm{r}=\bm{r}_2-\bm{r}_1
\end{equation}
e il \emph{centro di massa}
\begin{equation}
  \label{eq:cdm}
  \bm{r}_\textup{g} = \frac{m_1\bm{r}_1+m_2\bm{r}_2}{m_1+m_2}.
\end{equation}
Dalle equazioni~\eqref{eq:pos-rel} e \eqref{eq:cdm} possiamo ricavare $\bm{r}_1$
e $\bm{r}_2$ in funzione di $\bm{r}$ e $\bm{r}_\textup{g}$:
\begin{subequations}
  \begin{align}
    \bm{r}_1 &= \bm{r}_\textup{g} - \frac{m_2}{m_1+m_2}\bm{r} =
    \bm{r}_\textup{g} - \frac{\mu}{m_1}\bm{r},\\
    \bm{r}_2 &= \bm{r}_\textup{g} + \frac{m_1}{m_1+m_2}\bm{r} =
    \bm{r}_\textup{g} + \frac{\mu}{m_2}\bm{r},
  \end{align}
\end{subequations}
in cui abbiamo introdotto la \emph{massa ridotta} $\mu$ definita dalla metà
della media armonica fra $m_1$ e $m_2$
\begin{equation}
  \frac{1}{\mu} = \frac{1}{m_1} + \frac{1}{m_2} \iff \mu=\frac{m_1m_2}{m_1+m_2}.
\end{equation}
Osserviamo che la massa ridotta è sempre minore delle due masse, infatti
\begin{equation}
  \mu =\frac{m_1}{m_1/m_2+1} = \frac{m_2}{1+m_2/m_1}.
\end{equation}
Definiamo inoltre la \emph{massa totale} $M_\textup{T}$ del sistema
\begin{equation}
  M_\textup{T}=m_1+m_2.
\end{equation}
Dalle definizioni di massa ridotta e totale deriva che
$M_\textup{T}\mu=m_1m_2$. Se, per esempio, $m_2\gg m_1$ allora $\mu\simeq m_1$ e
$M_\textup{T}\simeq m_2$. L'equazione del moto del centro di massa può essere
trovata semplicemente calcolando l'accelerazione di $\bm{r}_\textup{g}$:
\begin{equation}
  \toder[2]{\bm{r}_\textup{g}}{t} = \frac{1}{M_\textup{T}}
  \left(
    m_1\toder[2]{\bm{r}_1}{t} + m_2\toder[2]{\bm{r}_2}{t}
  \right) = \frac{\bm{F}_{12}+\bm{F}_{21}}{M_\textup{T}} = \bm{0},
\end{equation}
quindi il centro di massa è in quiete o si muove di moto rettilineo uniforme. A
ogni modo il sistema in cui il centro di massa è a riposo è un sistema inerziale
allora lo scegliamo come nostro nuovo sistema di riferimento e in particolare
fissiamo come origine del sistema la posizione del centro di massa. In questo
nuovo sistema le posizioni delle due masse saranno dunque descritte dai vettori
\begin{subequations} %TODO: alla fine controllare aspetto di questa equazione
  \begin{align}
    \bm{r}_1 &= -\frac{\mu}{m_1}\bm{r},\\
    \bm{r}_2 &= \frac{\mu}{m_2}\bm{r}.
  \end{align}
\end{subequations}
Osserviamo che se, per esempio, $m_2\gg m_1$ allora $\bm{r}_2\simeq\bm{0}$, cioè
il corpo più massivo si trova praticamente nel centro di massa, come è logico
aspettarsi dalla definizione di centro di massa. Dividendo ambo i membri
della~\eqref{eq:f21} per $m_2$ e ambo i membri della~\eqref{eq:f12} per $m_1$ e
sottraendo membro a membro le equazioni così ottenuti ricaviamo
\begin{equation}
  \frac{\bm{F}_{21}}{m_2}-\frac{\bm{F}_{12}}{m_1} = \bm{F}_{21}
  \left(
    \frac{1}{m_2}+\frac{1}{m_1}
  \right) = \frac{\bm{F}_{21}}{\mu} = \toder[2]{(\bm{r}_2-\bm{r}_1)}{t} =
  \toder[2]{\bm{r}}{t}
\end{equation}
in cui abbiamo ricordato che vale il principio di azione e reazione espresso
dalla~\eqref{eq:az-reaz}. L'equazione precedente può essere riscritta nella forma
\begin{equation}
  \label{eq:f21-mu}
  \bm{F}_{21}=\mu\toder[2]{\bm{r}}{t}
\end{equation}
da cui possiamo dedurre che il problema dei due corpi interagenti si è ridotto
al problema del moto di un solo corpo puntiforme fittizio, di massa $\mu$ e
posizione $\bm{r}$ nel sistema del centro di massa, all'interno campo generato
da un altro corpo puntiforme fittizio di massa $M_\textup{T}$ fisso
nell'origine.

\subsection{Problema di Keplero}

Consideriamo il caso dell'interazione gravitazionale. Se indichiamo con $M$ la
massa del corpo che genera il campo a cui è soggetto il corpo di massa $m$ e con
$\bm{r}$ il vettore diretto dal corpo di massa $M$ verso il corpo di massa $m$
allora la forza $\bm{F}_\textup{G}$ è data da
\begin{equation}
  \label{eq:legge-attrazione}
  \bm{F}_\textup{G} = -\frac{GMm}{r^2}\versor{r},
\end{equation}
in cui $G=\SI{6.674
  28(67)e-11}{\metre\tothe{3}\per\kilogram\per\second\squared}$ è la costante di
gravitazione universale~\cite{codata-costanti}. Si tratta di una forza centrale,
cioè puramente radiale, e il segno meno indica che è sempre attrattiva verso il
centro del campo. Sostituendo la~\eqref{eq:legge-attrazione}
nella~\eqref{eq:f21-mu} abbiamo
\begin{equation}
  \label{eq:forza-mu}
  \bm{F}_{21} = -\frac{GM_\textup{T}\mu}{r^2}\versor{r} =
  -\frac{Gm_1m_2}{r^2}\versor{r} = \mu\toder[2]{\bm{r}}{t}.
\end{equation}

Per una qualsiasi forza di tipo centrale il momento angolare
$\bm{L}=\bm{r}\times\bm{p}$ rispetto al centro della forza è una costante del
moto. Infatti il momento torcente $\bm{N}$ di una forza $\bm{F}$ rispetto a un
polo $O$ è definito da
\begin{equation}
  \bm{N}= \toder{\bm{L}}{t} = \bm{r}\times\bm{F}
\end{equation}
dove qui con $\bm{r}$ indichiamo il vettore congiungente il polo con il punto di
applicazione della forza. Dunque il momento della forza rispetto alla sorgente
di un campo centrale è nullo essendo in questo caso $\bm{F}$ diretto lungo
$\bm{r}$ e il momento angolare calcolato rispetto allo stesso polo è
costante.

Ritorniamo al problema di Keplero. Abbiamo appena visto che il momento angolare
della particella di massa ridotta rispetto al centro di massa del sistema è un
integrale primo del moto. Notiamo ora che, per definizione di momento angolare,
i vettori $\bm{r}$ e $\bm{L}\equiv\bm{l}_0$ sono fra loro perpendicolari. La
costanza del vettore $\bm{l}_0$, e in particolare della sua direzione, allora
implica che, nel caso $\bm{l}_0\neq\bm{0}$, il vettore $\bm{r}$, che istante per
istante indica la posizione rispetto al centro di massa della particella
fittizia di massa $\mu$, giace sempre sullo stesso piano perpendicolare a
$\bm{l}_0$. Se invece risulta $\bm{l}_0=\bm{0}$, $\bm{r}$ è parallelo alla
quantità di moto $\bm{p}$ e il moto è unidimensionale. Da qui in seguito
supporremo che il momento angolare $\bm{l}_0$ della particella di massa ridotta
sia non nullo.

Dal momento che il moto della particella di massa ridotta si svolge in un piano
possiamo utilizzare le coordinate polari $r,\theta$ per rappresentare la
posizione del corpo nel sistema di riferimento del centro di massa. Dalla
meccanica si ricava che per un moto piano la velocità $\dot{\bm{r}}$ e
l'accelerazione $\ddot{\bm{r}}$ del vettore posizione $\bm{r}=r\versor{r}$ sono
date da
\begin{subequations}
  \begin{align} %TODO: alla fine controllare aspetto di questa equazione
    \dot{\bm{r}}  &= \dot{r}\versor{r} +
    r\dot{\theta}\versor{\theta}, \label{eq:velocita-polare}\\
    \ddot{\bm{r}} &= (\ddot{r}-r\dot{\theta}^2)\versor{r} + (r\ddot{\theta} +
    2\dot{r}\dot{\theta}) \versor{\theta}.
  \end{align}
\end{subequations}
Usando le coordinate polari l'equazione vettoriale~\eqref{eq:forza-mu} può
allora essere riscritta come due equazioni scalari:
\begin{subequations}
  \begin{gather}
    -\frac{GM_\textup{T}\mu}{r^2} = \mu(\ddot{r} - r\dot{\theta}^2) \iff
    -\frac{GM_\textup{T}}{r^2}=\ddot{r}-r\dot{\theta}^2, \label{eq:forza-mu-r}\\
    0 = \mu(r\ddot{\theta} + 2\dot{r}\dot{\theta}). \label{eq:forza-mu-az}
  \end{gather}
\end{subequations}
Dalla~\eqref{eq:forza-mu-az} si trova nuovamente la costanza del modulo del
momento angolare $l_0 = \mu r^2\dot{\theta}$, infatti
\begin{equation}
  0 = r\mu(r\ddot{\theta} + 2\dot{r}\dot{\theta}) = \toder{\mu
    r^2\dot{\theta}}{t} = \toder{l_0}{t}.
\end{equation}
L'area differenziale $\dd A$ spazzata dalla particella di massa ridotta che si
sposta di un angolo infinitesimo $\dd\theta$ è espressa da
\begin{equation}
  \dd A = \frac{r\cdot r\dd\theta}{2} = \frac{r^2\dd\theta}{2},
\end{equation}
quindi la velocità areolare $\ltoder{A}{t}$ è
\begin{equation}
  \toder{A}{t} = \frac{1}{2}r^2\dot{\theta} = \frac{1}{2}\frac{l_0}{\mu} =
  \text{costante}.
\end{equation}
Abbiamo così ottenuto la seconda legge di Keplero: \emph{il vettore posizione
  della particella rispetto al centro di massa spazza aree uguali in intervalli
  di tempo uguali}. Osserviamo che per giungere a questo risultato è stato
sufficiente utilizzare la costanza del momento angolare che deriva a sua volta
dal carattere centrale della forza, pertanto questo risultato è valido per tutte
le forze di questo tipo.

\subsection{Equazione dell'orbita}
\label{sec:equazione-dellorbita}

Nel problema di Keplero, vogliamo ora cercare un'espressione esplicita della
coordinata polare $r$ in funzione dell'altra coordinata $\theta$, che a sua
volta dipenderà dal tempo. Per fare ciò è conveniente effettuare la sostituzione
$u=1/r$. Espandiamo le derivate temporali di $r$:
\begin{subequations}
  \begin{align}
    \dot{r} &= \toder{r}{t} = \toder{r}{\theta}\toder{\theta}{t} =
    \toder{(1/u)}{\theta}\dot{\theta} =
    -\frac{\dot{\theta}}{u^2}\toder{u}{\theta}
    = -\frac{l_0}{\mu}\toder{u}{\theta}, \label{eq:derivata1-r}\\
    \begin{split}
      \ddot{r} &= \toder[2]{r}{t} = \toder{}{\theta} \left( \toder{r}{t}
      \right)\toder{\theta}{t} = \toder{}{\theta} \left(
        -\frac{l_0}{\mu}\toder{u}{\theta} \right)\dot{\theta} =
      -\frac{l_0\dot{\theta}}{\mu}\toder[2]{u}{\theta} \\
      &= - \left(
        \frac{l_0}{\mu} \right)^2u^2\toder[2]{u}{\theta}. \label{eq:derivata2-r}
    \end{split}
  \end{align}
\end{subequations}
Sostituendo la~\eqref{eq:derivata2-r} nella~\eqref{eq:forza-mu-r} abbiamo
\begin{equation}
  -GM_\textup{T}u^2 = -\left(
        \frac{l_0}{\mu} \right)^2u^2\toder[2]{u}{\theta} - \frac{l_0^2u^3}{\mu^2}.
\end{equation}
$r$ è la distanza della particella di massa ridotta dal centro di massa, quindi
assume valori finiti e il suo reciproco $u$ non è mai nullo, allora
nell'equazione precedente possiamo dividere ambo i membri per $u^2$ e otteniamo
\begin{equation}
  \toder[2]{u}{\theta} + u = \frac{GM_\textup{T}\mu^2}{l_0^2}.
\end{equation}
Questa è una equazione differenziale ordinaria lineare a coefficienti costanti
del secondo ordine, chiamata \emph{equazione di Binet}. Per semplificare i
calcoli poniamo
\begin{equation}
  \label{eq:rec-semilato}
  \frac{GM_\textup{T}\mu^2}{l_0^2} = \frac{1}{p}
\end{equation}
e così l'integrale generale dell'equazione di Binet è
\begin{equation}
  \label{eq:sol-binet}
  u(\theta) = c_0\cos\theta + c_1\sin\theta + \frac{1}{p}
\end{equation}
con $c_0$ e $c_1$ costanti determinate dalle condizioni iniziali. La soluzione
può essere riscritta in maniera diversa se imponiamo
\begin{subequations}
  \label{eq:cost-binet}
  \begin{align}
    c_0 &= \frac{e}{p}\cos\tilde\omega, \\
    c_1 &= \frac{e}{p}\sin\tilde\omega,
  \end{align}
\end{subequations}
dove $e$ (un'ampiezza) e $\tilde\omega$ (una fase) svolgono il ruolo di nuove
costanti di integrazione. Le~\eqref{eq:cost-binet} possono essere invertite:
\begin{subequations}
  \begin{align}
    \tilde\omega &= \arctan \frac{c_1}{c_0}, \\
    e &= \frac{c_0p}{\cos\omega}.
  \end{align}
\end{subequations}
Dunque fissate $c_0$ e $c_1$, $\tilde\omega$ e $e$ sono univocamente
determinate. In questo modo la~\eqref{eq:sol-binet} diventa
\begin{equation}
  u(\theta) = \frac{1}{p}(1+e\cos(\theta-\tilde\omega))
\end{equation}
e ricordando che $u=1/r$ giungiamo all'equazione polare dell'orbita della
particella di massa ridotta nel sistema di riferimento del centro di massa
\begin{equation}
  \label{eq:orbita}
  r(\theta) = \frac{p}{1+e\cos(\theta-\tilde\omega)}.
\end{equation}

\subsection{Classificazione geometrica delle orbite}
\label{sec:class-geom-orbite}

La~\eqref{eq:orbita} è l'equazione polare di una conica con centro nel fuoco o
uno dei fuochi nel caso dell'ellisse. La costante $e$ è chiamata
\emph{eccentricità} della conica, la costante $p$ invece è detta \emph{semilato
  retto}. Entrambe possono assumere valori non negativi. A seconda dei diversi
valori dell'eccentricità le orbite possono essere classificate nel seguente
modo:
\begin{itemize}
\item $0\leq e<1$: \emph{ellisse}. Nel caso particolare $e=0$ si ha una
  \emph{circonferenza};
\item $e=1$: \emph{parabola};
\item $e>1$: \emph{iperbole}.
\end{itemize}
\begin{figure}
  \centering
  \input{Immagini/gnuplot/ellisse}
  \caption{Ellisse in coordinate polari. L'origine del sistema di riferimento è
    il centro di massa (CDM) che coincide con uno dei fuochi dell'ellisse, $a$ è
    il semiasse maggiore, $b$ è il semiasse minore, $p$ è il semilato retto, $c$
    è l'eccentricità lineare.}
  \label{fig:ellisse}
\end{figure}

Ricordiamo alcune proprietà geometriche delle coniche. Per l'ellisse (vedi la
figura~\ref{fig:ellisse}) il semilato retto $p$ è dato da
\begin{equation}
  \label{eq:semilato-ellisse}
  p = a(1-e^2) = \frac{b^2}{a},
\end{equation}
dove $a$ è il \emph{semiasse maggiore} dell'ellisse e $b$ è il \emph{semiasse
  minore}. Dalla~\eqref{eq:semilato-ellisse} ricaviamo inoltre che
\begin{align}
  b &= a\sqrt{1-e^2},\\
  e &= \sqrt{1-\frac{b^2}{a^2}} =
  \sqrt{1-\frac{p}{a}}. \label{eq:eccentricita-ellisse}
\end{align}
L'\emph{eccentricità lineare} $c$ è la distanza fra il centro dell'ellisse e i
fuochi e risulta
\begin{equation}
  c=\sqrt{a^2-b^2} = ae.
\end{equation}
Il \emph{periapside} è il punto di un ellisse più vicino a uno dei due fuochi,
l'\emph{apoapside} è il punto dell'ellisse più lontano dallo stesso
fuoco. Osservando la figura~\ref{fig:ellisse} si riconosce che il periapside è
raggiunto in corrispondenza di $\theta=\tilde\omega$ e la distanza del
periapside dal fuoco è $r_\textup{min} = r(\tilde\omega) = a(1-e)$; l'apoapside
è raggiunto per $\theta=\tilde\omega+\pi$ e la distanza dell'apoapside dallo
stesso fuoco è $r_\textup{max} = r(\tilde\omega+\pi) = a(1+e)$. Si ottiene il
caso particolare della circonferenza ponendo nelle equazioni precedenti $e=0$,
quindi $p=a=b$, $c=e=0$. Per la parabola sussistono le seguenti relazioni:
\begin{align}
  c &= a,\\
  p &= 2a,
\end{align}
e per l'iperbole:
\begin{align}
  p &= \frac{b^2}{a} = a(e^2-1), \label{eq:semilato-iperbole}\\
  e &= \sqrt{1+\frac{b^2}{a^2}} =
  \sqrt{1+\frac{p}{a}}, \label{eq:eccentricita-iperbole}\\
  c &= \sqrt{a^2+b^2}.
\end{align}
Osserviamo infine che per tutte le coniche risulta $r(\tilde\omega\pm\pi/2)=p$.

Inserendo la~\eqref{eq:rec-semilato} nella~\eqref{eq:eccentricita-ellisse}
possiamo ricavare l'eccentricità di un'orbita ellittica legata ai parametri
fisici del sistema
\begin{equation}
  \label{eq:eccentricita-orbita-ellisse}
  e = \sqrt{1-\frac{p}{a}} = \sqrt{1-\frac{l_0^2}{GM_\textup{T}\mu^2a}}.
\end{equation}
Si noti che, poiché $l_0\neq 0$ per ipotesi, si ha $0\leq e<1$, coerentemente
con la classificazione illustrata prima. Nel caso di orbita iperbolica dobbiamo
invece utilizzare la~\eqref{eq:eccentricita-iperbole}:
\begin{equation}
  \label{eq:eccentricita-orbita-iperbole}
  e = \sqrt{1+\frac{p}{a}} = \sqrt{1+\frac{l_0^2}{GM_\textup{T}\mu^2a}}
\end{equation}
e qui risulta $e>1$.

L'unica conica chiusa è l'ellisse, quindi se uno dei due corpi del sistema è
molto più massivo dell'altro (per esempio il Sole e un pianeta che gli orbita
intorno) abbiamo ricavato la prima legge di Keplero: \emph{I pianeti,
  considerati puntiformi, descrivono orbite ellittiche di cui il Sole occupa uno
  dei fuochi}.

\subsection{Calcolo della velocità e applicazioni astronomiche}
\label{sec:velocita}
\begin{figure}
  \centering
  \input{Immagini/gnuplot/velocita}
  \caption{Andamento qualitativo del modulo quadro della velocità del corpo di
    massa ridotta in funzione dell'angolo $\theta$ nel caso di orbita ellittica,
    per diversi valori dell'eccentricità $e$ e con $a$ fissato. Abbiamo posto
    $\tilde\omega=0$.}
  \label{fig:velocita}
\end{figure}
Calcoliamo il modulo quadro della velocità del corpo fittizio di massa ridotta
durante la sua orbita, data dalla~\eqref{eq:velocita-polare}: $v(\theta)^2 =
\norm{\dot{\bm{r}}(\theta)}^2 = \dot{r}^2(\theta) +
r^2(\theta)\dot{\theta}^2$. Ricordando le formule~\eqref{eq:derivata1-r}
e~\eqref{eq:sol-binet} e che $l_0=\mu r^2\dot{\theta}$ abbiamo:
\begin{subequations}
  \begin{align}
    \dot{r}(\theta) &= -\frac{l_0}{\mu}\toder{u}{\theta} = -\frac{l_0}{p}
    \left(
      -\frac{1}{p}e\sin(\theta-\tilde\omega)
    \right) = \frac{l_0}{\mu p}e\sin(\theta-\tilde\omega), \\
    r(\theta)\dot{\theta} &= \frac{l_0}{\mu}u = \frac{l_0}{\mu
      p}(1+e\cos(\theta-\tilde\omega)).
  \end{align}
\end{subequations}
Per quanto riguarda il moto ellittico, mettendo insieme le relazioni appena trovate, la~\eqref{eq:rec-semilato}
e la~\eqref{eq:orbita} risulta
\begin{equation}
  \label{eq:velocita-ellisse}
  \begin{split}
    v^2(\theta) &= \dot{r}^2(\theta) + r^2(\theta)\dot{\theta}^2 \\
    &= \left(
      \frac{l_0}{\mu p}
    \right)^2(e^2\sin^2(\theta-\tilde\omega) + 1 + 2e\cos(\theta-\tilde\omega) +
    e^2\cos^2(\theta-\tilde\omega)) \\
    &= \left(
      \frac{l_0}{\mu p}
    \right)^2(1+2e\cos(\theta-\tilde\omega)+e^2) \\
    &= \left(
      \frac{l_0}{\mu p}
    \right)^2(2(1+e\cos(\theta-\tilde\omega)) + (e^2-1)) \\
    &= \frac{l_0^2}{\mu^2p}
    \left(
      2\frac{1+e\cos(\theta-\tilde\omega)}{p} - \frac{1-e^2}{p}
    \right) = \frac{l_0^2}{\mu^2 p}
    \left(
      \frac{2}{r(\theta)} - \frac{1}{a}
    \right) \\
    &= \frac{l_0^2}{\mu^2}\frac{GM_\textup{T}\mu^2}{l_0^2}
    \left(
      \frac{2}{r(\theta)} - \frac{1}{a}
    \right) = GM_\textup{T}
    \left(
      \frac{2}{r(\theta)} - \frac{1}{a}
    \right).
  \end{split}
\end{equation}
Questa relazione lega il modulo quadro della velocità alla distanza della
particella fittizia dal centro di massa. Si può osservare in particolare che
$v^2(\theta)$ è proporzionale a $1/r(\theta)$, quindi il modulo della velocità è
massimo quando il corpo di massa ridotta si trova nel periapside e vale
\begin{equation}
  v^2(\tilde\omega) = \frac{GM_\textup{T}}{a}\frac{1+e}{1-e}.
\end{equation}
Inoltre il modulo quadro della velocità sarà minimo quando la distanza dal
centro di massa è massima, cioè all'apoapside
\begin{equation}
  v^2(\tilde\omega+\pi) = \frac{GM_\textup{T}}{a}\frac{1-e}{1+e}.
\end{equation}
Questi risultati sono in accordo con la seconda legge di Keplero. L'andamento di
$v^2$ in funzione della coordinata $\theta$ è riportato nella
figura~\ref{fig:velocita}.

\begin{figure}
  \centering
  \input{Immagini/gnuplot/velocita_los}
  \caption{Andamento qualitativo del modulo quadro della velocità, lungo la
    direzione di vista, del corpo di massa ridotta in funzione dell'angolo
    $\theta$ nel caso di orbita ellittica. Il grafico è stato fatto per gli
    stessi valori di eccentricità utilizzati nella figura~\ref{fig:velocita},
    mantenendo costante il semiasse maggiore. Abbiamo posto $\tilde\omega=0$.}
  \label{fig:velocita-los}
\end{figure}
Nelle osservazioni astronomiche quello che viene effettivamente misurato non è
la velocità data dalla~\eqref{eq:velocita-ellisse} ma la componente lungo la
direzione di vista %TODO: aggiungere immagine come la 47 dello Smart e
                   %completare il discorso!

I calcoli svolti per determinare la velocità nel caso di orbite ellittiche sono
del tutto analoghi a quelli necessari nel caso di orbite paraboliche e
iperboliche, però per le orbite paraboliche $e=1$ quindi
\begin{equation}
  \label{eq:velocita-parabola}
  v^2(\theta) = \frac{2GM_\textup{T}}{r(\theta)},
\end{equation}
mentre per le orbite iperboliche semilato retto ed eccentricità sono legati
dalla~\eqref{eq:semilato-iperbole}, dunque
\begin{equation}
  \label{eq:velocita-iperbole}
  v^2(\theta) = GM_\textup{T}
    \left(
      \frac{2}{r(\theta)} + \frac{1}{a}
    \right).
\end{equation}


\subsection{Energia e classificazione energetica delle orbite}
\label{sec:energia-orbite}

Possiamo ora calcolare l'energia meccanica del sistema costituito dai due corpi:
\begin{equation}
  \label{eq:energia}
  E= T + U = \frac{1}{2}\mu v^2(\theta) - \frac{GM_\textup{T}\mu}{r(\theta)}
  = \frac{1}{2}\mu(\dot{r^2(\theta)} + r^2(\theta)\dot{\theta}^2) -
  \frac{GM_\textup{T}\mu}{r(\theta)}.
\end{equation}
Per le orbite ellittiche sostituiamo al posto della velocità l'espressione data
dalla~\eqref{eq:velocita-ellisse}
\begin{equation}
  \label{eq:energia-ellisse}
  E = \frac{1}{2}GM_\textup{T}\mu
  \left(
    \frac{2}{r(\theta)} - \frac{1}{a}
  \right) - \frac{GM_\textup{T}\mu}{r(\theta)} = -\frac{GM_\textup{T}\mu}{2a}.
\end{equation}
Per le orbite paraboliche usiamo la~\eqref{eq:velocita-parabola}:
\begin{equation}
  E = \frac{1}{2}\frac{2GM_\textup{T}\mu}{r(\theta)} -
  \frac{GM_\textup{T}\mu}{r(\theta)} = 0.
\end{equation}
Infine per le orbite iperboliche usiamo la~\eqref{eq:velocita-iperbole}:
\begin{equation}
    \label{eq:energia-iperbole}
  E = \frac{1}{2}GM_\textup{T}\mu
  \left(
    \frac{2}{r(\theta)} + \frac{1}{a}
  \right) - \frac{GM_\textup{T}\mu}{r(\theta)} = \frac{GM_\textup{T}\mu}{2a}.
\end{equation}
Dunque in tutti i casi l'energia è una costante del moto e dipende solo dalle
masse $m_1$ e $m_2$ dei due corpi e dal semiasse maggiore dell'ellisse descritta
dal loro moto relativo, ma non dall'eccentricità di questa. Naturalmente il
fatto che l'energia si conserva non deve sorprendere perché abbiamo supposto per
ipotesi che il sistema fosse isolato e che l'unica forza agente fosse quella
gravitazionale. %TODO: svolgere esercizio su dE in funzione di da.

Dalla~\eqref{eq:energia} possiamo osservare che il moto relativo è equivalente a
un moto unidimensionale e con un potenziale efficace
\begin{figure} %TODO: scegliere valori migliori per il grafico
  \centering
  % coefficienti che compaiono nel potenziale efficace
\pgfmathsetmacro{\a}{25}
\pgfmathsetmacro{\b}{50}
% funzione potenziale efficace
\pgfmathdeclarefunction{ueff}{1}{\pgfmathparse{\a/#1^2-\b/#1}}
%% valori dell'energia
% energia E_0 = minimo di U_eff, il minimo si ha nel punto r = 2a/b e
% U_eff(2a/b) = -b^2/(4a)
\pgfmathsetmacro{\ezero}{-\b^2/(4*\a)}
\pgfmathsetmacro{\euno}{-20} % energia E_1
\pgfmathsetmacro{\edue}{15} % energia E_2

\draw[->] (0,-30) -- (0,30) node[above] {$U_{\textup{eff}}$}; % asse y
\draw[->] (0,0) node[left] {$O$} -- (8,0) node[right] {$r(\theta)$}; %asse x
% grafico della funzione potenziale efficace
\draw[name path=U] [domain=0.4:8,samples=100] plot (\x,{ueff(\x)});
\draw[dashed,name path={E0}] (0,\ezero) node[left] {$E_0$} -- +(8,0);
\draw[dashed,name path={E1}] (0,\euno) node[left] {$E_1$} -- +(8,0);
\draw[dashed] (0,\edue) node[left] {$E_2$} -- +(8,0);
% individuo il punto di intersezione fra il potenziale e l'energia E_0
\path[name intersections={of=U and {E0},by={A}}];
\draw[loosely dashed] (A) -- +(0,-\ezero) node[above] {$r_0$};
% individuo i punti di intersezione fra il potenziale e l'energia E_1
\path[name intersections={of=U and {E1},by={B,C}}];
\draw[loosely dashed] (B) -- +(0,-\euno) node[above] {$r_{\textup{min}}$}
                      (C) -- +(0,-\euno) node[above] {$r_{\textup{max}}$};

%%% Local Variables:
%%% mode: latex
%%% TeX-master: "../../tesi"
%%% End:

  \caption{Andamento qualitativo del potenziale efficace nel problema di Keplero
    in funzione della distanza dal centro di massa. Il grafico rappresenta la
    funzione $k_1/r^2-k_2/r$, con $k_1$ e $k_2$ costanti scelte
    arbitrariamente.}
  \label{fig:potenziale-efficace}
\end{figure}
\begin{equation}
  U_\textup{eff}(r(\theta)) = \frac{1}{2}\mu r^2(\theta)\dot{\theta}^2 -
  \frac{GM_\textup{T}\mu}{r(\theta)} = \frac{1}{2}\frac{l_0^2}{\mu r^2(\theta)} -
  \frac{GM_\textup{T}\mu}{r(\theta)}.
\end{equation}
Il termine $l_0/(2\mu r^2(\theta))$ è chiamato \emph{potenziale
  centrifugo}. Nella figura~\ref{fig:potenziale-efficace} è riportato il grafico
della curva del potenziale efficace. Per $r=r_0=l_0/(GM_\textup{T}\mu^2)$ si ha
un punto di minimo per $U_\textup{eff}$ e risulta $U_\textup{eff}(r_0) =
-G^2M_\textup{T}^2\mu^3/(2l_0^2)$. La differenza fra l'energia e la curva di
$U_\textup{eff}$ dà $\mu\dot{r}^2/2$, quindi il moto è ammesso solo in quelle
regioni in cui $E-U_\textup{eff}\geq 0$, cioè:
\begin{enumerate}
\item se $E = E_0 = G^2M_\textup{T}^2\mu^3/(2l_0^2)$ risulta $\dot{r}=0$, cioè
  $r(t) = r_0 = \text{costante}$. L'orbita è dunque circolare e il moto
  circolare uniforme con frequenza $\omega_0 = l_0/(\mu r_0^2)$;
\item se $E = E_1 \in \mathopen{]}E_0,0\mathclose{[}$, il moto è limitato fra le
  distanze apsidali $r_\textup{min}$ e $r_\textup{max}$ e l'orbita è un'ellisse;
\item se $E = E_2 \geq 0$ il moto è limitato inferiormente ma non
  superiormente. Se $E_2 = 0$ l'orbita descritta è una parabola, se $E_2>0$ si
  ha un'iperbole.
\end{enumerate}

Inserendo la~\eqref{eq:energia-ellisse}
nella~\eqref{eq:eccentricita-orbita-ellisse} e la~\eqref{eq:energia-iperbole}
nella~\eqref{eq:eccentricita-orbita-iperbole} abbiamo che sia per orbite
ellittiche che per orbite iperboliche sussiste la seguente relazione fra energia
ed eccentricità dell'orbita
\begin{equation}
  e = \sqrt{1+\frac{2El_0^2}{G^2M_\textup{T}^2\mu^3}}.
\end{equation}
Si noti che questa relazione è valida anche per le orbite paraboliche perché in
questo caso l'energia è nulla e l'eccentricità vale $e$.

Il centro di massa esercita un'attrazione sulla particella fittizia di massa
ridotta, tuttavia la presenza del potenziale centrifugo non
nullo\footnote{Ricorda che abbiamo supposto per ipotesi che sia $l_0\neq 0$.}
impedisce che la massa ridotta possa cadere sul centro di massa se il campo
generato è quello gravitazionale. Infatti abbiamo
\begin{equation}
  0\leq \frac{1}{2}\mu\dot{r}^2 = E - U_\textup{eff}(r) = E -
  \frac{1}{2}\frac{l_0^2}{\mu r^2} - \frac{GM_\textup{T}\mu}{r}
\end{equation}
che può essere riscritta come
\begin{equation}
  \frac{1}{2}\frac{l_0^2}{\mu} + GM_\textup{T}\mu r \leq Er^2
\end{equation}
e facendo il limite per $r \to 0$ abbiamo
\begin{equation}
  \frac{1}{2}\frac{l_0^2}{\mu} \leq 0,
\end{equation}
ma questo non è impossibile.

\section{Formalismo lagrangiano}
\label{sec:formalismo-lagrange}



%%% Local Variables: 
%%% mode: latex
%%% TeX-master: "../tesi"
%%% End: 

\chapter{Elementi orbitali}
\label{chap:elementi-orbitali}

\section{Soluzioni dell'equazione di Keplero}
\label{sec:soluzioni}

\subsection{Metodo di Newton~-~Raphson}
\label{sec:newton}

\subsection{Integrali ellittici}
\label{sec:integrali-ellittici}

\subsection{Funzioni di Bessel}
\label{sec:bessel}

%%% Local Variables: 
%%% mode: latex
%%% TeX-master: "../tesi"
%%% End: 

\chapter{Applicazioni astrofisiche}
\label{chap:applicazioni}

\section{Il sistema binario Sgr~A*~-~S2}
\label{sec:sgra}

La terza legge di Keplero permette di determinare la massa di un corpo celeste
se sono noti i parametri orbitali e la massa di un altro corpo con cui
costituisce un sistema binario.

\begin{figure}
  \centering
  \includegraphics[width=7cm]{Immagini/orbite_sgra}
  \caption[Orbite di alcune delle stelle che orbitano intorno al buco nero
  Sgr~A*]{Rappresentazione delle orbite di alcune delle stelle che orbitano
    intorno al buco nero. La figura, tratta da \textcite{2009ApJ...692.1075G}, è
    centrata in Sgr~A*}
  \label{fig:orbite-sgra}
\end{figure}
Si suppone che la regione Sagittarius~A* (Sgr~A*), nel centro della nostra
galassia, sia sede di un buco nero supermassivo, cioè con una massa oltre $10^6$
volte più grande di quella del Sole. Intorno a questo buco nero orbitano
numerose stelle e le orbite di alcune di esse possono essere osservate nella
Figura~\ref{fig:orbite-sgra}. La stella più importante per i nostri scopi è S2
(chiamata a volte S0-2 e di massa circa \SI{15}{\solarmass}), poiché fra le
stelle che orbitano intorno al buco nero è quella che ha il più breve periodo di
rivoluzione, pari a circa $15$ anni, e fra le stelle di questa regione a breve
periodo è la più luminosa, quindi più facile da individuare. Le osservazioni
astronomiche di S2 sono cominciate nel 1992 e da pochi anni ha completato
un'intera rivoluzione a partire da quella data, quindi è una ricca fonte di
informazioni per lo studio del buco nero. L'orbita di S2 intorno al buco nero
può essere considerata con buona approssimazione kepleriana quindi possiamo
utilizzare la~\eqref{eq:terza-legge-keplero} per stimare la massa
$M_\textup{BH}$ del buco nero. \textcite{2008ApJ...689.1044G} hanno studiato il
moto di S2 ricavando i dati riportati nella
Tabella~\ref{tab:parametri-orbitali-S2}.
\begin{table}
  \centering
  \caption[Parametri orbitali della stella S2]{Parametri orbitali della stella
    S2. $R_0$ è la distanza dalla Terra, $P$ è il periodo di rivoluzione, $a$
    è il semiasse maggiore dell'orbita ed $e$ la sua eccentricità,
    $R_\textup{min}$ è la distanza di periapside e $M_{\textup{S}2}$ è la
    massa della stella}
  \label{tab:parametri-orbitali-S2}
  \begin{tabular}{lc}
    \toprule
    Grandezza         & Valore                       \\
    \midrule
    $R_0$             & \SI{7.96}{\kilo\parsec}      \\
    $P$               & \SI{15.86}{\year}            \\
    $a$               & \SI{126.5}{\milli\arcsecond} \\
    $e$               & $0.8970$                     \\
    $R_\textup{min}$  & \SI{0.535}{\milli\parsec}    \\
    $M_{\textup{S}2}$ & circa \SI{15}{\solarmass}    \\
    \bottomrule
  \end{tabular}
\end{table}
Il semiasse maggiore dell'orbita è espresso in millesimi di arcosecondo, questo
valore può essere convertito in parsec, conoscendo la distanza $R_0$ della Terra
dal corpo, con la seguente relazione
\begin{equation}
  a [\si{\parsec}] = \frac{a [\si{\arcsecond}] \cdot R_0 [\si{\parsec}] \cdot
    \pi}{3600 \cdot 180} = \SI{4.88e-3}{\parsec}.
\end{equation}
Poiché $M_\textup{BH}$ è sicuramente molto più grande di $M_{\textup{S}2}$
possiamo porre $M_\textup{T} \approx M_\textup{BH}$
nella~\eqref{eq:terza-legge-keplero} e otteniamo
\begin{equation}
  M_\textup{BH} \approx M_\textup{T} = \frac{4\pi^2a^3}{GP^2} =
  \SI{4.06e6}{\solarmass}.
\end{equation}
In realtà quella calcolata non è esattamente la massa del buco nero ma tutta la
massa che, nel piano dell'orbita di S2, è contenuta nella circonferenza di
raggio $R_\textup{min}$ e centro nel fuoco. Metodi più elaborati per la stima
della massa del buco nero possono essere trovati in
\textcite{2008ApJ...689.1044G} e \textcite{2009ApJ...692.1075G}.

Gli astronomi continuano a studiare il moto di S2 poiché sperano di osservare
delle deviazioni dall'orbita puramente kepleriana previste dalla teoria della
relatività generale le quali permetterebbero di effettuare una stima
indipendente della massa del buco nero.

\section{La funzione di massa. Sistemi binari X}
\label{sec:funzione-massa}

Vediamo ora un nuovo modo per stimare la massa di un corpo celeste sfruttando la
terza legge di Keplero. Sappiamo che
\begin{equation}
  \label{eq:terza-legge-keplero2}
  G(m_1 + m_2)  P^2 = 4\pi^2a^3,
\end{equation}
dove $a$ è il semiasse maggiore dell'ellisse descritta dalla particella
relativa. Per ragioni di similitudine, il rapporto $a_1/a$, con $a_1$ semiasse
maggiore dell'orbita descritta dal corpo di massa $m_1$, è uguale al rapporto
$\norm{\bm{r}_1/\bm{r}}$, cioè dalla~\eqref{eq:r1-nel-cdm}
\begin{equation}
  \label{eq:semiasse-m1}
  a_1 = \frac{\mu}{m_1}a = \frac{m_2}{m_1 + m_2}a.
\end{equation}
Quindi, sostituendo la~\eqref{eq:semiasse-m1}
nella~\eqref{eq:terza-legge-keplero2} abbiamo
\begin{equation}
  GP^2\frac{m_2^3}{(m_1 + m_2)^3} = 4\pi^2a_1^3.
\end{equation}
Nelle osservazioni spettroscopiche non possono essere misurati separatamente il
semiasse maggiore $a_1$ e l'angolo di inclinazione $i$, ma la proiezione di
$a_1$ nel piano del cielo data da $a_1\sin i$. Moltiplicando ambo i membri per
$\sin^3 i$ e portando al secondo membro tutte le quantità misurabili risulta
\begin{equation}
  \label{eq:valore-funzione-massa}
  \frac{(m_2\sin i)^3}{(m_1 + m_2)^2} = \frac{4\pi^2}{GP^2}(a_1\sin i)^3.
\end{equation}
Il primo membro dell'equazione prende il nome di \emph{funzione di massa} per il
corpo $1$
\begin{equation}
  f_1(m_1,m_2,i) = \frac{(m_2\sin i)^3}{(m_1 + m_2)^2}.
\end{equation}
Il valore della funzione di massa è noto quando si misurano il periodo orbitale
$P$ e il semiasse maggiore proiettato $a_1\sin i$. È possibile fare questo, in
particolare misurare $a_1\sin i$, solo nei sistemi binari visuali, quelli cioè
in cui entrambi i corpi sono visibili. Vedremo più avanti come si può calcolare
la funzione di massa negli altri casi. Ragionando in maniera analoga per il
corpo di massa $m_2$ possiamo definire la funzione di massa $f_2$
\begin{equation}
  f_2(m_1,m_2,i) \equiv \frac{(m_1\sin i)^3}{(m_1 + m_2)^2} =
  \frac{4\pi^2}{GP^2}(a_2\sin i)^3,
\end{equation}
con $a_2 = am_1/(m_1+m_2)$ semiasse maggiore dell'ellisse descritta dal corpo di
massa $m_2$. Conoscendo solo le funzioni di massa non è possibile determinare
univocamente le due masse se l'angolo di inclinazione non è noto. Sarà quindi
necessaria un'altra equazione per poter fissare i valori di tutte e tre queste
grandezze. Tuttavia, dato un qualsiasi valore di $m_1$ e $i$, la funzione di
massa $f_1$ fornisce il minimo valore della massa $m_2$. Allo stesso modo, il
valore di $f_2$ è un limite inferiore per la massa del corpo $1$.

Utilizziamo la funzione di massa per stimare la massa di un corpo non visibile
che costituisce insieme alla stella variabile supergigante HD 226868 il sistema
binario
Cygnus-X1.\footnote{I dati riportati di seguito sono presi
  da~\textcite[212]{melia:astrophysics}.} Poiché il corpo non emette nella banda
ottica sicuramente non è una stella. È noto che l'eccentricità del sistema è
molto piccola ($e \lesssim 0.02$), quindi possiamo considerare l'orbita
circolare. Misure basate sull'effetto Doppler forniscono la velocità orbitale
proiettata $v_1$ della stella e risulta $v_1 =
\SI{75}{\kilo\metre\per\second}$. Inoltre le variazioni periodiche del flusso
misurato della stella fornisce il periodo di rotazione $P = \SI{5.6}{\day}$.

Possiamo scrivere la velocità orbitale proiettata come
\begin{equation}
  \label{eq:velocità-proiettata}
  v_1 = \frac{2\pi}{P}a_1\sin i,
\end{equation}
con $a_1$ semiasse maggiore dell'orbita della stella. Inserendo
la~\eqref{eq:velocità-proiettata} nella~\eqref{eq:valore-funzione-massa} abbiamo
\begin{equation}
  f_1(m_1,m_2,i) \equiv \frac{(m_2\sin i)^3}{(m_1 + m_2)^2} = \frac{v_1^3P}{2\pi
    G}.
\end{equation}
Per questo sistema si hanno delle eclissi e, come vedremo nel
paragrafo~\ref{sec:extrasolari}, in questo caso l'angolo di inclinazione vale
$i \simeq \pi/2$, da cui $\sin \simeq 1$. Da osservazioni nella banda ottica si
è trovato
\begin{equation}
  f_1 = \SI{0.252(10)}{\solarmass}.
\end{equation}
Si stima che la stella abbia una massa $m_1 \gtrsim \SI{8.5}{\solarmass}$,
quindi la sua compagna ha una massa
\begin{equation}
  m_2 \gtrsim \SI{4}{\solarmass}.
\end{equation}
Poiché questo valore è maggiore di $\SI{3.2}{\solarmass}$, valore massimo della
massa di una stella di neutroni in rotazione (si
veda~\textcite{1974PhRvL..32..324R}), dobbiamo concludere che il corpo non
visibile del sistema Cygnus-X1 è un buco nero.

\section{I pianeti extrasolari}
\label{sec:extrasolari}

Molti pianeti extrasolari costituiscono, almeno in prima approssimazione, un
sistema binario insieme alla stella intorno alla quale orbitano e dallo studio
delle eclissi della stella dietro al pianeta è possibile ricavare delle
caratteristiche dei due corpi. Qui ci limiteremo a studiare alcune proprietà del
fenomeno dell'eclissi.

\begin{figure}
  \centering
  \begin{tikzpicture}
    % coordinata x del centro del pianeta (uguale al semiasse maggiore perché ho
    % posto il centro della stella nell'origine)
    \pgfmathsetmacro{\a}{8}
    \pgfmathsetmacro{\runo}{2} % raggio stella = 2
    \pgfmathsetmacro{\rdue}{1} % raggio pianeta = 1
    \pgfmathsetmacro{\i}{acos(sqrt(1-\rdue*\rdue/(\a*\a)))} % angolo i

    \coordinate (O1) at (0,0); % centro stella
    \coordinate (O2) at (\a,0); % centro pianeta
    \draw[thick] (O1) circle (\runo); % disco stella
    \node at ($(O1) + (2.3,-1.2)$) {$m_1$};
    \draw[thick] (O2) circle (\rdue); % disco pianeta
    \node at ($(O2) + (-1.1,-1)$) {$m_2$};
    \draw[->] (O1) -- +(0,3) node[above] {$z$}; % asse z
    \draw[->] (O1) -- +(10,0) node[right] {$x$}; % asse x
    % asse x''
    \draw[->] (O1) -- ($10*({cos(\i)},{sin(\i)})$) node[right] {$x''$};
    % angolo di inclinazione
    \draw[dashed] ($(O1) + (0,2.5)$) to[out=0,in=90+\i] node[right=5]
                  {$i_{\textup{min}}$} ($2.5*({cos(\i)},{sin(\i)})$);
    % distanza fra i centri di pianeta e stella
    \draw[<->] ($(O1) + (0,-0.2)$) -- node[fill=white] {$a$} ($(O2)+(0,-0.2)$);
    \draw (O1) -- node[left] {$r_\star$} +($-\runo*({cos(45)},{sin(45)})$);
    \draw (O2) -- node[right] {$r$} +($\rdue*({cos(90+\i)},{sin(90+\i)})$);
  \end{tikzpicture}
  \caption[Angolo di inclinazione minimo sotto il quale un osservatore può
  vedere
  un'eclissi]{Angolo di inclinazione minimo sotto il quale un osservatore può
    vedere un'eclissi. La direzione della linea di vista è l'asse $x''$. Il moto
    della stella e del pianeta si svolge nel piano ortogonale al piano della
    figura}
  \label{fig:minimo-angolo-eclissi}
\end{figure}
Vediamo sotto quali condizioni è possibile per un osservatore esterno vedere
un'eclissi della stella dietro al pianeta studiato. Per semplicità assumiamo che
le orbite dei due corpi siano circolari, quindi $e \approx 0$, che questi siano
perfettamente sferici e che possano essere trascurati gli effetti dovuti
all'atmosfera. Con riferimento alla Figura~\ref{fig:minimo-angolo-eclissi}, il
corpo di massa $m_1$ è la stella, di raggio $r_\star$, e quello di massa $m_2$ è
il pianeta, il cui raggio è $r$. La distanza fra i centri dei due corpi vale
$a$. Se indichiamo con $i \in \mathopen{[}0, \pi/2\mathclose{]}$ l'angolo di
inclinazione sotto il quale l'osservatore vede il sistema, l'angolo minimo che
permette all'osservatore di assistere a un'eclissi è quello rappresentato nella
figura, cioè con l'asse $x''$ passante per il centro della stella e tangente
alla superficie del pianeta. Dunque
\begin{equation}
  \cos(\pi/2 - i) = \frac{\sqrt{a^2 - r^2}}{a} = \sqrt{1 - \frac{r^2}{a^2}}.
\end{equation}
Generalmente si ha $a \gg r_\star \gg r$, quindi $\cos(\pi/2 - i) \approx 1$,
cioè $i \approx \pi/2$.

\begin{figure}
  \centering
  \begin{tikzpicture}[scale=0.6,font=\footnotesize]
    \pgfmathsetmacro{\rdue}{1} % raggio pianeta = 1
    \pgfmathsetmacro{\runo}{4*\rdue} % raggio stella = 4

    \coordinate (O) at (0,0); % centro stella
    \draw (O) circle (\runo); % disco stella
    \draw (O) -- node[right] {$r_\star$} +(0,-\runo);
    \draw (-\runo-\rdue,0) -- node[above] {$r$}
          +($-\rdue*({cos(45)},{sin(45)})$);
    \foreach \x in {-5,-3,-1,1,3,5} % varie posizioni del pianeta
      \draw (\x,0) circle (\rdue);
    \draw (-8,-\runo+1) node[left] {luminosità} -- (-\runo-\rdue,-\runo+1) --
          (-\runo+\rdue,-\runo-1) -- (\runo-1,-\runo-1) --
          (\runo+\rdue,-\runo+1) -- (8,-\runo+1); % curva di luce
    \draw[dashed] (-\runo-\rdue,0) -- (-\runo-\rdue,-\runo-2) node[below]
                  {$t_{\textup{ii}}$};
    \draw[dashed] (-\runo,-\runo) -- (-\runo,-\runo-2) node[below]
                  {$t_{\textup{mi}}$};
    \draw[dashed] (-\runo+\rdue,0) -- (-\runo+\rdue,-\runo-2) node[below]
                  {$t_{\textup{if}}$};
    \draw[dashed] (\runo-\rdue,0) -- (\runo-\rdue,-\runo-2) node[below]
                  {$t_{\textup{ei}}$};
    \draw[dashed] (\runo,-\runo) -- (\runo,-\runo-2) node[below]
                  {$t_{\textup{mf}}$};
    \draw[dashed] (\runo+\rdue,0) -- (\runo+\rdue,-\runo-2) node[below]
                  {$t_{\textup{ef}}$};
    \draw[->] (-8,-\runo-2) -- (8,-\runo-2) node[right] {$t$}; % asse del tempo
  \end{tikzpicture}
  \caption[Transito di un pianeta davanti alla stella compagna nel caso
  $i =
  \pi/2$]{Transito di un pianeta davanti alla stella compagna nel caso $i =
    \pi/2$. Davanti alla stella di raggio $r_\star$, fissa, sono riportate le
    posizioni del pianeta, di raggio $r$, in diversi istanti. La curva sotto lo
    schema mostra l'andamento della luminosità del sistema in funzione del tempo
    $t$}
  \label{fig:schema-transito}
\end{figure}
Nella Figura~\ref{fig:schema-transito} è rappresentata una schematizzazione di
un'eclissi della stella, nel caso in cui $i = \pi/2$. L'osservatore è fisso
davanti alla stella e vede il pianeta che le passa davanti, in vari
istanti. Quando il pianeta non copre la stella la luminosità del sistema è
massima, quando inizia a nasconderla parzialmente (istante di
\emph{ingresso iniziale}, $t_{\textup{ii}}$) la luminosità decresce fino a
raggiungere un minimo nell'istante in cui si trova completamente davanti alla
stella (\emph{ingresso finale}, $t_{\textup{if}}$). La luminosità rimane minima
per tutto il tempo in cui il pianeta si trova davanti alla stella,
successivamente aumenta quando il pianeta esce parzialmente dalla stella
(\emph{egresso iniziale}, $t_{\textup{ei}}$) e ritorna massima non appena il
pianeta non copre più la stella (\emph{egresso finale}, $t_{\textup{ef}}$). Gli
istanti fin qui definiti sono detti \emph{punti di contatto}. Esistono poi i
\emph{punti di mezzo ingresso} che sono quelli in cui la curva di luminosità
raggiunge il valore intermedio fra il massimo e il minimo. In particolare
abbiamo il \emph{mezzo ingresso iniziale} ($t_{\textup{mi}}$) quando la
luminosità sta diminuendo e il \emph{mezzo ingresso finale} ($t_{\textup{mf}}$)
quando la luminosità ricomincia ad aumentare. Definiamo la
\emph{durata dell'eclissi} $\Delta t$ come il tempo che intercorre fra
l'ingresso iniziale e l'egresso finale:
$\Delta t = t_{\textup{ef}} - t_{\textup{ii}}$. La velocità di rivoluzione del
pianeta intorno alla stella è $P/(2\pi a)$, avendo indicato con $P$ il periodo
di rivoluzione del pianeta intorno alla stella, e, sempre nell'approssimazione
$a \gg r_\star \gg r$, lo spazio percorso dal pianeta fra gli istanti
$t_{\textup{ii}}$ e $t_{\textup{ef}}$ è $2(r_\star + r)$. Allora la durata
dell'eclissi è
\begin{equation}
  \Delta t \approx \frac{P}{2\pi a}2(r_\star + r) = \frac{P(r_\star + r)}{\pi
    a}.
\end{equation}

\begin{figure}
  \centering
  \subfloat[][Posizione del pianeta, davanti alla stella, in corrispondenza dei
  punti di contatto\label{fig:transito-i-non-90.a}]
  {\begin{tikzpicture}[scale=0.7,font=\footnotesize]
      \pgfmathsetmacro{\rdue}{1} % raggio pianeta = 1
      \pgfmathsetmacro{\runo}{4*\rdue} % raggio stella = 4
      \pgfmathsetmacro{\acosi}{2.5} % a*cos(i) = 2.5
      % modulo della coordinata x dell'ingresso finale (e dell'egresso finale)
      \pgfmathsetmacro{\xingresso}{sqrt((\runo+\rdue)^2-\acosi^2)}
      % modulo della coordinata x dell'egresso iniziale (e dell'ingrsso finale)
      \pgfmathsetmacro{\xegresso}{sqrt((\runo-\rdue)^2-\acosi^2)}

      \draw (\runo,0) arc (0:180:\runo); % semi-disco stella
      \draw (-6,\acosi) -- (6,\acosi); % asse passante lungo equatori dei pianeti
      \draw (-\runo,0) -- (\runo,0); % diametro stella
      \draw[<->] (0,0) -- node[sloped,above,fill=white] {$a\cos i$} (0,\acosi);
      \draw[->] (-6,-1) -- (6,-1) node[right] {$t$}; % asse del tempo
      \draw (-\xingresso,\acosi) circle (\rdue) % posizione di ingresso iniziale
            (\xingresso,\acosi)  circle (\rdue) % posizione di egresso finale
            (0,0) -- node[sloped,below,fill=white] {$r_\star + r$}
            (-\xingresso,\acosi);
      \draw[dashed] ($-\xingresso*(1,0) + (0,\acosi)$) -- +(0,{-\acosi-1})
                    node[below] {$t_{\textup{ii}}$}
                    ($\xingresso*(1,0) + (0,\acosi)$) -- +(0,{-\acosi-1})
                    node[below] {$t_{\textup{ef}}$};
      \draw (-\xegresso,\acosi) circle (\rdue) % posizione di ingresso finale
            (\xegresso,\acosi)  circle (\rdue) % posizione di egresso iniziale
            (0,0) -- node[sloped,below,fill=white] {$r_\star-r$}
            (\xegresso,\acosi);
      \draw[dashed] ($-\xegresso*(1,0) + (0,\acosi)$) -- +(0,{-\acosi-1})
                    node[below] {$t_{\textup{if}}$}
                    ($\xegresso*(1,0) + (0,\acosi)$) -- +(0,{-\acosi-1})
                    node[below] {$t_{\textup{ei}}$};
    \end{tikzpicture}} \qquad
  \subfloat[][Geometria del sistema\label{fig:transito-i-non-90.b}]
  {\begin{tikzpicture}[scale=0.7,font=\footnotesize]
      \coordinate (O1) at (0,0); % centro stella
      \coordinate (O2) at (8,0); % centro pianeta
      \draw[thick] (O1) circle (2); % disco stella (r=2)
      \node at ($(O1) + (2.3,-1.2)$) {$m_1$};
      \draw[thick] (O2) circle (1); % disco pianeta (r=1)
      \node at ($(O2) + (-1.1,-1)$) {$m_2$};
      \draw[->] (O1) -- (0,3) node[above] {$z$}; % asse z
      \draw[->] (O1) -- (10,0) node[right] {$x$}; % asse x
      \draw[->] (O1) -- (10,1) node[right] {$x''$}; % asse x''
      % retta parallela all'asse x'' e passante per il centro del pianeta
      \draw (0,-0.8) -- (10,0.2);
      \draw[<->] let \n1 = {0.8/10.1},
                     \n2 = {-8/10.1}
                 in
                 (O1) -- node (P) {} (\n1,\n2); % distanza fra le due rette
      \draw[->] (-0.8,0.5) node[above] {$a\cos i$} to[out=-90,in=180] (P);
      \draw[dashed] let \n1 = {atan(0.1)} in
                        ($(O1) + (0,2.5)$) to[out=0,in=90+\n1] node[right=5]
                        {$i$} ($cos(\n1)*(2.5,0) + sin(\n1)*(0,2.5)$);
      \draw (O1) -- node[left] {$r_\star$} +($-2*({cos(45)},{sin(45)})$);
      \draw (O2) -- node[left] {$r$} +($({cos(45)},{-sin(45)})$);
    \end{tikzpicture}}
  \caption{Transito di un pianeta davanti alla stella compagna nel caso
    $i \neq \pi/2$}
  \label{fig:transito-i-non-90}
\end{figure}
Come abbiamo visto, per poter osservare un'eclissi deve aversi $i \approx
\pi/2$. Se l'angolo di inclinazione $i$ non vale esattamente $\pi/2$,
l'osservatore non vede il centro del pianeta passare davanti all'equatore della
stella ma leggermente più spostato, come mostrato nella
Figura~\ref{fig:transito-i-non-90.a}. In questa figura, della stella è
rappresentata solo metà del disco visibile. Per calcolare lo spostamento
apparente del pianeta rispetto all'equatore si può vedere la
Figura~\ref{fig:transito-i-non-90.b}. Da semplici calcoli trigonometrici si
ricava che lo spostamento apparente vale $a \cos i$. Le definizioni dei punti di
contatto e di mezzo ingresso valgono anche per questo caso. Osservando la
Figura~\ref{fig:transito-i-non-90.a} si ricava, sempre usando la trigonometria,
che lo spazio percorso dal pianeta fra gli istanti $t_{\textup{ii}}$ e
$t_{\textup{ef}}$ è circa $2\sqrt{(r_\star + r)^2 - a^2\cos^2 i}$, quindi la
durata dell'eclissi in questo caso è
\begin{equation}
  \Delta t \approx \frac{P}{2\pi a} 2\sqrt{(r_\star + r)^2 - a^2\cos^2 i} =
  \frac{P}{\pi} \sqrt{\left(\frac{r_\star + r}{a}\right)^2 - \cos^2 i}.
\end{equation}
Per $i = \pi/2$ si riottiene il risultato precedente.

\begin{figure}
  \centering
  \subfloat[][$\sqrt{r_\star^2 - r^2} \leq d \leq r_\star +
  r$\label{fig:area-coperta.a}]
  {\begin{tikzpicture}[scale=1.1,font=\footnotesize]
      \coordinate (O1) at (0,0); % centro stella
      \coordinate (O2) at (2.5,0); % centro pianeta
      \draw[name path=S,thick] (O1) circle (2); % disco stella (r=2)
      \draw[name path=P,thick] (O2) circle (1); % disco pianeta (r=1)
      % individuo punti di intersezione fra dischi di pianeta e stella
      \path[name intersections={of=S and P,by={A,B}}];
      \draw[dashed] (O1) -- (A) -- (O2) -- node[below,sloped] {$r$} (B) --
                    node[above,sloped] {$r_\star$}  (O1);
      \draw (A) -- (B);
      \draw[->] ($(B) - (0,0.5)$) node[left,fill=white] {$S_1$} to
                [out=45,in=-80] (1.9,-0.1);
      \draw[->] ($(A) + (0,0.5)$) node[right] {$S_2$}
                to[out=180] (1.7,0.1);
      \draw let \n1 = {acos((1-4+2.5*2.5)/(2*1*2.5))} in
                ($(O2) - cos(\n1)*(0.2,0) + sin(\n1)*(0,0.2)$) arc
                (180-\n1:180+\n1:0.2);
      \draw[->] ($(O2) + (-0.2,0.5)$) node[right] {$\theta_2$}
                to[out=180,in=180] (2.3,0);
      \draw let \n2 = {acos((4-1+2.5*2.5)/(2*2*2.5))} in
                ($(O1) + cos(\n2)*(0.3,0) - sin(\n2)*(0,0.3)$) arc
                (-\n2:\n2:0.3);
      \draw[->] ($(O1) + (0.5,-0.5)$) node[below] {$\theta_1$} to[in=0] (0.3,0);
    \end{tikzpicture}} \qquad
  \subfloat[][$r_\star - r \leq d <
  \sqrt{r_\star^2 - r^2}$\label{fig:area-coperta.b}]
  {\begin{tikzpicture}[scale=1.1,font=\footnotesize]
      \coordinate (O1) at (0,0); % centro stella
      \coordinate (O2) at (1.5,0); % centro pianeta
      \draw[name path=S,thick] (O1) circle (2); % disco stella (r=2)
      \draw[name path=P,thick] (O2) circle (1); % disco pianeta (r=1)
      % individuo punti di intersezione fra dischi di pianeta e stella
      \path[name intersections={of=S and P,by={A,B}}];
      \draw[dashed] (O1) -- (A) -- (O2) -- node[below,sloped] {$r$} (B) --
                    node[above,sloped] {$r_\star$} (O1);
      \draw (A) -- (B);
      \draw[->] ($(B) - (0,0.5)$) node[left,fill=white] {$S_1$} to
                [out=0,in=-60] (1.9,-0.1);
      \draw[->] ($(A) + (0,0.5)$) node[right] {$S_2$} to[out=180] (1.2,0.3);
      \draw let \n1 = {acos((1-4+1.5*1.5)/(2*1*1.5))} in
                ($(O2) - cos(\n1)*(0.2,0) + sin(\n1)*(0,0.2)$) arc
                (180-\n1:180+\n1:0.2);
      \draw[->] ($(O2) + (1,0.6)$) node[right] {$\theta_2$}
                to[out=200,in=0] (1.3,0);
      \draw let \n2 = {acos((4-1+1.5*1.5)/(2*2*1.5))} in
            ($(O1) + cos(\n2)*(0.3,0) - sin(\n2)*(0,0.3)$) arc (-\n2:\n2:0.3);
      \draw[->] ($(O1) + (0.4,-0.4)$) node[below] {$\theta_1$} to[in=0] (0.3,0);
    \end{tikzpicture}}
  \caption{Schema della sovrapposizione fra i dischi di una stella e di un
    pianeta compagno durante un transito}
  \label{fig:area-coperta}
\end{figure}
Calcoliamo l'area del disco della stella coperta dal pianeta durante
l'eclissi. La distanza che prenderemo in considerazione non sarà la distanza
reale $a$ fra i due corpi ma quella proiettata $d$ nel piano del cielo
dell'osservatore e introdotta nel paragrafo~\ref{sec:geometria-sistema}. Se in
un certo istante di tempo si ha $d > r_\star + r$, i due corpi non sono
sovrapposti nel cielo dell'osservatore quindi non si sta verificando
l'eclissi. Se invece $d \leq r_\star + r$ il pianeta copre, almeno parzialmente,
la stella. Finché $\sqrt{r_\star^2 - r^2} \leq d \leq r_\star + r$, l'area di
sovrapposizione è la somma delle due aree $S_1$ e $S_2$ della
Figura~\ref{fig:area-coperta.a}. Per il teorema del coseno, l'angolo $\theta_1$
è dato da
\begin{equation}
  \theta_1 = 2 \arccos \frac{r_\star^2 - r^2 + d^2}{2r_\star d}
\end{equation}
e analogamente per l'angolo $\theta_2$ abbiamo
\begin{equation}
  \label{eq:theta2}
  \theta_2 = 2 \arccos \frac{r^2 - r_\star^2 + d^2}{2rd}.
\end{equation}
Le aree $S_1$ e $S_2$ valgono rispettivamente
\begin{subequations}
  \begin{align}
    S_1 &= \frac{\theta_1}{2\pi}\pi r_\star^2 - \frac{r_\star^2}{2}\sin\theta_1
    = \frac{r_\star^2}{2}(\theta_1 - \sin\theta_1), \\
    S_2 &= \frac{\theta_2}{2\pi}\pi r^2 - \frac{r^2}{2}\sin\theta_2 =
    \frac{r^2}{2}(\theta_2 - \sin\theta_2).
  \end{align}
\end{subequations}
Dunque in questo caso l'area di sovrapposizione $\delta A$ vale
\begin{equation}
  \delta A = S_1 + S_2 = \frac{r_\star^2}{2}(\theta_1 - \sin\theta_1) +
  \frac{r^2}{2}(\theta_2 - \sin\theta_2).
\end{equation}
Se $r_\star - r \leq d < \sqrt{r_\star^2 - r^2}$, l'area di sovrapposizione
$\delta A$ è sempre data da $S_1 + S_2$ (si veda la
Figura~\ref{fig:area-coperta.b}) e risulta $\theta_2 > \pi$ (si osservi
l'equazione~\eqref{eq:theta2}). In questo caso, quindi, $S_2$ è la somma del
settore circolare di angolo $\theta_2$ e del triangolo isoscele con vertice nel
centro del disco del pianeta e angolo al vertice di $2\pi - \theta_2$. Così
l'area di sovrapposizione diventa
\begin{equation}
  \begin{split}
    \delta A &= S_1 + S_2 = \frac{r_\star^2}{2}(\theta_1 - \sin\theta_1) +
    \frac{r^2}{2}(\theta_2 + \sin(2\pi -\theta_2)) \\
    &= \frac{r_\star^2}{2}(\theta_1 - \sin\theta_1) + \frac{r^2}{2}(\theta_2 -
    \sin\theta_2).
  \end{split}
\end{equation}
Infine se $d < r_\star - r$ significa che il pianeta si trova completamente
davanti alla stella rispetto all'osservatore quindi l'area di sovrapposizione è
uguale all'area del disco del pianeta, cioè
\begin{equation}
  \delta A = \pi r^2.
\end{equation}
Riepilogando
\begin{equation}
  \delta A =
  \begin{dcases}
    0 & \text{se $d > r_\star + r$}, \\
    \frac{r_\star^2}{2}(\theta_1 - \sin\theta_1) + \frac{r^2}{2}(\theta_2 -
    \sin\theta_2) & \text{se $r_\star - r \leq d \leq r_\star + r$}, \\
    \pi r^2 & \text{se $d < r_\star - r$}.
  \end{dcases}
\end{equation}

Vediamo ora in dettaglio come varia la luminosità della sistema che giunge
all'osservatore nelle varie fasi dell'eclissi. Poiché nel sistema in esame solo
la stella emette luce propria, la luminosità che prendiamo in considerazione è
esattamente quella della stella. La luminosità propria $L_\star$ di una stella,
cioè l'energia emessa sotto forma di luce nell'unità di tempo, ha le dimensioni
di una potenza, quindi nel sistema internazionale si misura in \si{\watt},
mentre nel sistema CGS si userà \si{erg\per \second}. Questa grandezza è spesso
misurata anche in unità di luminosità solari ($L_\odot$). Definiamo il
\emph{flusso superficiale} $F_\star$ come
\begin{equation}
    F_\star = \frac{L_\star}{4\pi r_\star^2},
\end{equation}
avendo assunto che il disco stellare appaia uniformemente illuminato
all'osservatore. Abbiamo anche $L_\star = 4\pi r_\star^2 F_\star$. Definiamo
inoltre il \emph{flusso a Terra} $F_{\textup{T}}$ come
\begin{equation}
  F_{\textup{T}} = \frac{L_\star}{4\pi R_0^2} = F_\star \frac{r_\star^2}{R_0^2},
\end{equation}
in cui $R_0$ è la distanza fra l'osservatore e la stella. Le definizioni fin qui
date sono valide se la stella è direttamente visibile dall'osservatore senza
alcun ostacolo. Durante l'eclissi, però, il disco della stella sarà parzialmente
coperta dal pianeta, quindi bisognerà moltiplicare le quantità sopra definite
per la frazione di disco della stella visibile all'osservatore, vale a dire
$(A - \delta A)/A$, con $A= \pi r_\star^2$. Dunque
\begin{align}
  F_\star &= \frac{L_\star}{4\pi r_\star^2} \frac{A - \delta A}{A} =
  \frac{L_\star}{4\pi r_\star^2} \left( 1 - \frac{\delta A}{A} \right), \\
  F_{\textup{T}} &= \frac{L_\star}{4\pi R_0^2} \frac{A - \delta A}{A} =
  \frac{L_\star}{4\pi R_0^2} \left( 1 - \frac{\delta A}{A} \right).
\end{align}
Al di fuori della fase di eclissi $\delta A = 0$, quindi riotteniamo le
definizioni precedenti. Più in generale possiamo definire una funzione
\emph{flusso} $F$ come
\begin{equation}
  F = \frac{f}{4} L_\star \left( 1 - \frac{\delta A}{A} \right),
\end{equation}
con $f$ fattore geometrico che cambia a seconda della quantità che si vuole
misurare. Nel caso del flusso superficiale avremo
\begin{equation}
  f \equiv f_\star = \frac{1}{\pi r_\star^2},
\end{equation}
mentre per il flusso a Terra
\begin{equation}
  f \equiv f_{\textup{T}} = \frac{1}{\pi R_0^2}.
\end{equation}

Ho scritto un programma in linguaggio C che simula un'eclissi. Il programma
genera un file con i valori, per ogni istante di tempo in cui viene svolta al
simulazione, delle coordinate della particella relativa nel sistema di
riferimento iniziale e del piano del cielo dell'osservatore, delle coordinate
della stella e del pianeta nel piano del cielo, della distanza fra i due corpi
proiettata nel piano del cielo e del flusso luminoso della stella. Il codice
sorgente del programma è riportato
nell'appendice~\ref{cha:simulazione-eclissi}. Nelle
Figure~\ref{fig:sim-ecl-piano-cielo}, \ref{fig:sim-ecl-distanza-proiettata} e
\ref{fig:sim-ecl-flusso} sono rappresentati i risultati della simulazione. Ho
inserito i seguenti valori: $r_\star = \SI{e12}{\centi\metre}$,
$r = \SI{1e11}{\centi\metre}$, $\text{massa stella} = m_1 = \SI{1}{\solarmass}$,
$\text{massa pianeta} = m_2 = \SI{0.02}{\solarmass}$,
$a = \SI{e13}{\centi\metre}$, $e = 0.8$, $\phi = \SI{30}{\degree}$,
$i = \SI{85}{\degree}$. Ho posto uguale a $1$ la luminosità intrinseca della
stella in modo che il grafico del flusso avesse come massimo proprio $1$. Il
rapporto $r_\star / r$ è stato scelto volutamente molto più vicino all'unità
rispetto agli usuali rapporti fra i raggi di stelle e pianeti in maniera da
rendere più accentuata la curva del flusso. La fase indica la quantità $(t -
t_0)/P$, con $t_0$ istante iniziale scelto per la simulazione.

\begin{figure}
  \centering
  \input{programmi/piano_cielo}
  \caption[Orbite della stella e del pianeta nel piano del cielo]{Orbite della
    stella e del pianeta nel piano del cielo ottenute dalla simulazione discussa
    in questo paragrafo}
  \label{fig:sim-ecl-piano-cielo}
\end{figure}
\begin{figure}
  \input{programmi/distanza_proiettata}
  \caption[Distanza proiettata in funzione della fase]{Andamento della distanza
    proiettata, normalizzata alla somma dei raggi della stella e del pianeta, in
    funzione della fase ottenuto dalla simulazione discussa in questo paragrafo}
  \label{fig:sim-ecl-distanza-proiettata}
\end{figure}
\begin{figure}
  \input{programmi/flusso}
  \caption[Flusso luminoso, normalizzato a $1$, di un sistema binario a
  eclissi]{Andamento del flusso luminoso, normalizzato a $1$, ottenuto dalla
    simulazione discussa in questo paragrafo}
  \label{fig:sim-ecl-flusso}
\end{figure}

\section{Possibili sviluppi futuri}
\label{sec:sviluppi-futuri}

Nel paragrafo precedente abbiamo studiato i transiti di pianeti extrasolari
davanti alle stelle compagne ma adottando alcune approssimazioni. Questo lavoro
può essere migliorato prendendo in considerazione i fatti che i due corpi non
sono perfettamente sferici ma possono essere deformati, per esempio a causa del
potenziale gravitazionale, la luminosità della stella non è uniforme su tutta la
superficie ma generalmente si ha un \emph{limb darkening}, cioè un oscuramento
al bordo, inoltre gli effetti dovuti all'atmosfera stellare non sono
trascurabili. Tutto ciò comporta che la curva di luce di un'eclissi non ha la
forma squadrata della Figura~\ref{fig:sim-ecl-flusso} ma risulta più smussata e
irregolare.

%%% Local Variables:
%%% mode: latex
%%% TeX-master: "../tesi"
%%% End:


\backmatter{}
\clearpage{}
\phantomsection{}
\nocite{*} % TODO: citare i libri
\addcontentsline{toc}{chapter}{Riferimenti bibliografici}
\printbibliography[title=Riferimenti bibliografici]

\end{document}
